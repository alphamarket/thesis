\documentclass[12pt,a4paper]{article}

\include{preamble}

% define a new command for ease of use for rendering reference
\newcommand{\renderref}[1] { \begingroup \let\clearpage\relax \include{#1} \endgroup }

\begin{document}
\title{
    \includegraphics[width=0.25\textwidth]{iut}\\\vspace{20pt}
    بهبود یادگیری مشارکتی در سیستمهای چندعاملی\\
\lr{Cooperative learning improvement in multi-agent systems}
}
\author{
داریوش حسن\ پور آده\\
۹۳۰۸۱۶۴
}
\date{استاد راهنما: دکتر پالهنگ}
\maketitle

\قسمت*{خلاصه}
اهمیت مشارکت و کار جمعی را نمی‌توان نادیده گرفت و می‌توان گفت راز بزرگ موفقیت انسان و حیوانات همین اجتماعی
بودن است. محققان هوش مصنوعی که همیشه دنبال رسیدن به یک هوشمندی در ماشین بوده‌اند، در این راه از انسان و حیوانات
الگوبرداری‌های زیادی داشته‌اند که یکی از این الگوبرداری‌ها شاخه سیستم‌های چندعاملی را در هوش مصنوعی به وجود آورده
است. مثل جوامع انسانی و حیوانی که گاهی مشارکتی و گاهی رقابتی به زندگی می‌پردازند، سیستم‌های چندعاملی نیز در
زیرشاخه‌های مشارکتی و رقابتی موردپژوهش قرار می‌گیرند. در این پژوهش که با هدف بهبود یادگیری مشارکتی در سیستم‌های
چندعاملی انجام خواهد شد تکیه‌ بر روی مشارکت و مهمتر از آن یادگیری مشارکتی در سیستم‌های چندعاملی است.
باوجود عمر کمی که سیستم‌های چندعاملی و مخصوصاً یادگیری مشارکتی دارد پژوهش‌های فراوانی در این زمینه صورت
گرفته که می‌توان شروع آن را با \مرجع{whitehead1991complexity, whitehead1991study} دانست که با هدف افزایش سرعت یادگیری در یادگیری تقویتی ارائه شد. \بند
بعد از آن‌هم در تکمیل این کار پژوهش‌هایی صورت گرفته که معمولاً تقلید از یادگیری انسان است. ارائه ایده‌هایی چون تقلید، پند دهی، خبرگی
و تخته‌سیاه که مفید هم بوده‌اند. نکته مهمی که در کارهای گذشته وجود دارد تخصیص یک وزن برای تمام جدول(که میزان پاداش تجمعی تجربه شده برای تمامی عمل-حالت‌ها در این جدول ذخیره می‌شوند و معیاری برای تصمیم گیری عامل در محیط می‌باشد) است در صورتی که ممکن است قسمتی از جدولی که وزن بالایی می‌گیرد به درستی مقدار نگرفته باشد و باعث یادگیری نادرست شود. برای حل
این مشکل می‌توان برای هر سلول وزن در نظر گرفت که به دلیل مشکل بودن محاسبه وزن زمان‌گیر خواهد شد. راه‌حل دیگر یک
تقسیم‌بندی بهینه از جدول است. هدف نهایی این پژوهش نیز رسیدن به یک تقسیم‌بندی بهینه و روشی برای محاسبه وزن هر بخش
جدول است. در پایان آزمایشهایی انجام خواهد شد که نشان‌دهنده نکات مثبت و منفی کار باشد. آزمایش‌ها در دو فاز انجام خواهد
گرفت. فاز اول که مقایسه بین این روش و روشهای مشابه است و فاز دوم آزمایشهای انفرادی، که نشان‌دهنده مثبت و منفی بودن
اثر کار است.\بند
\vspace{1em}
\noindent\textbf{کلمات‌کلیدی:} ۱- سیستم‌های چندعاملی ۲- یادگیری مشارکتی ۳- یادگیری تقویتی

\قسمت*{موضوع کلی و زمینه اصلی تحقیق}
یادگیری یکی از بزرگترین دغدغه‌های رشته‌های مختلفی چون روانشناسی، هوش مصنوعی و ... است که در هوش مصنوعی
منجر به شاخه یادگیری ماشین شده است. میشل در \مرجع{mitchell1997machine} تعریفی برای یادگیری آورده است.\بند
تعریف ۱: یادگیری یعنی اینکه یک عامل رفتارش را بر اساس تجربیات گذشته‌اش، عوض کند.
معمولاً در تحقیقات یادگیری ماشین از یادگیری انسان و حیوانات الگوبرداری می‌شود که این تقلیدها نتایج خوبی به دنبال
داشته است؛ یکی از بزرگترین الگوهایی که می‌توان از زندگی انسان و حیوانات برداشت زندگی گروهی و جمعی آنهاست که
گاه به مشارکت و گاه به رقابت می‌انجامد. در این پژوهش تمرکز بر روی مشارکت در زندگی جمعی است که ما را به مفهوم
یادگیری مشارکتی می‌رساند. در سالهای اخیر تحقیقات زیادی درزمینه یادگیری مشارکتی انجام گرفته است در \مرجع{panait2005cooperative} تعریفی برای
سیستم چندعاملی مشارکتی ارائه شده است.\بند
تعریف ۲: سیستمی که در آن چند عامل برای رسیدن به یک هدف و یا انجام یک وظیفه‌ی مشترک باهم همکاری می‌کنند،
سیستم چندعاملی مشارکتی نامیده می‌شود.\بند
همچنین اثبات شده که یادگیری مشارکتی از سرعت و قدرت بالاتری نسبت به یادگیری انفرادی برخوردار است. معمولاً در
یادگیری مشارکتی از یادگیری تقویتی و معروف‌ترین عضو این خانواده یعنی یادگیری Q استفاده می‌شود و نتایج نیز قابل‌توجه بوده
و نشان می‌دهد که این ترکیب می‌تواند یک روش یادگیری قدرتمند باشد. می‌توان بزرگترین دغدغه‌های این شاخه را به‌صورت
سؤالاتی مطرح نمود.
\begin{enumerate}\setlength\itemsep{-.5em}
\فقره چه زمان باید اطلاعات عامل‌ها منتقل شود؟
\فقره اطلاعات باید به کدام عامل فرستاده شود و یا از کدام عامل دریافت شود؟
\فقره چه میزان اطلاعات باید ارسال شود؟
\فقره چگونه این اطلاعات با اطلاعات خود عامل ترکیب شود؟
\end{enumerate}
\newpage\noindent
رسیدن به پاسخ سؤالات بالا می‌تواند یادگیری را تا حد زیادی بهینه نماید. در اکثر کارهایی که انجام شده به موضوع نحوه
ترکیب بحث شده و معمولاً کل جدول ارسال می‌شود.

\قسمت*{مروری بر کارها و نتایج گذشته}
با وجود اینکه یادگیری مشارکتی زمینه نسبتاً جدیدی در یادگیری ماشین محسوب می‌شود کارهای فراوانی در این زمینه انجام
شده است و همانگونه که گفته شد معمولاً الگوبرداری از رفتار انسانها یا حیوانات بوده است. می‌توان گفت اولین تلاش در زمینه
یادگیری مشارکتی برمیگردد به \مرجع{whitehead1991complexity,whitehead1991study} که مکانیزمی برای افزایش سرعت یادگیری تقویتی با مشارکت بین عاملها ارائه نمودند.
بعد از آن تان در \مرجع{tan1993multi} سه روش انتقال برای یادگیری مشارکتی مطرح نموده و به مقایسه عملکرد چند عامل هنگام استفاده از یادگیری
مشارکتی و بدون استفاده از یادگیری مشارکتی پرداخته که نشان می‌دهد درصورتیکه مشارکت به‌درستی پیاده‌سازی شود می‌تواند
برای کلیه عامل‌ها مفید باشد. روشی که در این مقاله استفاده شده بر اساس میانگین‌گیری از جداول Q عامل‌ها است که نام روش را
نیز SA\زیرنویس{Simple Averaging} نهاده است. بعد از آن برنجی و همکاران در \مرجع{berenji1999cooperation} به مزایای روش یادگیری مشارکتی پرداخته است.\بند
برنجی در مقاله بعد خود
نیز مفهوم یادگیری مشترک را مطرح نمودند که در آن سیاست بین عاملها مشترک بوده و عاملها می‌توانند آن را بروز رسانی
نمایند \مرجع{berenji2000advantages} سپس تووافن در \مرجع{trevarthen2004learning} به تقلید که یکی از ابزارهای یادگیری انسان است اشاره نموده است. بعد از آن یادگیری جمعی
که آن نیز برگرفته از انسان است توسط گارلند در \مرجع{garland1996multiagent} مطرح شد که با الهام از ایده شناخت توزیع شده در علوم اجتماعی شکل گرفته
است. سپس ایده پند دادن عاملهای باتجربه و پند گرفتن عاملهای تازه‌کار نونس در \مرجع{nunes2003cooperative} آمد؛ و بعد احمدآبادی یادگیری بر مبنای
خبرگی را در \مرجع{ahmadabadi2000expertness} ارائه نمودند، بعد از آن نیز یادگیری تخته‌سیاه در \مرجع{carver1994evolution, mcmanus1996design} ارائه شده است.
در سال ۱۳۹۲ خانم پاکیزه با در نظرگرفتن چندین معیار(مانند میزان پاداش جمع شده، میزان جریمه‌های اخذ شده و غیره توسط عامل) برای عوامل در مرحله ادغام دانش این عوامل، نشان دادند که این معیارها می‌توانند به صورت ضرایب ترکیب موثر برای ترکیب دانش عامل‌ها استفاده شوند \مرجع{pakizeh2013multi}. چند سال بعد در سال ۱۳۹۵ آقای میرزایی با معرفی معیار کوتاه‌ترین مسیر تجربه شده یادگیری مشترک را تسریع بخشیدند و نشان دادند که این معیار کوتاه‌ترین مسیر تجریه شده مزایای بیشتری نسبت به معیارهای خانم پاکیزه دارد و می‌تواند نتایج بهتری بدست بدهد \مرجع{mohammad2015speedup}.

\قسمت*{ضرورت انجام، دیدگاه و اهداف تحقیق پیشنهادی}
معمولاً یادگیری تقویتی را با یادگیری Q می‌شناسیم که مهمترین عضو خانواده یادگیری تقویتی است. اصلی‌ترین رکن در
یادگیری Q جدولی با همین نام است که یادگیر باید بتواند در طول یادگیری به‌درستی این جدول را تکمیل نماید و معمولاً روشهایی
که بر مبنای یادگیری تقویتی ارائه شده با هدف تسریع در بروز رسانی این جدول بوده است. تقریباً می‌توان گفت دلیل به وجود
آمدن یادگیری تقویتی نیز همین بود و با این ایده که عاملها بتوانند با مشارکت زودتر جدول خود را کامل کنند شروع به کار نمود
اما چالشهایی وجود داشته که باعث تولید ایده‌های فراوانی در این زمینه شده است.\بند
روش‌هایی که تا به حال ارائه‌ شده‌اند برای تمام جدول ضریبی تخصیص می‌دهند. درنهایت مجموع این جداول با ضرایبی که
جمع یک دارند محاسبه می‌شود. معمولاً روش‌های ارائه‌ شده برای تولید همین ضرایب بوده است؛ اما اینکه برای تمام جدول یک
مقدار در نظر گرفته شود می‌تواند گاهی نتیجه منفی داشته و مقادیر جدول را نادرست نماید. همچنین در نظر گرفتن و محاسبه‌ی معیاری برای تک‌تک موقعیت/عمل‌ها می‌تواند هزینه‌ی زمانی و مکانی زیادی به سیستم یادگیرنده تحمیل کند.\بند
تحقیقات خانم پاکیزه و آقای میرزایی \مرجع{mohammad2015speedup, pakizeh2013multi} یک مبادله‌ای بین محاسبه‌ی ضرایب تاثیر کلی و جزئی درنظر نگرفته‌اند، بدین معنی که یا از اطلاعات کلی عامل‌ها برای ارائه‌ی یک معیاری استفاده می‌کنند با به محاسبه‌ی معیاری برای تک‌تک حالت/عمل‌ها می‌پردازند، همچنین در هنگام ترکیب دانش خاصیت غیرافزایشی بودن دانش‌های عوامل را که در ماهیت مساله وجود دارد را در نظر نگرفته‌اند \مرجع{torra2014non}. همچنین تاثیر استفاده از دیگر روش‌های اکتشاف چون اپسیلون-حریصانه\زیرنویس{\lr{$\epsilon$-greedy}} بررسی نشده است.\بند

در این پژوهش به بررسی امکان ارائه‌ی محاسباتی نرم برای معیار‌ها خواهیم پرداخت، با این هدف که سعی شود تعادلی بین کلی و جزئی نگری به عملکرد عامل‌ها در هنگام ادغام دانش‌های آن‌ها برقرار شود. همچنین تاثیر دیگر روش‌های انتخاب عمل را در ترکیب با معیار‌های ارائه شده را مورد بررسی قرار  می‌دهیم و دستاورد‌های این پژوهش را با در نظر گرفتن ماهیت غیرافزایشی بودن ذات مساله ارائه خواهیم داد.
\vspace{-.5em}
\قسمت*{نحوه ارزیابی دستاوردهای تحقیق}
برای آزمایش روشهایی که در یادگیری مشارکتی ارائه می‌شود محیط‌های زیادی وجود دارد که دو مورد از مهمترین آن‌ها
پلکان «مارپیچ» و «صید و صیاد» است که هر یک بر معیارهایی تأکید دارند. پلکان مارپیچ محیطی ساده و ایستا است و صید و صیاد یک
محیط پیچیده و پویا است. خود آزمایش‌ها نیز به دو دسته مقایسه با روشهای قبلی و آزمونهای انفرادی تقسیم خواهد شد. تا هم
بتوان به مقایسه با روشهای قبلی پرداخت و هم نکات مثبت کار را نمایش داد.\بند
\vspace{-.5em}
\قسمت*{جدول زمانبندی تحقیق}
\vspace{-1em}
\begin{table}[h!]
\centering
\begin{tabular}{p{8cm}|c|c}
فعالیت & زمان\ شروع & زمان\ خاتمه
\\\hline
 مطالعه مقالات و شناخت کامل موضوع & - & اواسط آبان ۹۵
\\\hline
پیاده‌سازی & اواسط آبان ۹۵ & اواخر آذر ۹۵
\\\hline
تجزیه تحلیل و آزمایش & اواخر آذر ۹۵ & اواخر دی ۹۵
\\\hline
نگارش پایان‌نامه و آمادگی جهت دفاع & اواخر دی ۹۵ & اواخر اسفند ۹۵
\\\hline
دفاع & \multicolumn{2}{c}{فروردین ۹۶}
\end{tabular}
\end{table}

\قسمت*{فهرست مراجع اصلی}
%\nocite{*}
\renderref{reference}
\end{document}
