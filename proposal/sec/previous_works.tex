%!TeX TS-program = XeLaTeX
%!TeX root = thesis.proposal.tex
\قسمت{مرور کارها و نتایج قبلی}
چهارپره با استفاده از سخت\ افزارهای مناسب، باید بتواند خطرات تصادف احتمالی را با اشیا ناشناس متحرک یا ایستا تشخیص بدهد و با اجرای مانورهای مناسب از برخورد با مانع تشخیص داده شده،‌ اجتناب کند. برای اجتناب از مانع ساده\ ترین روش\ ها روش هندسی می\ باشد. در\
\cite{park2008uav}
یک روش هندسی برای تشخیص برخورد\زیرنویس{\lr{Conflict Detection}} معرفی شده است. که با محاسبه\ ی «نزدیک\ ترین نقطه\ ی برخورد\
\زیرنویس{\lr{Point of Closest Approach}}\
»(\lr{PCA}) بدترین رخداد ممکن بین دو پرنده(دو پهپاد) ارزیابی می\ شود. اگر خطر برخورد تشخیص داده شود، از \lr{PCA} جهت محاسبه\ ی بردار سرعت مرجع دو پرنده استفاده می\ شود تا فاصله\ ی مناسب از یکدیگر حفظ گردد. بنابراین ارتباط بین دو پرنده از مفروضات روش می\ باشد -- که در اکثر مواقع امکان این امر وجود ندارد.\بند
در
\cite{khatib1986real}
موانع توسط میدان نیروهای دافعه\زیرنویس{\lr{Repulsive Force Field}} و هدف توسط میدان نیروهای جاذبه علامت گذاری می\ شود بدین\ گونه ربات می\ تواند یک مسیر بدون-تصادم را با محاسبه\ ی بردار حرکتی از میدان\ های علامت گذاری شده بدست آورد. که این روش در
\cite{kandil2010collision}
بدین صورت بهبود یافته است که فقط موانع علامت\ گذاری می\ شود و از علامت\ گذاری اهداف بنابه این دلیل که فقط می\ خواهیم از موانع اجتناب کنیم، چشم\ پوشی می\ شود.\بند
در روش دیگری\
\cite{gardiner2011collision}
نیاز به داشتن مدل شبکه\ ای\فوتنت{Grid Model} از محیط پهباد دارد که از الگوریتم\ های جستجوی مانند \lr{A*} گراف جهت اجتناب از مانع استفاده می\ کند -- که این روش بنابه این نیاز که حتما باید مدل شبکه\ ای محیط در اختیار باشد در بسیاری از کاربردهای دنیای واقعی نمی\ تواند پیاده\ سازی شود. همچنین در راستای اجتناب از مانع از الگوریتم\ های تکاملی استفاده شده است\
\cite{rathbun2002evolution}.
هرگاه مانعی تشخیص داده شد، مسیر اولیه به چندین قسمت تقسیم می\ شود که بعد از چندین جهش تصادفی آن قسمت\ ها بهترین جهش\ ها برای مشخص کردن مسیر بدون-تصادم اختیار می\ شود.