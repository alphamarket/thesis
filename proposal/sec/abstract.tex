%!TeX TS-program = XeLaTeX
%!TeX root = thesis.proposal.tex

\قسمت*{خلاصه}
\iffalse
چهارپره\زیرنویس{\lr{Quadrotor, Quadcopter}} از دسته پهباد\زیرنویس{\lr{Unmanned Aerial Vehicle (UAV)}}های چندپروانه\ ای می\ باشد که توسط ۴ عدد پره\زیرنویس{\lr{Propeller}} به پرواز درآورده می\ شود. کنترل چهارپره توسط سرعت\ های چرخش ۴ پره\ ی آن صورت می\ گیرد که جهت خنثی کردن نیروی گشتاوری تحمیلی توسط پره\ ها به بدنه\ ی چهارپره ۲ عدد از پره\ ها در جهت عقربه\ های ساعت و ۲ عدد دیگر در خلاف جهت عقربه\ های ساعت می\ چرخند. و در کل با تنظیم مناسب سرعت\ های چرخش هریک از این پره\ ها در جهت\ های چرخشی از پیش تعیین آن\ ها می\ توان جهت و سرعت حرکت چهارپره را کنترل کرد.\بند
\fi
یکی از خطرات پرواز که همیشه از زمان امکان پرواز برای بشر تا به\ کنون وجود داشته، خطر تصادف با اشیا در حین پرواز می\ باشد.
برای حداکثر کردن امنیت پهپادها در حین پرواز، پهپادها نیاز به یک سامانه\ ای جهت حداکثر ساختن احتمال عدم\ تصادف در حین پرواز دارند. برای همین وجود یک سامانه اجتناب از برخورد در پهپادها کاملا ضروری به نظر می\ آید. از آنجایی که ربات\ های چهارپره در دسته\ ی پهپادها قرار می\ گیرند بنا به کاربرد\ های روزافزون چهارپره\ ها در زمینه\ های امداد و نجات، تفریحی و تجاری\مرجع{wiki:UAV} پیاده\ سازی یک سامانه\ ی اجتناب از مانع متناسب با ساختار چهارپره، موضوع تحقیق جالبی به نظر می\ رسد.\\\\
\تاکید{کلمات\ کلیدی:} ۱- پهپاد ۲- پرواز خودکار ۳- امنیت پرواز ۴- اجتناب از موانع