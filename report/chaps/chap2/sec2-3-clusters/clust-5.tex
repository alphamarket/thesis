% uav
\بخش{\جام در پهپادها}
تا به این قسمت پژوهش‌های انجام شده در زمینه‌ی \جام در حالت کلی(بدون درنظرگرفتن و طبقه‌بندی براساس ربات مورد تحقیق) در زمینه‌ها و روش‌های متعددی معرفی شد؛ در این قسمت به پژوهش‌های انجام شده بروی انواع پهپاد متمرکز می‌شویم. زیرا دینامیک و کنترل پهپادهای به‌مراتب پیچیده‌تر از دیگر ربات‌ها می‌باشند و همچنین محدوده‌ی حسگرهای مورد استفاده این گونه از ربات‌ها به نوع پهپاد، سرعت پرواز و میزان قابلیت پردازش‌های برخطی که ربات می‌تواند بروی سیستم‌های خود انجام دهد بستگی دارد. زیرا که به عنوان مثال در ربات‌های زمینی چهارچرخ این امکان وجود دارد ربات در زمان پردازش کردن اطلاعات حسگرهای خود بدون اینکه تعادل خود را از دست دهد به راحتی توقف کرده و بعد از تصمیم‌گیری در مورد مسیر حرکت به ادامه‌ی حرکت بپردازد، ولی همچنین امکانی در اکثر پهپادها وجود ندارد یا اگر هم داشته باشد از نظر توان مصرفی و کنترل بسیار هزینه‌بر است. لذا در این قسمت به دلیل ارتباط با ربات هدف این پژوهش صرفا به مرور پژوهش‌های انجام شده بروی پهپادها در حالت کلی متمرکز خواهیم شد.\بند