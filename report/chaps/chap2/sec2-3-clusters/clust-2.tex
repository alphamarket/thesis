% nn
\بخش{شبکه‌های عصبی مصنوعی در \جام}
در \سال{2009} در طی پژوهشی\مرجع{singh2009intelligent, parhi2009real} از سه عدد شبکه‌های عصبی برای حل مشکل \جام با استفاده کرده‌اند که بعدها در \سال{2011} توسط پژوهشی دیگر\مرجع{singh2011path} بهبود یافت. در این روش سه عدد شبکه‌ی عصبی ۴ لایه‌ای برای اهداف جستجوی هدف، \جام و دنبال کردن دیوار استفاده کردند، برای آموزش هرکدام از این شبکه‌های عصبی داده‌های آموزشی متفاوتی در نظر گرفته شد. داده‌های ورودی این شبکه‌ها فاصله تا هدف و اطلاعات دریافتی از سنسورها هستند که شامل فاصله سنجیده شده توسط سونارها از ۴ جهت اصلی ربات و خروجی این شبکه‌ها نیز زاویه‌ی فرمان ربات می‌باشد.\بند
رساله‌ای در \سال{2013} به بررسی این مساله پرداخت که «چگونه می‌توان یک معماری شبکه عصبی ارائه داد که برای هرنوع از ربات‌ها با هر تعداد و نوع از سنسورها سازگار باشد؟\مرجع{dezfoulian2013generalized}» زیرا که پرواضح است تمامی کارهایی که در رابطه با راهبری ربات‌ها با استفاده از شبکه‌های عصبی در صورتی که نوع یا تعداد سنسورهای متصل به ربات تغییر کنن(همانند روش‌های پیشنهادی در \مرجع{singh2009intelligent, parhi2009real, chi2011obstacle, singh2011path})، علاوه بر ساختار شبکه‌های عصبی باید کل داده‌های آموزشی نیز به‌روز رسانی شوند. در این رساله با فرض اینکه داده‌های سنسورها دوبعدی هستند با استفاده از شبکه‌های \مق{PCNN} و استخراج ویژگی \مق{PCA} ابعاد داده‌های ورودی را عادی‌سازی\زیرنویس{Normalization} می‌کند سپس با استفاده از شبکه‌های عصبی دستورات کنترلی به عنوان خروجی می‌دهد.\بند
در پژوهشی\مرجع{duguleana2016neural} که در \سال{2016} صورت گرفته با استفاده از الگوریتم یادگیری تقویتی \مق{Q-Learning} و شبکه‌های عصبی به تشخیص و اجتناب از موانع ثابت و متحرک پرداخته است. در پژوهش که بروی یک ربات چهارچرخ آکرمن\زیرنویس{Ackermann} انجام شد، با ترکیب استفاده‌ی یادگیری تقویتی و شبکه‌ی عصبی سیستمی خود-یادگیر\زیرنویس{Self-learning} بجهت \جام ارائه دادند. در این روش از جدول \مق{Q} برای ذخیره‌سازی حالات و اعمالی که ربات در طی مسیر از نقطه‌ی شروع تا خاتمه انجام می‌دهد و در هر گام بعد از بروز رسانی جدول \مق{Q} به بروز رسانی وزن‌های شبکه با توجه به ورودی و خروجی جدول \مق{Q} می‌پردازد و در نهایت شبکه‌ی عصبی یادگرفته شده معادل با جدول یادگرفته شده \مق{Q} می‌شود و در صورت شکست ربات(برخورد با مانع) مقادیر جدول \مق{Q} با استفاده از شبکه‌ی عصبی یادگرفته شده بروز رسانی می‌شود.\بند