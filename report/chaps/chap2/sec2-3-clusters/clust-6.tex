% controls/apps
\بخش{دیگر فعالیت‌ها و کاربردهای \جام}
در این قسمت در مورد مسائل کلی که در دیگر قسمت‌های این فصل نمی‌گنجند صبحت خواهد شد. مباحث این قسمت شامل دیگر روش‌های معرفی شده در سال‌های اخیر برای حل مساله‌ی \جام که در زمینه‌ی پژوهشی این تحقیق نیستند، کاربردهای\زیرنویس{Applications} \جام در مسائل روزمره چون کمک به افراد با معلولیت‌های جسمی که مجبور به استفاده از صندلی چرخ‌دار\زیرنویس{Wheelchair} و سیستم‌هایی برای کمک به افراد نابینا و غیره معرفی خواهند شد که نشان‌دهنده‌ از اهمیت و بروز بودن مساله‌ی \جام در دنیای مدرن امروز می‌دهد.\بند
در طول مطالعاتی که برای تهیه‌ی این فصل صورت گرفت به مقالاتی برخورد شد که به مساله‌ی \جام در انواع ربات‌ها از بعد کنترل حمله شده است
\cite{geng2013obstacle, dong2015obstacle, yao2015real, yang2016nonlinear}.
در این مقالات با دست بردن در روابط کنترلی مربوط به ربات مورد پژوهش، سعی کرده‌اند که سهم خود را در مهار کردن چالش \جام ایفا کنند. با توجه به اینکه این مقالات خارج از حوضه‌ی دانش این تحقیق بوده و صرفا به جهت جامعیت دادن به مطالب مندرج در این فصل آورده شده است، برداشت‌هایی که در مطالعه این مقالات بدست آمده است روند \جام در زمینه‌ی کنترل در حالت کلی به این اصل برمی‌گردد که فرض شده است که اطلاعاتی از پیش تعریف شده از مانع موجود در محیط که بسته به نوع ربات و کنترل آن حداقل شامل اطلاعات موقعیتی مانع می‌باشد، در اختیار سیستم هست. سپس با دخیل دادن این اطلاعات در روابط کنترلی ربات سعی شده است که بصورت اتوماتیک در سطح کنترل از موانع اجتناب گردد.\بند
از طرف دیگر \جام علاوه بر علم رباتیک و نظامی کاربردهای غیر نظامی مدرن نیز در طی سالیان اخیر پیدا کرده است. به عنوان مثال در صنعت قایق‌رانی سیستم‌های \جام به کمک ملوانان آمده که زمان‌های استراحت خود را با آرامش خاطر سپری کنند\مرجع{stelzer2010reactive, bandyophadyay2010simple, heidarsson2011obstacle}.
در \سال{2010} الگوریتم میدان پتانسیل به یاری افراد بروی صندلی‌های چرخ‌دار شتافته است\مرجع{seki2010real}. در این تحقیق با استفاده از حسگرهای فاصله‌سنج در اطراف صندلی به فرد معلول نشسته بر صندلی چرخ‌دار کرده‌اند که همانند آنچه که در شکل \ref{fig:wheelchair_app} آمده است، از راهروهای پیچ در پیچ که کنترل و حرکت در آن‌ها مشکل می‌باشد به صورت خودکار عبور کنند. پژوهش مشابه دیگری توسط پتری و همکاران در همین سال با ترفندی تفاوت در همین زمینه‌ صندلی چرخ‌دار صورت گرفت\مرجع{petry2010shared}.
بار دیگر در \سال{2011} فناوری‌های \جام به کمک صنعت صندلی‌های چرخ‌دار آمد که به کمک بینایی و یادگیری ماشین به یاری افرادی که از آسیب‌های ادراکی\زیرنویس{\مق{Cognitive impairment}} برای گذر از معابر شلوغ شتافته است\مرجع{viswanathan2011navigation}. همچنین در این تحقیق علمی-کاربردی از بینایی ماشین برای یافتن‌مسیر\زیرنویس{Wayfinding} جهت کمک به افراد با اختلالات ادارکی و حافظه‌ای استفاده کرده است. یک بررسی، به کاربرد \جام در یاری رساندن به افراد نابینا یا کم‌بینا در \سال{2010} پرداخته است\مرجع{dakopoulos2010wearable}، حالت کلی تکنیک‌هایی که در این بررسی آورده شده این است که با استفاده از حسگر‌هایی(سونار، دوربین‌های استریو و ...) که محیط اطراف را می‌سنجند و بعد از شناسایی اشیا موجود در مسیر حرکت که احتمال برخورد وجود دارد، به یک وسیله‌ای(یک دستگاه سوتی در گوش مخاطب، جهت‌دهی به چرخ‌های راهبر یا ارسال سیگنال لرزشی قسمت‌های متفاوت بدن فرد و ...) موقعیت نسبی شی به فرد و دستورات لازم به جهت رفع مانع اطلاع رسانی می‌شود.\بند

\fig[.35]{wheelchair_app}{از نتایج پژوهش‌های صورت گرفته در \جام\hspace{-3pt}، الگوریتم میدان پتانسیل کمک کرده است که صندلی‌های چرخ‌دار در راهروهای پیچ در پیچ به صورت خودکار حرکت کنند.}