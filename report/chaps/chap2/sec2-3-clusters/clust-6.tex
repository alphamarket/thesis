% controls/apps
\بخش{دیگر فعالیت‌ها در \جام}
در این قسمت در مورد مسائل کلی که در دیگر قسمت‌های این فصل نمی‌گنجند صبحت خواهد شد. مباحث این قسمت شامل دیگر روش‌های معرفی شده در سال‌های اخیر برای حل مساله‌ی \جام که در زمینه‌ی پژوهشی این تحقیق نیستند، کاربردهای\زیرنویس{Applications} \جام در مسائل روزمره چون کمک به افراد با معلولیت‌های جسمی که مجبور به استفاده از صندلی چرخ‌دار\زیرنویس{Wheelchair} و سیستم‌هایی برای کمک به افراد نابینا و غیره معرفی خواهند شد که نشان‌دهنده‌ از اهمیت و بروز بودن مساله‌ی \جام در دنیای مدرن امروز می‌دهد.\بند
در طول مطالعاتی که برای تهیه‌ی این فصل صورت گرفت به مقالاتی برخورد شد که به مساله‌ی \جام در انواع ربات‌ها از بعد کنترل حمله شده است
\cite{geng2013obstacle, dong2015obstacle, yao2015real, yang2016nonlinear}.
در این مقالات با دست بردن در روابط کنترلی مربوط به ربات مورد پژوهش، سعی کرده‌اند که سهم خود را در مهار کردن چالش \جام ایفا کنند. با توجه به اینکه این مقالات خارج از حوضه‌ی دانش این تحقیق بوده و صرفا به جهت جامعیت دادن به مطالب مندرج در این فصل آورده شده است، برداشت‌هایی که در مطالعه این مقالات بدست آمده است روند \جام در زمینه‌ی کنترل در حالت کلی به این اصل برمی‌گردد که فرض شده است که اطلاعاتی از پیش تعریف شده از مانع موجود در محیط که بسته به نوع ربات و کنترل آن حداقل شامل اطلاعات موقعیتی مانع می‌باشد، در اختیار سیستم هست. سپس با دخیل دادن این اطلاعات در روابط کنترلی ربات سعی شده است که بصورت اتوماتیک در سطح کنترل از موانع اجتناب گردد.\بند
