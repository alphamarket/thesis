% fuzzy
\بخش{سیستم‌های فازی در \جام}
در سال‌های اخیر کاربرد سیستم‌های فازی در \جام نیز نسبتا فعالیت‌هایی بجهت تحقیق بوده و از این بُعد در پژوهش‌ها نگاه‌های متفاوتی به مساله‌ی \جام شده است. پژوهشی روسو و همکاران در \سال{2010} به معرفی سیستمی فازی با استفاده از حسگرهای مادون‌قرمز به \جام پرداختند\مرجع{rusu2010obstacle}. در این پژوهش سیستمی با استراتژی واکنش‌گرا\زیرنویس{\مق{Reactive Strategy}} برمبنای کنترل قواعد فازی\زیرنویس{\مق{Rule-based}} که به وسیله‌ی اطلاعات دریافتی از حسگرهای مادون‌قرمز تغزیه می‌شود، ارائه شد. کنترل کننده‌ی منطق فازی این سیستم اطلاعات دریافتی از ۳ عدد سنسور متصل به ربات(که یک ربات دیفرانسیلی می‌باشد) به عنوان ورودی گرفته و سرعت هریک از چرخ‌ها را به عنوان خروجی برمی‌گرداند. به دلیل محاسباتی دو تابع عضویت\زیرنویس{\مق{Membership function}} «نزدیک» و «دور» برای فازی کردن مقدار ورودی‌ها مورد استفاده واقع شد. همچنین ۷ عدد تابع عضویت برای فازی کردن سرعت موتورها مورد استفاده واقع شده است. بعد از تعیین این توابع عضویت برای ورودی‌ها و خروجی‌ها با استفاده از ۸ قانون کنترلی نوشته به‌صورت فازی و استفاده از عملگر \مق{$min-max$} با غیرفازی کردن\زیرنویس{Defuzzification} خروجی سرعت موتورها، نهایتا اقدام به کنترل ربات و \جام کرده است.\بند

در همان سال در پژوهشی دیگر، از روش \مق{Neuro-Fuzzy} برای یادگیری و بهبود کنترل ربات برای \جام استفاده شده است\مرجع{dutta2010obstacle}. در این روش قوانین فازی که توسط برخی از مسیرهایی برای \جام که توسط عامل انسانی به ربات ارائه می‌شود، توسط سیستم \مق{Neuro-Fuzzy} یادگرفته می‌شود. این سیستم در طی روند یادگیری قواعد مربوط به نحوه‌ی \جام در مسیرهای ارائه شده و همچنین توابع عضویت را استخراج می‌کند. در این روش عامل انسانی در سناریو‌های مختلف اقدام به هدایت ربات کرده و اطلاعات حسگرها به عنوان ورودی شبکه و زاویه‌ی فرمان متناظر با هر ورودی به عنوان خروجی شبکه برای آموزش داده می‌شود.\بند

در تلاشی دیگر کیم و همکاران در \سال{2015} راه‌حل دیگری برای مساله‌ی \جام ارائه شد که این بار استفاده‌ از شبکه‌های عصبی با داده‌های ورودی فازی نوع-۲\زیرنویس{Type-2 Fuzzy} پیشنهاد شد که بهبودی به روش قبلی ارائه شده در \مرجع{kim2007unifying} که از توابع عضویت\زیرنویس{Membership functions} نوع-۱ در ورودی‌های شبکه استفاده می‌کرد\مرجع{kim2015obstacle}. در این روش داده‌های ورودی کریسپ\زیرنویس{Crisp} به توابع عضویت فازی نوع-۲ داده می‌شوند و به ازای هر بعد از داده ۳ خروجی «مقدار عادی» توابع عضویت(همان امیدریاضی در هر نقطه از دامنه‌ی تابع)، «حداکثر مقدار» و «حداقل مقدار» به عنوان ابعاد ورودی جدید به شبکه داده می‌شوند. در این روش نشان داده شده است که استفاده از توابع عضویت نوع-۲ برای مدیریت کردن شرایط غیر مطمئن و ناشناخته بهتر از توابع نوع-۱ عمل می‌کنند. در شکل \ref{fig:it1fnn} شبکه‌ی عصبی با توابع نوع-۱ آورده شده است که برای ربات‌های فوتبالیست مورد استفاده واقع شد؛ در نسخه توابع نوع-۲ این شبکه توابع ورودی شبکه از نوع-۱ به نوع-۲ تغییر پیدا کرده‌اند.\بند

\fig[0.25]{it1fnn}{شبکه‌ی عصبی با توابع فازی عضویت نوع-۱، به جهت بدست آوردن دستورات کنترلی اطلاعات ورودی شبکه قبل از پردازش به توابع فازی نوع-۱ داده شده و سپس به شبکه داده می‌شوند.}