% visual
\بخش{بینایی ماشین در \جام}
پژوهشی\مرجع{barry2015pushbroom} در \سال{2015} به ارائه‌ی الگوریتمی سریع برای شناسایی و \جام در پهپادهایی با سرعت پرواز بالا ارائه داد. ایده‌ای که این مقاله داده جالب است و مساله‌ی استخراج نقشه‌ی عدم‌شباهت\زیرنویس{\مق{Disparity Map}} و تطبیق بلوک\زیرنویس{\مق{Block Matching}} را به جستجو میان اعماق تعریف کرده است. حال با محدود کردن جستجوی میزان جابجایی بلوک‌ها، می‌توان فقط به شناسایی اشیایی که در یک فاصله‌ی معین قراردارند پرداخت و به ازای در نظر نگرفتن اشیایی که در فاصله‌ای غیر از این قرار دارند،میتوان سرعت الگوریتم را بصورت توانی افزایش داد. در نهایت با استفاده از ادومتری پهپاد و اطلاعات تجمعی حاصل از این تطبیق الگوهای محدود می‌توان اطلاعات فاصله‌ی پیکسل‌هایی که از قبل از و فاصله‌ی دور شناسایی شده‌اند را بازسازی کند.\بند
در تحقیقی که در \سال{2016} بروی راهبری مبتنی بر تصاویر استریو\مرجع{van2016persistent} صورت گرفت که تلاشی در راستای یادگیری خود-مختاری ربات‌های پرنده با رویکرد راهبری تصویری\زیرنویس{\مق{Visual Navigation}} به جهت \جام می‌باشد. در این تحقیق با استفاده از دو دوربین استریو تصاویر، نقشه‌ی عدم‌شباهت قالب\زیرنویس{Frame}‌های این دو دوربین را بدست می‌آورند، سپس با استفاده از یک تخمین‌زن نقشه‌ی عدم شباهت بروی تصویر سمت چپ مدلی را یادگرفته و بعد از گذرانده شدن از فیلتری به واحد تصمیم‌گیری ارسال می‌گردد. در حین یادگیری تخمین‌‌زن نقشه مشغول به یادگیری می‌باشد ولی بعد از دوره‌ی یادگیری فقط با استفاده از تصاویر دوربین سمت چپ و تخمین‌زن به تصمیم‌گیری می‌پردازد و فقط در صورتی که تخمین‌زن در انجام وظیفه‌ی خود شکست بخورد و نتواند تخمینی معتبر ارائه دهد با استفاده از تصاویر استریو از تصادف جلوگیری به عمل می‌آید؛ با این روش از سربار محاسباتی‌ای که هربار توسط پردازش تصاویر استریو به عمل می‌آید جلوگیری می‌شود.