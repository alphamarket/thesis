\قسمت{مرور کارهای پیشین}
مساله‌ی چالش برانگیز اجتناب از مانع از قدیم تا به کنون یکی از مسائلی بوده که توجهات زیادی را به خود جلب کرده است بطوری که از \سال{2010} تا به کنون چندصد مقاله علمی در این رابطه منتشر شده است؛ در این پژوهش ما علاوه بر درنظرداشتن روش‌های قدیمی که موفقیت‌آمیز بودن آن‌ها در طی زمان ثابت شده است، روش‌های نوین را نیز مورد بررسی قرارداده‌ایم. از آنجایی که این مسائله حجم قابل توجهی تحقیق را به خود تخصیص داده است ما بجهت اینکه این پژوهش به‌روز باشد در این قسمت فقط به مروری خلاصه از آنچه که از \سال{2010} تا به کنون منتشر شده است بسنده می‌کنیم.\بند
در این قسمت به بررسی روش‌های متعددی که بجهت اجتناب از مانع صورت گرفته است خواهیم پرداخت که در ابتدا الگوریتم‌های پایه این شاخه از رباتیک را مرور خواهیم کرد سپس به بررسی روش‌های مختلفی چون کنترل، شبکه‌های عصبی مصنوعی، پردازش تصویر و بینایی ماشین در اجتناب از مانع خواهیم پرداخت؛ همچنین به مروری بر روش‌های اجتناب مانع پیاده‌سازی شده در ربات‌های زمینی و پهپادها در حالت کلی خواهیم پرداخت و در نهایت با معرفی کارهایی که در ربات‌های چندپره به صورت انحصاری و همچنین از دیگر کاربردهای الگوریتم‌های اجتناب از مانع در دنیای مدرن به این بخش خاتمه خواهیم داد. لازم به ذکر است که تمامی الگوریتم‌هایی که در این قسمت معرفی و مرور می‌گردد جز الگوریتم‌های بلادرنگ می‌باشند زیرا که ماهیت \جام این پژوهش نیازمند روش‌‌های بلادرنگ می‌باشد به همین جهت از بررسی روش‌های برون ‌خطی\زیرنویس{Offline} خودداری می‌کنیم.

\bgroup

\def\clustdir{sec2-3-clusters}
\renewcommand{\بخش}{\subsection}
\newcommand{\زیربخش}{\subsubsection}

\foreach \ClustNumber in {1, ..., 100} {
	\IfFileExists{\curdir/\clustdir/clust-\ClustNumber.tex} { \subimport{\clustdir/}{clust-\ClustNumber.tex}\par } { }
}

\egroup