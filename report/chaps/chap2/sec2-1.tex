\قسمت{مقدمه}
\قسمت{تاریخچه پرواز و پرنده‌های بدون سرنشین}
از دیرباز رویای پرواز در ذهن انسان‌ها جا باز کرده بود آسمان محلی مقدسی بود که استوره‌های باستان از آن به زمین می‌آمدند، لذا آرزوی توانایی پرواز یکی از کهن‌ترین آرزوهای آدمی می‌باشد؛ در حدود ۴۰۰ سال ق.م. مردمان چین با اختراع کایت\زیرنویس{Kite} که می‌توانست پرواز کند به آتش این رویا دامن زده شد و در آن موقع کایت به عنوان یک وسیله مقدس برای مراسم‌های مذهبی نگاه می‌شد. بعد از گذشت سالیان دراز لئوناردو داوینچی در \سال{1480} اولین مطالعه را بروی ماهیت پرواز انجام داد که این مطالعه شامل بیش از ۱۰۰ نقشه و تئوری پرواز بود. در \سال{1783} اولین بالن هوای گرم توسط برادران منتگولفیر\زیرنویس{\مق{Joseph and Jacques Montgolfier}} ارائه شد. همچنین اولین گلایدر به همت آقای کی‌لی\زیرنویس{\مق{George Cayley}} در یک دوره ۵۰ ساله در بین سال‌های \تاریخ{1799} و \تاریخ{1850} اختراع شد و بهبود پیدا کرد. در \سال{1891} یک مهندس آلمانی\زیرنویس{\مق{Otto Lilienthal}} روی ایرودینامیک و طراحی گلایدرها مطالعه کرد و اولین فردی بود که توانست گلایدری را طراحی کند که می‌توانست یک انسان را در مسافت‌های طولانی حمل کند. در همان سال آقای لنگلی\زیرنویس{\مق{Samuel P. Langley}} متوجه شد که به نیرو جهت پرواز انسان نیاز هست و مدلی را ارائه داد که دارای موتور بخار بود توانست ۳/۴ مایل را قبل اینکه سوختش تمام شود حرکت کند\مرجع{nasa:hist}.\بند

جنگ‌ها در کنار ویرانگری‌هایی که از خود پشت سر می‌گذارند همیشه باعث تکامل و جهش عمل بشری بوده‌اند؛ در جنگ‌های جهانی(بخصوص جنگ جهانی دوم) نوآوری‌های زیادی در زمینه‌ی علوم هواوفضا و رباتیک بدست آمد. اولین بار در اواخر جنگ جهانی اول بود که یک هواپیمای بدون سرنشین اختراع شد که توسط یک سامانه‌ی رادیویی کنترل می‌شد. در میانه‌ی جنگ‌های جهانی(سال‌های \تاریخ{1927} تا \تاریخ{1929}) اولین موشک کوروز(شکل \ref{fig:early_cruise_missile}) که بصورت یک هواپیمای تک-باله ساخته شد که از روی یک کشتی جنگی پرتاب و توسط خلبان خودکار هدایت می‌شد. موفقیت‌آمیز بود ساخت این موشک باعث شد که چند سال بعد هواپیماهای بدون سرنشین و کنترل کننده‌ی رادیویی در \سال{1930} ساخته شود. در دوره طی جنگ جهانی دوم نیروی دریایی ایالات متحده آمریکا شروع به انجام آزمایشاتی در زمینه‌ی هواپیما‌های رادیوکنترلی در دهه‌‌ی \تاریخ{1930} کرد که نهایتا منجر به ساخت هواپیمای بدون سرنشین \مق{Curtiss N2C-2} شد که به صورت کنترل از راه‌دور از یک هواپیمای دیگر کنترل می‌شد که عنوان یک سامانه‌ی ضد هوایی به خدمت گرفته شد. در همین دوران ایالات متحده آمریکا تلاش کرد دستاوردهای خود را در زمینه‌ی هواپیماهای بدون سرنشین کنترل شونده از راه دور را بروی بمب افکن‌های \مق{B-17 Flying Fortress} و \مق{B-24 Liberator} خود به اجرا در بیاورد که نهایتا منجر به شکست و از دست رفتن شمار زیادی از بمب افکن‌ها شد. هواپیمای \مق{TDN-1} یک هواپیمایی بدون سرنشین بود که در \سال{1940} ساخته شد که می‌توانست یک بمب ۱۰۰۰ پوندی(حدودا ۴۵۰ کیلوگرم) را به پرواز درآورده و به هدف بزند\مرجع{wiki:hist_uav}.\بند


\lfig{larynx}{موشک کروز اولیه به نام \مق{RAE Larynx}}{fig:early_cruise_missile}
\fig{Curtiss_N2C-2_Naval_Aviation_Museum}{هواپیمای \مق{Curtiss N2C-2} کنترل شونده از راه‌دور که در توسط ایالات متحده آمریکا در \سال{1938} ساخته شد.}

در تاریخچه‌ی هواپیماهای بدون سرنشین تا قبل از جنگ سرد به دلیل نبود تکنولوژی‌های مدرن امروزی جنس هواپیماها از جنس موتور، پیستون و گازوییل بودند و ارتباط کنترلی آن‌های بصورت رادیویی بود و معمولا دارای خلبان خودکار نبوده و در صورت وجود چنین سامانه‌ای، سیستمی بسیار ساده داشته و ادومتری آن‌های صرفا بر مبنای قطب‌نما، میزان سرعت و مدت زمان حرکت بود. در دوران جنگ سرد و بعد از آن بود که جهش‌های بزرگ در تکنولوژی‌های ساخت هواپیماهای بدون سرنشین ایجاد شد.\بند