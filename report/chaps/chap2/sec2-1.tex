\قسمت{مقدمه}
در سال‌های گذشته پژوهش‌های فراوانی در سیستم‌های چند عامله انجام‌شده است که در این پژوهش‌ها محققان سعی داشتند مزایای کار گروهی در انسان را در رایانه نیز ایجاد نمایند. یکی از قابلیت‌های عامل‌های هوشمند که می‌تواند باکار گروه سریع‌تر و بهتر شود موضوع یادگیری است که در این زمینه هم کارهایی انجام‌شده که معمولاً الگوبرداری از عامل‌های انسانی بوده است. همان‌طور که می‌دانیم انسان تنها از یک مکانیزم دررسیدن به یادگیری بهره نمی‌برد، عامل‌های انسانی با تقلید از عامل‌هایی که دارای اطلاعات بیشتری هستند توانسته‌اند یادگیری خود را بهبود دهند؛ عامل‌های انسانی در شرایط بحرانی زندگی از عامل‌های باتجربه‌تر پند می‌گیرند، عامل‌های انسانی در مراتبی از خبرگی قرار دارند؛ همه این موارد الگوهایی مناسب بوده که توانسته یادگیری در سیستم‌های چندعاملی را بهبود بخشد.
اما می‌توان کارهایی که در یادگیری مشارکتی انجام می‌شود را به دسته‌هایی تقسیم کرد، هر دسته از پژوهش‌های انجام‌شده در این رشته سعی در رفع یک یا چند چالش از چالش‌های این رشته داشته‌اند. پژوهش پیش رو را می‌توان از دسته پژوهش‌های یادگیری مشارکتی دانست که سعی در حل مشکل ترکیب داده‌های عامل‌ها دارند که روش‌های ارائه‌شده در این فصل نیز روش‌هایی هستند که در تقسیم داده‌های یادگیری مشارکتی فعالیت کرده‌اند.
پیچیدگی ترکیب داده‌های عامل به این دلیل است که معیار مناسبی جهت مشخص کردن داده‌ی درست وجود ندارد. در بسیاری از کارهایی که در این فصل ارائه خواهد شد در ترکیب داده‌ها معیار جایگزینی معرفی‌شده و آن معیاری جهت نمایش برتری عامل است. ایده این جایگذاری ازآنجاست که عاملی که از برتری برخوردار باشد داده‌های بهتری نیز نسبت به عامل‌های دیگر خواهد داشت.