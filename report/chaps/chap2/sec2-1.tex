\قسمت{مقدمه}
«پرنده هدایت‌پذیر از دور یا به اختصار پَهپاد که به آن وسیله هوایی بدون سرنشین نیز گفته می‌شود، نوعی وسیله هوایی هدایت‌پذیر از راه دور است.» تعریفی که از پهپاد در ویکی‌پدیا شده است\مرجع{wiki:drone}. پهپاد به دو دسته کنترل شونده از راه دور توسط عامل انسانی و به صورت کاملا خودکار و برنامه‌ریزی شده می‌شوند. تاریخچه به وجود آمدن پهپادهای مدرن ریشه نظامی داشته و در ماموریت‌های نظامی که برای انسان خطیر یا خسته کننده بودند استفاده می‌شد. به جهت پیشرفت روزافزون تکنولوژی‌های ساخت پهپاد، اکنون شاهد کاربردهای غیرنظامی آن‌ها هستیم. راهبری پهپادها همانند سایر ربات‌ها دارای خطراتی هستند که مهمترین آن‌ها خطر برخورد با موانع موجود در مسیر هست که در مورد پهپادها غالبا منجر به از دست رفتن کنترل، سقوط و از بین رفتن ربات می‌شود. از اینجا هست که نیاز به ارائه روش‌های اجتناب از مانع برخط\زیرنویس{Online} در پهپادها ضروری به‌نظر می‌رسد. از میان روش‌های اجتناب از مانع روش حسگر-مبنا\زیرنویس{\مق{Sensor-based}} در زمینه‌ی ربات‌های هوایی استفاده می‌شود زیرا علاوه بر دینامیک پویا و غیرخطی پهپادها که هم‌بستگی شدیدی با متفییرهای محیطی(همانند سرعت جریان، تراکم هوا و غیره.) دارد تغییرات محیط خارجی نیز از پویایی بالایی برخوردار است. روش‌های دیگری همانند طرح‌ریزی سراسری\مرجع{thesis:izadi} نیز به جهت اجتناب از مانع وجود دارد ولی به دلیل آنکه این روش در صنعت هوایی به دلایل ذکر شده توانایی مورد استفاده قرار گرفته شدن را ندارد و از پیگیری این روش در این پژوهش اجتناب می‌کنیم.\بند
در ادامه‌ی این فصل به مروری کوتاه از تایخچه‌ی پرواز و پهپادها می‌پردازیم و سپس به بررسی کارهای قبلی انجام شده در رابطه با اجتناب از موانع ربات‌های چندپره به صورت خاص می‌پردازیم. دلیل آنکه به صورت خاص بروی روش‌های پیاده‌سازی شده بروی ربات‌های چندپره تمرکز می‌کنیم این است که پهپادها در حالت عموم دارای دینامیک‌ و مشخصات منحصر به فرد و نهایتا دارای کنترل‌های متفاوتی هستند که این امر منجر خواهد شد که هر حسگری را نتوان در هر پهپادی مورد استفاده قرار داد؛ که این دلایل باعث می‌شود روش‌های متفاوتی بجهت اجتناب از مانع برای انواع پهپادها مطرح شود. برخی از روش‌ها مانند روش‌های \مق{VFH}\مرجع{borenstein1991vector} و \مق{VFH+}\مرجع{ulrich1998vfhplus} دارای عمومیت هستند که می‌توان آن‌ها را در ربات‌های زمینی و اکثر ربات‌های هوایی مورد استفاده قرار داد. لذا در مرور این بخش علاوه بر کارهای انجام شده در زمینه‌ی اجتناب از مانع ربات‌های چندپره به بررسی مختصر این روش‌های عمومی نیز خواهیم پرداخت.