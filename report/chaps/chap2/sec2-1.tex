\قسمت{مقدمه}
«پرنده هدایت‌پذیر از دور یا به اختصار پَهپاد که به آن وسیله هوایی بدون سرنشین نیز گفته می‌شود، نوعی وسیله هوایی هدایت‌پذیر از راه دور است.» تعریفی که از پهپاد در ویکی‌پدیا شده است\مرجع{wiki:drone}. پهپاد به دو دسته کنترل شونده از راه دور توسط عامل انسانی و به صورت کاملا خودکار و برنامه‌ریزی شده می‌شوند. تاریخچه به وجود آمدن پهپادهای مدرن ریشه نظامی داشته و در صنعت نظامی در ماموریت‌هایی که برای انسان خطیر یا خسته کننده بودند استفاده می‌شد. به جهت پیشرفت روزافزون تکنولوژی‌های ساخت پهپاد، اکنون شاهد کاربردهای غیرنظامی آن‌ها هستیم. راهبری پهپادها همانند سایر ربات‌ها دارای خطراتی هستند که مهمترین آن‌ها خطر برخورد با موانع موجود در مسیر هست که در مورد پهپادها غالبا منجر به از دست رفتن کنترل، سقوط و از بین رفتن ربات می‌شود. از اینجا هست که نیاز به ارائه روش‌های اجتناب از مانع برخط\زیرنویس{Online} در پهپادها ضروری به‌نظر می‌رسد. از میان روش‌های اجتناب از مانع روش حسگر-مبنا\زیرنویس{\ {Sensor-based}} در زمینه‌ی ربات‌های هوایی استفاده می‌شود زیرا علاوه بر دینامیک پویا و غیرخطی پهپادها که هم‌بستگی شدیدی با متفییرهای محیطی(همانند سرعت جریان، تراکم هوا و غیره.) دارد تغییرات محیط خارجی نیز از پویایی بالایی برخوردار است. روش‌های دیگری همانند طرح‌ریزی سراسری\مرجع{thesis:izadi} نیز به جهت اجتناب از مانع وجود دارد ولی به دلیل آنکه این روش در صنعت هوایی به دلیل ذکر شده توانایی مورد استفاده قرار گرفته شدن را ندارد از پیگیری این روش در این پایان‌نامه اجتناب می‌کنیم.\بند
در ادامه‌ی این فصل به مروری کوتاه از تایخچه‌ی پرواز و پهپادها می‌پردازیم و سپس به بررسی کارهای قبلی انجام شده در رابطه با اجتناب از موانع ربات‌های چندپره به صورت خاص می‌پردازیم. دلیل آنکه به صورت خاص بروی روش‌های پیاده‌سازی شده بروی ربات‌های چندپره تمرکز می‌کنیم این است که پهپادها در حالت عموم دارای دینامیک‌های و مشخصات منحصر به فرد و پی این مساله دارای کنترل‌های متفاوتی هستند که نهایتا این امر منجر خواهد شد که هر حسگری را نتوان در هر پهپادی مورد استفاده قرار داد؛ که دلایل باعث می‌شود روش‌های متفاوتی بجهت اجتناب از مانع برای انواع پهپادها مطرح شود. برخی از روش‌ها مانند روش‌های \ {VFH}\مرجع{borenstein1991vector} و \ {VFH+}\مرجع{ulrich1998vfhplus} دارای عمومیت هستند که می‌توان آن‌ها را در ربات‌های زمینی و اکثر ربات‌های هوایی مورد استفاده قرار داد. لذا در مرور این بخش علاوه بر کارهای انجام شده در زمینه‌ی اجتناب از مانع ربات‌های چندپره به بررسی مختصر این روش‌های عمومی نیز خواهیم پرداخت.

\قسمت{تاریخچه پرواز و پهپاد}
از دیرباز رویای پرواز در ذهن انسان‌ها جا باز کرده بود، آسمان محلی مقدسی بود که استوره‌های باستان با هیبتی خداوندی از آن به زمین می‌آمدند... که این طرز نگرش نیازمند این بود که پرواز کردن و صعود به گنبد کبود به کهن‌ترین آرزوی آدمی بدل شود. این آرزو در اولین فرصت خود یعنی در حدود ۴۰۰ سال ق.م. با اختراع کایت\زیرنویس{Kite} که می‌توانست پرواز کند توسط مردمان چین به آتشی شعله‌کش در میان نسل بشر بدل گردید. جایگاه پرواز بقدری باارزش بود که در آن موقع کایت را به عنوان یک وسیله مقدس برای مراسم‌های مذهبی استفاده می‌کردند. بعد از گذشت سالیان دراز لئوناردو داوینچی در \سال{1480} فرصتی دوباره به این رویای کهن داد تا بلکه بتواند این رویا را به واقعیت بدل کند؛ وی اولین مطالعه رسمی تاریخ را بروی ماهیت پرواز انجام داد که این مطالعه شامل بیش از ۱۰۰ نقشه و تئوری پرواز بود. در \سال{1783} اولین بالن هوای گرم توسط برادران منتگولفیر\زیرنویس{\ {Joseph and Jacques Montgolfier}} ارائه شد. همچنین اولین گلایدر به همت آقای کی‌لی\زیرنویس{\ {George Cayley}} در یک دوره ۵۰ ساله در بین سال‌های \تاریخ{1799} و \تاریخ{1850} اختراع شد و بهبود پیدا کرد. در \سال{1891} یک مهندس آلمانی\زیرنویس{\ {Otto Lilienthal}} روی ایرودینامیک و طراحی گلایدرها مطالعه کرد و اولین فردی بود که توانست گلایدری را طراحی کند که می‌توانست یک انسان را در مسافت‌های طولانی حمل کند. در همان سال آقای لنگلی\زیرنویس{\ {Samuel P. Langley}} متوجه شد که به نیرو جهت پرواز انسان نیاز هست و مدلی را ارائه داد که دارای موتور بخار بود توانست ۳/۴ مایل را قبل اینکه سوختش تمام شود حرکت کند\مرجع{nasa:hist}.\بند

جنگ‌ها در کنار ویرانگری‌هایی که از خود پشت سر می‌گذارند همیشه باعث تکامل و جهش عمل بشری بوده‌اند؛ در جنگ‌های جهانی(بخصوص جنگ جهانی دوم) نوآوری‌های زیادی در زمینه‌ی علوم هواوفضا و رباتیک شد. اولین بار در اواخر جنگ جهانی اول بود که یک هواپیمای بدون سرنشین اختراع شد که توسط یک سامانه‌ی رادیویی کنترل می‌شد. در میانه‌ی جنگ‌های جهانی(سال‌های \تاریخ{1927} تا \تاریخ{1929}) اولین موشک کوروز(شکل \ref{fig:early_cruise_missile}) که بصورت یک هواپیمای تک-باله ساخته شد که از روی یک کشتی جنگی پرتاب و توسط خلبان خودکار هدایت می‌شد. موفقیت‌آمیز بود ساخت این موشک باعث شد که چند سال بعد هواپیماهای بدون سرنشین و کنترل کننده‌ی رادیویی در \سال{1930} ساخته شوند. در طی جنگ جهانی دوم نیروی دریایی ایالات متحده آمریکا شروع به انجام آزمایشاتی در زمینه‌ی هواپیما‌های رادیوکنترلی در \دهه{1930} کرد که نهایتا منجر به ساخت هواپیمای بدون سرنشین \ {Curtiss N2C-2} شد که به صورت کنترل از راه‌دور از یک هواپیمای دیگر کنترل می‌شد که به عنوان یک سامانه‌ی ضد هوایی به خدمت گرفته شد. در همین دوران ایالات متحده آمریکا تلاش کرد دستاوردهای خود را در زمینه‌ی هواپیماهای بدون سرنشین کنترل شونده از راه دور را بروی بمب افکن‌های \ {B-17 Flying Fortress} و \ {B-24 Liberator} خود به اجرا در بیاورد که نهایتا منجر به شکست و از دست رفتن شمار زیادی از بمب افکن‌ها شد. هواپیمای \ {TDN-1} یک هواپیمایی بدون سرنشین بود که در \سال{1940} ساخته شد که می‌توانست یک بمب ۱۰۰۰ پوندی(حدودا ۴۵۰ کیلوگرم) را به پرواز درآورده و به هدف بزند\مرجع{wiki:hist_uav}.\بند


\lfig{larynx}{موشک کروز اولیه به نام \ {RAE Larynx}}{fig:early_cruise_missile}
\fig{Curtiss_N2C-2_Naval_Aviation_Museum}{هواپیمای \ {Curtiss N2C-2} کنترل شونده از راه‌دور که در توسط ایالات متحده آمریکا در \سال{1938} ساخته شد.}

در تاریخچه‌ی هواپیماهای بدون سرنشین تا قبل از جنگ سرد به دلیل نبود تکنولوژی‌های مدرن امروزی جنس هواپیماها از جنس موتور، پیستون و گازوییل بودند و ارتباط کنترلی آن‌های بصورت رادیویی بود و معمولا دارای خلبان خودکار نبوده و در صورت وجود چنین سامانه‌ای، سیستمی بسیار ساده داشته و ادومتری آن‌های صرفا بر مبنای قطب‌نما، میزان سرعت و مدت زمان حرکت بود. در دوران جنگ سرد و بعد از آن بود که جهش‌های بزرگ در تکنولوژی‌های ساخت هواپیماهای بدون سرنشین ایجاد شد.\بند

در دوران جنگ سرد درپی موفقیت‌آمیز پهپاد پستونی\مق{OQ-2} هواپیماهای رادیویی\زیرنویس{Radioplane} به دوره‌ی جدیدی از نوآوری‌ها وارد شدند و موج جدیدی از استفاده و بکارگیری پهپادها در ارتش ایالات متحده‌ی آمریکا به راه افتاد. شرکت Globe بعد از ساخت پهپاد پیستونی \ {KDG Snipe} در \سال{1946} به ساخت پهپادهای \ {KD2G} و \ {KD5G} پرداخت که از نمونه‌های اولیه پهپادهای موتور-جت می‌باشند، کرد. در نهایت در اواخر \دهه{1950} پهپادهای جنگی پرقدرت پا به عرصه‌ی کاربردهای نظامی در سطح گسترده گذاشتند.

\fig[.75]{OQ-2A-Radioplane}{پهپاد پستونی \مق{OQ-2} یکی از موفق‌ترین پهپادهای اولیه که در دوران جنگ جهانی دوم ساخته شد و با تولید بیش از ۹,۴۰۰ عدد به تولید انبوه رسید\مرجع{wiki:oq-2}.}

در همین دوره که مسابقه‌ی اتمی بین ایالات متحده‌ی آمریکا و شوروی سابق شدت یافته بود، ایالات متحده‌ی آمریکا ۸ فقره از بمب افکن‌های \ {B-17 Flying Fortresses} خود را به پهپادها تبدیل کرد. این که تلاش قبلا در دوران جنگ جهانی دوم با شکست مواجه شده بود این دفعه موفقیت‌آمیز از آب درآمد و این هواپیماها به‌جهت جمع‌آوری اطلاعات در ابر-رادیواکتیو\زیرنویس{\ {Radioactive Cloud}} به خدمت گرفته شد. این هواپیماها در هنگام برخواست و فرود توسط یک کنترل کننده بروی یک جیپ کنترل می‌شد و در هنگام پرواز وسیله‌‌ی یک هواپیمای \ {B-17} دیگر از راه دور کنترل می‌شد. گرچه پیکربندی این پهپاد دارای موفقیت‌هایی در اجرا بود ولی به دلیل سیستم پیچیده‌ی پیاده‌سازی شده روی آن میزان اتفاقات آن نیز بالا بود.\بند

پهپادها همیشه به عنوان وسیله‌ی غیرقابل اعتماد و پرهزینه‌ی دیده می‌شد تا اینکه نیروی هوایی اسرائیل جهش بزرگی در پیشرفت روزبه‌روز پهپاد‌ها در پیروزی بر نیروی هوایی سوریه در \سال{1982} ایجاد کرد. اسرائیل با پیاده‌سازی سیستمی که با همکاری پهپاد و جنگده‌های دارای خلبان توانستند به سرعت تعداد زیادی از هواپیماهای جنگده سوری را از بین ببرند. در این جنگ پهپادها به عنوان طعمه‌\زیرنویس{Decoy}، متخل کننده‌\زیرنویس{Jammer} الکترونیکی و شناساگر ویدئویی\زیرنویس{Video Reconnaissance} مورد استفاده واقع می‌شدند\مرجع{wiki:hist_uav}.\بند
