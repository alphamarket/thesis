\قسمت{تاریخچه پرواز و ربات‌های پرنده}
از دیرباز رویای پرواز در ذهن انسان‌ها جا باز کرده بود آسمان محلی مقدسی بود که استوره‌های باستان از آن به زمین می‌آمدند، لذا آرزوی توانایی پرواز یکی از کهن‌ترین آرزوهای آدمی می‌باشد؛ در حدود ۴۰۰ سال ق.م. مردمان چین با اختراع کایت\زیرنویس{Kite} که می‌توانست پرواز کند به آتش این رویا دامن زده شد و در آن موقع کایت به عنوان یک وسیله مقدس برای مراسم‌های مذهبی نگاه می‌شد. بعد از گذشت سالیان دراز لئوناردو داوینچی در \سال{1480} اولین مطالعه را بروی ماهیت پرواز انجام داد که این مطالعه شامل بیش از ۱۰۰ نقشه و تئوری پرواز بود. در \سال{1783} اولین بالن هوای گرم توسط برادران منتگولفیر\زیرنویس{\مق{Joseph and Jacques Montgolfier}} ارائه شد. همچنین اولین گلایدر به همت آقای کی‌لی\زیرنویس{\مق{George Cayley}} در یک دوره ۵۰ ساله در بین سال‌های \تاریخ{1799} و \تاریخ{1850} اختراع شد و بهبود پیدا کرد. در \سال{1891} یک مهندس آلمانی\زیرنویس{\مق{Otto Lilienthal}} روی ایرودینامیک و طراحی گلایدرها مطالعه کرد و اولین فردی بود که توانست گلایدری را طراحی کند که می‌توانست یک انسان را در مسافت‌های طولانی حمل کند. در همان سال آقای لنگلی\زیرنویس{\مق{Samuel P. Langley}} متوجه شد که به نیرو جهت پرواز انسان نیاز هست و مدلی را ارائه داد که دارای موتور بخار بود توانست ۳/۴ مایل را قبل اینکه سوختش تمام شود حرکت کند\مرجع{nasa:hist}.
\cite{nesta:drobes, wiki:quadcopter}