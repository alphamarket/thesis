\قسمت{مقدمه}
در سالهای گذشته پژوهش های فراوانی در سیستم های چند عامله انجام شده است. محققان سعی داشتند در این پژوهش ها مزایای کار گروهی در انسان را در رایانه نیز ایجاد نمایند. یکی از قابلیت های عامل های هوشمند که میتواند با کار گروه سریعتر و بهتر شود موضوع یادگیری است.در این زمینه هم کارهایی انجام شده که معمولا الگو برداری از عامل های انسانی بوده است.
عامل های انسانی با تقلید از عامل هایی که دارای اطلاعات بیشتری هستند توانسته اند یادگیری خود را بهبود دهند. عامل های انسانی در شرایط بحرانی زندگی از عامل های با تجربه تر پند میگیرند،عامل های انسانی در مراتبی از خبرگی قرار دارند؛ همه این موارد الگوهایی مناسب بوده که توانسته یادگیری در سیستم های چند عاملی را بهبود بخشد.