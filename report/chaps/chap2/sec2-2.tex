\قسمت{اشتراک‌گزاری اطلاعات}
برای اولین بار در 10 اشتراک گزاری داده ها در سیستم های چند عامله مورد ارزیابی قرار گرفت. هدف این بررسی نمایش اثر اشتراک گزاری داده ها در مقابل سیستم های تک عاملی بود که آزمایش با تعداد عامل های یکسان انجام شد. نتیجه این پژوهش نشان داد که اگر اشتراک گزاری به خوبی انجام شود میتواند سرعت و کیفیت یادگیری را به صورت چشم گیری افزایش دهد. در این پژوهش سه نوع اشتراک گزاری مورد بررسی قرار گرفت در نوع اول که اشتراک گزاری ادراک نام گرفت عامل ها تنها نتایج مشاهدات خود را به اشتراک میگذاشتند، در نوع دوم اشتراک گزاری سه تایی حالت، عمل، کیفیت اشتراک گزاری شده و اشتراک گزاری واقعیت نامیده شد و نهایتا در نوع سوم اشتراک گزاری که اشتراک گزاری سیاست خوانده میشود اطلاعات داخلی عامل ها که منبع استخراج سیاست انهاست به اشتراک گذاشته شده است. اشتراک گزاری در این پژوهش با یک میانگین گیری ساده بین اطلاعات عامل ها انجام میشد. در این پژوهش که SA نامیده شد است ثابت شده ممکن است اشتراک گزاری سربارهایی در ترکیب داده ها به سیستم بیفزاید یا در شروع یادگیری از سرعت یادگیری بکاهد اما در طول یادگیری این سربارها جبران شده و اشتراک داده ها میتواند به صورت چشم گیری در افزایش سرعت سیستم های چند عامله موثر باشد.

\قسمت{یادگیری مشترک}
در [11] روشی تحت عنوان یادگیری مشترک مطرح میشود. در این روش اشتراک گزاری با در نظر گرفتن تنها یک سیاست برای تمام عامل ها انجام شد .نتایج این پژوهش نشان میدهد که در دسته بزرگی از مسائل روش های یادگیری مشترک میتواند مفید تر از روشهای یادگیری مستقل باشد. در این پژوهش یادگیری با منطق فازی ادغام شده است و نویسندگان سعی دارند اثر فازی کردن داده ها در یادگیری مشارکتی را نمایش دهند.

\قسمت{تقلید}
انسان در طول زندگی برای رسیدن به یادگیری روشهای متفاوتی دارد. گاهی برای رسیدن به یادگیری باید آزمایش کرد گاهی تحلیل کرد و گاهی تجربه اما یک روش که انسان از ان مخصوصا در مراحل رشد بسیار بهره میبرد تقلید است. همین موضوع باعث شده که در یادگیری مشارکتی نیز به تقلید عامل ها از هم توجه شود. بر همین اساس در 14 با ایده برداری از تقلید در انسان پیشنهاد شده که رابطه عامل ها از طریق تقلید از یکدیگر باشد.

موضوع دیگری که در مورد تقلید عامل های انسانی باید در نظر گرفته میشد این است که عامل های انسانی از عامل های انسانی تقلید میکنند که اطلاعات بیشتری دارند .در پیاده سازی انجام شده نیز بر همین اساس سه نوع تقلید پیشنهاد میشود. تقلید میتواند به صورت ساده باشد. پیشنهاد داده شده است که عامل ها همیشه از عامل های همسایه(همسایگی در این روش بر اساس همسایگی محلی است چرا که عامل هایی که در منطقه یکسانی قرار دارند کمک بیشری میتوانند به هم کنند) خود تقلید نمایند. این موضوع یک دور در عامل ها ایجاد میکند که هر عامل منتظر میماند تا عامل دیگر حرکتی انجام دهد. برای رفع این موضوع نوع دیگری از تقلید به نام تقلید شرطی مطرح میشود در تقلید شرطی عامل از کسانی تقلید میکند که عملکرد بهتری نسبت به او داشته اند در این حالت موضوع دور و انتظار عامل ها بر طرف شده است. اما در روش سوم که تقلید انطباقی نام دارد عامل همیشه تقلید نکرده و تقلید بر  اساس یک احتمال انجام میشود.

\قسمت{حافظه جمعی}
در 17 ایده جدید با عنوان یادگیری حافظه جمعی مطرح میشود. در یادگیری حافظه جمعی که برگرفته از شناخت توزیع شده در علوم اجتماع می باشد عامل ها تجارب خود را در یک حافظه مشترک نگهداری میکنند. هر عامل در زمان بر خورد با مشکلات میتواند با بهره گیری از این تجارب راه درست را پیدا کند.
این روش در دو دیدگاه مورد ارزیابی قرار گرفته است. در دیدگاه اول عامل ها الگوهای موفق خود در طول یادگیری را در حافظه مشترک نگهداری می کنند تا در زمان نیاز تمام عامل ها با استفاده از این الگو ها بتوانند راه حل مشکلات خود را پیدا کنند. در دیدگاه دیگر احتمال موفقیت عامل ها نگه داری میشود که با بهره گیری از این داده میتوان میزان موفقیت عامل ها در اعمال مختلف را ارزیابی کرده و در جهت بهبود طراحی سیستم مورد ارزیابی قرار داد.در 19 روشهای پیاده سازی این روش آورده شده است. حافظه جمعی را میتوان در دو حالت حافظه مرکزی و حافظه توزیع شده بین عامل ها مورد ارزیابی قرار داد.

\قسمت{پند}
در 20 روشی با عنوان پند دهی مطرح میشود. در جوامع انسانی پند دادن بسیار رواج داشته و در زمان مشکلات بسیار کارا میباشد. یک عامل انسانی در زمان برخورد با مشکلات از عامل هایی که اطلاعات بیشتری دارند پند گرفته و مشکلات خود را حل میکند. عاملی انسانی که دارای اطلاعاتی است هم اطلاعات خود را با تجربه کردن و یا گرفتن پند در زمان های دیگر به دست میاورد. مشخصا یادگیر تقویتی در حالت معمول با تجارب به یادگیری میرسد. اگر هر تجربه را بازخوردی از محیط در نظر بگیریم هر پند را نیز میتوان بازخوردی از عامل های دیگر دانست. با این ایده دیگر حتی نیازی نیست که عامل ها از روشهای یکسانی در یادگیری بهره ببرند زیرا پند دادن به عامل ها را میتوان فارق از روش یادگیری پیاده سازی کرد.
ایده پردازان پند در 22 کار قبل خود را کامل تر کرده و این ایده را به صورتی که عامل ها در یک محیط به تعامل میپرداختند پیاده سازی کردند. هر عامل بعد از رسیدن به هر حالت موقعیت خود را به عامل های دیگر ارسال مینماید. عامل های که تجربه مشابهی داشته اند در پاسخ به مقداری را به عامل ارسال میکنند عامل از این مقادیر همانند پاداش دریافتی از محیط بهره میبرد.

\قسمت{یادگیری مشارکتی بر مبنای خبرگی}
تشریح یادگیری مشارکتی بر مبنای خبرگی را با یک سوال میتوان آغاز کرد. آیا عامل ها در شناخت محیط از خبرگی یکسانی برخوردار هستند؟ مسلما چنین نیست، در 24 ایده یادگیری مشارکتی بر مبنای خبرگی با عنوان WSS مطرح میشود.
همان طور که در تشریح روش SA مطرح شد در این روش با میانگین گیری از اطلاعات عامل ها ترکیب انجام میشود. در این میانگین گیری تمام عامل ها به یک اندازه سهیم هستند. ایده پردازان WSS با طرح این موضوع که میزان خبرگی عامل ها یکسان نیست سعی کردند هر عامل در ترکیب داده ها به میزان توانایی و خبرگی خودش موثر باشد.

نویسندگان با ارائه معیار هایی میزان خبرگی عامل ها را سنجیده و بر همین اساس داده ها با هم ترکیب میشوند. در WSS روال یادگیری به دو فاز یادگیری مستقل و یادگیری مشارکتی شکسته شده است.
در یادگیری مستقل هر عامل به طور مستقل به یادگیری میپردازد این یادگیری منجر به کسب اطلاعاتی میشود که در فاز یادگیری مشارکتی با هم ترکیب میشوند. یادگیر در فاز یادگیری مستقل چندین چرخه یادگیری را تجربه میکند. تعداد این چرخه ها میتواند در بین عامل ها یکسان و یا متفاوت باشد. اما باید در انتخاب تعداد چرخه های یادگیری هر فاز یادگیری مستقل دقت کرد چرا که اگر این تعداد کم در نظر گرفته شود عامل اطلاعات کافی را جمع اوری نکرده است و اگر زیاد در نظر گرفته شود از تاثیر یادگیری مشارکی خواهد کاست.

در فاز دوم یادگیری عامل ها باید به یادگیری مشارکتی بپردازند. در آغاز این فاز میزان خبرگی عامل ها سنجیده میشود و پس از ان داده ها ترکیب شده و جداول Q  عامل ها بروز رسانی میشود. در 24 روشهایی جهت ترکیب دادها ارائه شده در روشی پیشنهاد شده که جدول تمام عامل ها با بهره گیری از میزان خبرگی میانگین گیری شده و جدول تولید شده به تمام عامل ها داده شود که در صورت انجام این کار بعد از فاز یادگیری مشارکتی تمام عامل ها جدول Q یکسانی خواهند داشت. در روش دیگری پیشنهاد شده که هر عامل جدول جدید خود را با ترکیب جدول خود با جدول عامل های خبره تر از خودش تولید کند. در این ترکیب نیز هر عامل به میزان خبرگی خودش در ترکیب داده ها سهم خواهد داشت.

در WSS یا در نظر گرفتن خبرگی عامل ها تاثیر زیادی در بهبود یادگیری مشارکتی داشته است اما نکته ای که در نظر گرفته نشده است اینجاست که میزان خبرگی عامل ها در دامنه های مختلف بسیار متفاوت بوده و بهتر است که در ترکیب داده ها این دامنه ها هم در نظر گرفته شود. در 26 با در نظر گرفتن دامنه خبرگی عامل ها سعی شده تا  نقصان WSS برطرف شود. بعد از ان در 29 سعی شده تا استفاده از جدول Q یک عامل در ترکیب داده ها قطعی نباشد. در این راستا در فاز ترکیب برای اطلاعات هر عامل احتمالی در نظر گرفته شده است  که نشان دهنده احتمال حضور اطلاعات ان عامل در ترکیب داده ها است. میزان این احتمال نیز بر اساس تفاوت میزان خبرگی عامل ها محاسبه شده است. در ادامه تعدادی از معیار های خبرگی معرفی شده در 24 خواهد امد.

معیار خبرگی معمولی: در این معیار میزان خبرگی عامل ها بر اساس مجموع پاداش های دریافتی آنها در نظر گرفته شده است. در نتیجه عاملی که میزان پاداش منفی کمتر و میزان پاداش مثبت بیشتری گرفته است را عامل خبره تر میداند.
معیار خبرگی مثبت: در این معیار سعی شده با شمارش پاداش ها مثبت عامل ها میزان خبرگی اندازه گیری شود. ایده انتخاب این معیار این بوده که عاملی که پاداش مثبت بیشتری گرفته است از خبرگی بالاتری برخوردار است.
معیار خبرگی منفی: این معیار برعکس معیار خبرگی مثبت با این ایده که عاملی که پاداش منفی بیشتری دارد نقاط بحرانی بیشتری را میشناسد عمل شده و تعداد پاداش های منفی عامل یادگیری را شمارش مینماید.
معیار خبرگی قدر مطلق: در معیار خبرگی قدر مطلق میزان خبرگی عامل با محاسبه مجموع قدر مطلق پاداش ها دریافتی او انجام میشود. در نتیجه به پاداش های منفی و مثبت ارزش یکسانی داده شده است.
معیار خبرگی گرادیان: در این معیار  ماننده معیار اول عمل میشود با این تفاوت که جمع پاداشها از ابتدای اخرین چرخه یادگیری مستقل انجام میشد. در نتیجه میشود گفت میزان خبرگی به دست امده عامل در اخرین فاز یادگیری مستقل شمارش میشود.
معیار خبرگی تعداد قدم ها: این معیار بر عکس پنج معیار دیگری به جای تاکید بر روی پاداش ها مجموع تعداد قدم های عامل در چرخه های یادگیری را معیار میداند. این انتخاب با این ایده انجام شده که عامل های خبره تر با تعداد قدم های کمتر چرخه های یادگیری را به اتمام میرسانند.

\قسمت{یادگیری مشارکتی بر مبنای تخته سیاه}
در 32 مکانیزم تخته سیاه مطرح شد. تخته سیاه یک حافظه مرکزی است که تمام عامل ها به ان دسترسی دارند. در این روش عامل ها به طور مستقیم با هم ارتباط نداشته و ارتباطات از طریق همین تخته سیاه انجام میشود. هر عامل میتواند بر روی تخته نوشته و یا از ان بخواند.
در روش پیشنهاد شده در 32 به این شکل است که عامل بعد از رسیدن به هر موقعیت حالت خود را به تخته سیاه اعلام میکند و تخته سیاه عملی را بر اساس حالت جاری به عامل برمیگرداند. عامل بعد از انجام ان عمل و دریافت بازخورد از محیط این بازخورد را به تخته سیاه بر میگرداند.

تخته سیاه دو دسته از داده ها را نگه داری میکند. دسته اول داده ها همان جدول Q عامل ها است و دسته دوم از داده ها عمل های انجام شده توسط هر عامل است. همان طور که مشخص است در این روش بروز رسانی جدول Q و انتخاب عمل از عامل به تخته سیاه منتقل شده و مشخصا جدول Q باید در تخته سیاه انجام شود. اما دسته دوم اطلاعات صرفا جهت کمک به انتخاب عمل عامل ها انجام میشود. به عنوان مثال اگر عامل در حالتی قرار گیرد و عملی تجربه نشده باشد ان عمل پیشنهاد میشود. پس ذخیره سازی دسته دوم اطلاعات در جهت مدیریت اکتشاف و بهره برداری عامل ها از اطلاعات است. در شکل زیر مکانیسم تخته سیاه نمایش داده شده است.

\قسمت{یادگیری مشارکتی بر مبنای پختگی سیاست}
در 33 روشی با عنوان یادگیری مشارکتی بر مبنای خبرگی چند معیاره ارائه شده است. این روشی تا حدودی ترکیب روش تخته سیاه با WSS میباشد. در این روش عامل ها حافظه مرکزی خود یا تخته سیاه را دارند که وجود تخته سیاه عامل ها را از شکستن بازه یادگیری به دو فاز بی نیاز میسازد. در روشی چون WSS یادگیری به دو باز یادگیری مستقل و یادگیری مشارکتی شکسته میشد تا عامل ها دادهای خود را به اشتراک بگزارند اما زمانی که عامل ها دائما میتوانند دادهای خود را بر روی تخته سیاه نوشته و بخوانند ارتباط از طریق همین تخته سیاه انجام خواهد شد.

اما عامل ها در عکس روش تخته سیاه ارتباط در اینجا از ارتباط مستقیم هم در تصمیم گیری ها و انتخاب اعمال بهره میبرند.عاملی که در وضعیت انتخاب عامل قرار گرفته میتواند در انتخاب عمل از عامل های دیگر بیاموزد.در این روش جهت شناخت عامل هایی که اطلاعات خوبی دارند و میتوانند اموزگار باشند از معیار های خبرگی ارائه شده در WSS استفاده شده است.با این کار عامل از عامل هایی می اموزد که واقعا از خبرگی بالاتری برخوردار هستند. این کار باعث میشود که در شروع یادگیری که عامل ها داده کمی دارند نیز عامل اموزگاری پیدا نکرده و با کمک اطلاعات و دستورات تخته سیاه عمل کند و بعد طی مراحلی از یادگیری که عامل ها دادهای زیادی کسب کردند  با بهره برن از نظرات انها انتخاب بهتری داشته باشد.

\قسمت{یادگیری مشارکتی بر مبنای خبرگی چند معیاره}
پاکیزه و همکاران در فلان با نقد از روش WSS روش جدید ارائه کردند. ایشان با اشاره به این موضوع که خبرگی در یک رشته نبوده در کار خود از ترکیب 6 معیار خبرگی WSS در کنار هم بهره برده اند. ایشان تاکید دارند که عامل های انسانی در زمینه های مختلف خبرگی های متفاوتی دارند و این موضوع در عامل های هوشمند نیز وجود دارد. ایشان هر یک از معیار های ارائه شده در WSS را مانند یک زمینه در عامل انسانی دانسته و در روش خود از تمام این معیار ها در کنار هم بهره برده اند.

ایشان مانند مانند WSS یادگیری را در دو فاز یادگیری مستقل و یادگیری مشارکتی تقسیم مینمایند عامل ها در فاز یادگیری مشترک از هر معیار برای ترکیب دادا های جدول Q بهره میبرند و بعد از ترکیب جدول به وسیله هر معیار 6 جدول مشارکتی تولید میشود که هر یک بر اساس یک معیار خبرگی است. ایشان برای ترکیب این جدوال انها را با هم جمع میکنند. اما موضوعی که وجود دارد این است که جدول تولید شده به وسیله ی جمع چندین جدول دیگر خواص جدولQ را ندارد.

 پاکیزه در کار خود برای رفع این مشکل این جدول را نه در جایگزینی با جدول Qعامل ها بلکه در کنار جدول Q عامل نگه داری مینمایند.
پس در کار خانم پاکیزه هر عامل دو جدول دارد یک جدول Q که بر اساس یادگیری تقویتی است و جدول دیگر که جدول مشارکتی عامل ها است. خانم پاکیزه پیشنهاد کردند که از جدول مشارتی که خواص جدول Q عامل ها را ندارد صرفا برای انتخاب عمل استفاده شود و عامل بر اساس این جدول عمل را انتخاب کرده انجام دهد سپس جدول Q خود را بروز رسانی نماید.

\قسمت{تسریع یادگیری مشارکتی با بهره گیری از کوتاهترین فاصله تجربه شده}
میرزایی در فلان جهت تسریع در یادگیری مشارکتی دو معیار جدید را ارائه داده است. معیار اول یک معیار مکاشفه است که کوتاهترین فاصله تجربه شده توسط عامل از هر حالت و عمل را شمارش میکند. ایشان نام این معیار را SET گذاشته است. معیار دیگر که شوک نام گزاری شده است میزان شناخت عامل از هر حالت و عمل را محاسبه مینماید.

میرزایی بر خلاف دیگران فقط در فاز ترکیب داده های یادگیری مشارکتی ویرایش ایجاد نکرده است. ایشان در فاز انتخاب عمل توسط عامل های مشارکتی نیز از جدول SEP در کنار جدولQ استفاده کرده است. استدلال ایشان در انجام این کار چنین بوده که عامل های یادگیری تقویتی در فاز های اول یادگیری داده زیادی ندارند و از انجایی که جدولSEP با سرعت بیشتری بروزرسانی میشود بهتر است انتخاب اعمال در فازهای اولیه یادگیری بیشتر بر اساس SEP انجام شود. ایشان با استفاده از شوک که نمایشی از میزان شناخت عامل از هر حالت و عمل است تعادلی بین بهره برداری از جدول SEP و جدول Q برقرار کرده است. در شروع یادگیری که شناخت عامل کمتر است بیشتر انتخاب بر اساس SEP انجام میشود و در طول یادگیری با افزایش میزان شناخت عامل از محیط انتخاب عمل بر اساس جدول Q افزایش می یابد.

ایشان همچنین در فاز ترکیب داده ها نیز ویرایش ایجاد کرده است. از انجایی که ایشان یک جدول جدید به سیستم افزوده است در فاز ترکیب داده ها جدول SEP عامل ها را ترکیب مینماید. ایشان جداول SEP عامل ها را تنها با یک حداقل گیری با هم ترکیب کرده و به عامل ها بر میگرداند. سپس ترکیب جداول Q عامل ها را به صورت محلی انجام میشد به این صورت که هر سطر از جدول که نمایش یک حالت از محیط است به صورت جداگانه بروز رسانی میشود. ایشان در ترکیب داده های هر سطر عامل ها را به دو گروه تقسیم نموده و داده های هر گروه را جداگانه ترکیب مینماید. این تقسیم بندی بر اساس رابطه بین سیاست های استخراج شده از جدول Q و SEP عامل در یک حالت میباشد. ایشان عامل هایی که سیاست استخراج شده از جدول Q و SEP در انها همخوانی داشته باشد در یک گروه و عامل هایی که سیاست استخراج شده انها عمل های متفاوتی را پیشنهاد میکنند را در گروه دیگر قرار داده است. ترکیب داده های هر گروه با استفاده از میزان شناخت عامل از ان حالت (شوک) انجام میشود به این صورت که داده های عملی که شناخت بیشتری دارند بیشتر مورد استفاده قرار میگیرند. در فصل بعد روش محاسبه جدوال SEP و شوک تشریح شده و مورد بررسی قرار میگیرد.