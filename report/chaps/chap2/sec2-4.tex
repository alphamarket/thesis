\قسمت{نتیجه‌گیری}
در این فصل ابتدا به مروری خلاصه از تاریخچه‌ی پرواز و روند به‌وجود آمدن پهپادها پرداختیم، سپس با معرفی الگوریتم‌های اولیه \جام که از الگوریتم‌های بنیادین این حوزه می‌باشند پرداختیم و بعد از آن به‌صورت ساختارمند پژوهش‌هایی که از \سال{2010} تا به کنون برای تعمیم این الگوریتم‌ها انجام شده است، ارائه شد. در قسمت‌های بعدی کاربردهای شبکه عصبی مصنوعی، بینایی ماشین و سیستم‌های فازی در \جام معرفی شد. سپس پژوهش‌های انجام شده در زمینه‌ی \جام بروی ربات‌های پهپاد‌ها به صورت خاص مورد مرور واقع شدند و نهایتا با گذری خلاصه بر اثرهای پژوهش‌های انجام شده در علم رباتیک برای حل مساله‌ی \جام بر زندگی روزمره انسان‌ها به این فصل خاتمه دادیم.\بند
همانطور که قبلا نیز ذکر شد در این فصل فقط پژوهش‌های بروز که از \سال{2010} تا به کنون انجام شده است آورده‌ایم، که این تصمیم برمبنای دو اصل «بروز بودن مطالب پیش‌زمینه‌ی این پژوهش» و «حفظ تناسب متنی این نگارش» صورت گرفته شده است؛ همچنین علاوه بر بروز بودن پژوهش‌های معرفی شده، همان‌گونه که نشان داده شد، پرکاربرد بودن این زمینه در مسائل گوناگون دنیای مدرن نشان از اهمیت پژوهشی این موضوع می‌دهد.\بند