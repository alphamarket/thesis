\قسمت{مقدمه}
معمولا در دنیایی واقعی هنگامی که افراد برای انتقال دانش گرد هم می‌آیند و از تجربیات خوب و بد گذشته خود سخن می‌گویند هرکسی متناسب با جایگاهی که دارد دارای دانشی می‌باشد و در این انتقال دانش‌ها تجربیات هیچ کسی را نمی‌توان نادیده گرفت ولی گاها پیش می‌آید که تجربیات و دانش فردی دارای بار محتویاتی بیشتری نسبت به اطرافیان خود می‌باشد، مردم معمولا از دانش فرد خبره‌تر بیشتر بهره می‌برند تا افراد دیگر. دستاورد‌های این پژوهش بر مبنای همین فلسفه بنا شده است که سخن و دانش هرکسی باید شنیده شود -- یعنی آزادی بیان!!

انتگرال فازی یکی از قوی‌ترین و منعطف‌ترین ابزارهای ریاضی برای مدل کردن آزادی بیان می‌باشد، لذا در این پژوهش از انتگرال فازی برای شنیدن بازتاب ندای دانش هر عامل در دانش جمعی استفاده شده است. ولی در این راه مشکلاتی نیز وجود داشت و آن این بود که چگونه منصفانه بفهمیم که کدام عامل خبره‌تر از دیگری می‌باشد؟ در گذشته روش‌های متنوعی برای تخمین این معیار ارائه شد است که از شمارش میزان پاداش‌های مثبت و منفی عامل‌ها گرفته تا محاسبات پیچیده‌ای چون معیار‌های شوک و کوتاه‌ترین مسیر تجربه شده. در طی پژوهش که منجر به نگارش این پایان‌نامه گردید احساس شد که تمامی روش‌های قبلی در یک چیز مشترکند: بسیار پیچیده و غیر منعطف!

وجود این فصل مشترک ناکارا انگیزه‌ای شد که در صدد ارائه‌ای معیاری برآیم که نه تنها ساده باشد بلکه در زندگی روزمره ما انسان‌ها هم تجلی داشته باشد. بعد از اندکی تفکر و تفحص در نهایت این معیار چیزی جز معیار «تنبلی» نبود! معیار تنبلی که در این پایان‌نامه با اصطلاح علمی «میزان ارجاع» ارائه شد می‌گوید که «عاملی هرچقدر تنبل‌تر خبره‌تر»! در نگاه اول ممکن است این معیار چندان معقولانه به نظر نرسد ولی اگر کمی به زندگی روزمره خودمان توجه کنیم متوجه می‌شویم که این معیار در تار و پود معیارهایی که ما برای سنجش میزان خبرگی خودمان، دوستان‌مان و همکاران‌مان استفاده می‌کنیم، وجود دارد.

اگر اندکی به مسائلی که افراد انجام می‌دهند و ما آن‌ها را در آن خبره می‌بینیم توجه کنیم متوجه خواهیم شد که زمانی که فردی در موردی خبره می‌شود بطور طبیعی انرژی نسبتا کمتری در انجام آن مصرف می‌کند. این معیار همان معیار تنبلی می‌باشد که می‌گوید عاملی در انجام وظیفه‌ای خبره‌تر است که در طی انجام آن انرژی کمتری مصرف کند. این معیار که از فلسفه‌ی بسیار ساده‌ای برخوردار است برخلاف معیار‌های گذشته بسیار منعطف می‌باشد زیرا که در تعریف این معیار عبارت «میزان انرژی» می‌تواند تعابیر مختلفی به خود بگیرد و در هر مورد قابل استفاده باشد.\\
در این قسمت به مروری خلاصه بر هرآنچه که در این پژوهش صورت گرفته و ارائه‌ی یک نتیجه‌گیری نهایی حاصل از این پژوهش و همچنین ارائه‌ی مسیر پژوهشی پیشنهادی برای آیندگان این زمینه از یادگیری مشارکتی خواهیم پرداخت.

\قسمت{نوآوری‌ها و نتایج کلی پایان‌نامه}
در طی این پایان‌نامه معیار جدیدی به نام معیار «میزان ارجاع» ارائه شد که می‌گوید عاملی که کمتر در محیط مورد تعاملش پرسه بزند از خبرگی بیشتری برخوردار است و سپس با استفاده از این معیار خبرگی به سنجش عامل‌های فعال در محیط در هنگام مشارکت در دانش جمعی پرداختیم. در هنگام ترکیب دانش عامل‌ها از انتگرال فازی چوکت استفاده شد که طبق آنچه که در فصل آزمایش‌ها نشان‌ داده شد در بهبود کیفیت و سرعت عامل‌ها موثر واقع گردیده است.

در طی آزمایشات از میانگین وزنی نیز به جای انتگرال فازی استفاده شد و نشان داده شد که انتگراف فازی توانایی بهتری نسبت به میانگین وزنی برای بهبود کیفیت و سرعت یادگیری مشارکتی دارد. همچنین از ۴ تابع به عنوان مدل کننده‌ی تابع $g(\cdot)$ استفاده شد، که هرکدام یک دیدگاهی نسبت به نحوه‌ی ترکیب دانش‌های ورودی ارائه می‌دهد. از بین این ۴ تابع، تابع \مق{Const-One} در کلیه‌ی آزمایشات نسبت به دیگر توابع برتریت قابل توجه‌ای از خود نشان داد؛ طبق آنچه که فصول قبلی این پایان‌نامه آورده شده این تابع معادل با حداکثرگیری بروی دانش عامل‌ها بر اساس معیار خبرگی آن‌ها می‌باشد. یعنی اینکه این تابع در واقع در هر ناحیه فقط دانش عاملی را در نظر می‌گیرد از همه خبره‌تر (تنبل‌تر) است که این امر تاییدی بر تئوری
\ref{experties_theorem}
و متعاقبا تعریف
\ref{experties_definition}
می‌باشد. در نهایت در انتهای فصل آزمایشات نشان داده شد که می‌توان معیار خبرگی ارائه شده در تعریف
\ref{experties_definition}
را به کل محیط خلاصه کرد؛ یعنی عاملی خبره‌تر است که میزان حضور آن در کل محیط کمتر باشد -- یعنی با تعداد گام کمتری به اهداف خود برسد. همین نتیجه‌گیری باعث می‌شود که آزمودن دیگر توابع برای مدل کردن $g(\cdot)$ (مثلا تابع اندازه‌گیری-$\lambda$ سوگنو) نیازی نباشد.

در این پژوهش تعادلی بین کلی و جزئی نگری به عملکرد عامل‌ها در هنگام ادغام دانش‌های آن‌ها برقرار شد. همچنین تاثیر دیگر روش‌های انتخاب عمل را در ترکیب با معیار‌های ارائه شده را مورد بررسی قرار گرفته است و به این نتیجه رسیدیم که تابع بولتزمن نتیجه‌ی با کیفیت‌تری را تولید می‌کند. همچنین دستاورد‌های این پژوهش را با در نظر گرفتن ماهیت غیرافزایشی بودن ذات مساله ارائه دادیم.

یکی از مزایای روش پیشنهادی این است که در عین کارایی و قدرت روشی ساده در مفهومی و پیاده‌سازی می‌باشد که این سادگی طبق آنچه که در آزمایش‌ها آمده است نهایتا منجر شد که روش پیشنهادی از پیچیدگی کمتری برخوردار باشد. از دیگر مزیت روش پیشنهادی کلی بودن تئوری خبرگی‌ای که این پژوهش برمبنای آن ارائه شد، می‌باشد که می‌توان آن را به تمامی مسائل یادگیری مشارکتی به راحتی اعمال کرد.

\قسمت{راهکارهای آینده و پیشنهادها}
همانطور که آزمایشات نشان دادند با توجه به معیار خبرگی ارائه شده در قسمت یادگیری مشارکتی اگر فقط دانش عامل خبره را در نظر بگیریم حداکثر نتیجه‌ی ممکن (در قالب روش پیشنهادی) را خواهیم گرفت. در طی این پژوهش دو مفهوم مهم ارائه شد: ۱. انتگرال فازی چوکت می‌تواند عملگر بسیار قوی‌ای نسبت به روش‌ها سنتی چون میانگین‌گیری وزنی باشد. ۲. تئوری خبرگی معرفی شده بخوبی می‌تواند هر نوع معیار خبرگی را توجیه کند.

در این پژوهش سعی شده است که حداکثر نتیجه‌ی ممکن حاصل از استفاده از این دو مفهوم باهم را استخراج کنیم ولی پیشنهادات زیر می‌تواند شروع خوبی برای پژوهش‌های آینده در این زمینه باشد.

\begin{enumerate}
\فقره ارائه‌ی معیار خبرگی جدیدی مبتنی بر تئوری خبرگی معرفی شده در این پژوهش.
\فقره بررسی تاثیر استفاده از انتگرال فازی چوکت در پژوهش‌های گذشته.
\end{enumerate}