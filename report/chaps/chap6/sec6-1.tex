\قسمت{مقدمه}
اگر اندکی به مسائلی که افراد انجام می‌دهند و ما آن‌ها را در آن خبره می‌بینیم توجه کنیم، متوجه خواهیم شد که زمانی که فردی در موردی خبره می‌شود بطور طبیعی انرژی نسبتا کمتری در انجام آن مصرف می‌کند. این معیار همان معیاری است که می‌گوید عاملی در انجام وظیفه‌ای خبره‌تر است که در طی انجام آن انرژی کمتری مصرف کند. این معیار که از فلسفه‌ی بسیار ساده‌ای برخوردار است برخلاف معیار‌های گذشته بسیار منعطف می‌باشد زیرا که در تعریف این معیار عبارت «میزان انرژی» می‌تواند تعابیر مختلفی به خود بگیرد و در هر مورد قابل استفاده باشد.

وجود این فلسفه انگیزه‌ای شد که در صدد ارائه‌ای معیاری برآییم که نه تنها ساده باشد بلکه در زندگی روزمره ما انسان‌ها هم تجلی داشته باشد. بعد از اندکی تفکر و تفحص در نهایت این معیار چیزی جز معیار «تنبلی» نبود! معیار تنبلی که در این پایان‌نامه با اصطلاح علمی «میزان ارجاع» ارائه شد می‌گوید که «عاملی که برای به نتیجه رساندن فعالیت‌هایش انرژی کمتری صرف کند خبره‌تر است».\\
در این قسمت به مروری خلاصه بر هرآنچه که در این پژوهش صورت گرفته و ارائه‌ی یک نتیجه‌گیری نهایی حاصل از این پژوهش و همچنین ارائه‌ی مسیر پژوهشی پیشنهادی برای آیندگان این زمینه از یادگیری مشارکتی خواهیم پرداخت.

\قسمت{نوآوری‌ها و نتایج کلی پایان‌نامه}
در طی این پایان‌نامه معیار جدیدی به نام معیار «میزان ارجاع» ارائه شد که می‌گوید عاملی که کمتر در محیط مورد تعاملش پرسه بزند از خبرگی بیشتری برخوردار است و سپس با استفاده از این معیار خبرگی به سنجش عامل‌های فعال در محیط در هنگام مشارکت در دانش جمعی پرداختیم. در هنگام ترکیب دانش عامل‌ها از انتگرال فازی چوکت استفاده شد.

در طی آزمایش‌ها از میانگین وزنی نیز به جای انتگرال فازی استفاده شد و نشان داده شد که در معیار ارائه شده توسط این پژوهش انتگرال فازی توانایی بهتری نسبت به میانگین وزنی برای بهبود کیفیت و سرعت یادگیری مشارکتی دارد. همچنین از ۴ تابع به عنوان مدل کننده‌ی تابع $g(\cdot)$ استفاده شد، که هرکدام یک دیدگاهی نسبت به نحوه‌ی ترکیب دانش‌های ورودی ارائه می‌دهد. از بین این ۴ تابع، تابع \مق{Const-One} در کلیه‌ی آزمایش‌ها نسبت به دیگر توابع برتری قابل توجهی از خود نشان داد؛ طبق آنچه که فصول قبلی این پایان‌نامه آورده شده این تابع معادل با حداکثرگیری بروی دانش عامل‌ها بر اساس معیار خبرگی آن‌ها می‌باشد. یعنی اینکه این تابع در واقع در هر ناحیه فقط دانش عاملی را در نظر می‌گیرد از همه خبره‌تر (تنبل‌تر) است که این امر تاییدی بر فرضیه
\ref{experties_theorem}
و متعاقبا تعریف
\ref{experties_definition}
می‌باشد. همچنین تاثیر استفاده از انتگرال فازی در روش SEP را نیز مورد بررسی قرار دادیم و مشاهده کردیم که انتگرال فازی \textbf{با توابع $g(\cdot)$ معرفی شده در این پژوهش} در روش SEP تاثیر مثبتی ندارد ولی در این بررسی نیز تابع \مق{Const-One} بیشترین بهبود را در پی‌داشت که معادل می‌شود با انتخاب حریصانه‌ی بین دانش عامل‌ها به عنوان دانش جمعی، که یک نقطه‌ی مشترک بین نتایج روش پیشنهادی و این بررسی می‌باشد؛ از طرفی بررسی‌های انجام شده در رابطه با تاثیر انتگرال فازی \textbf{با توابع $g(\cdot)$ معرفی شده در این پژوهش} در روش MCE نشان می‌دهد که انتگرال فازی موثر واقع شده است، که نشان می‌دهد در صورت انتخاب مناسب تابع $g(\cdot)$ انتگرال فازی می‌تواند جایگزین بهتری بجای میانگین‌گیری وزنی باشد. همچنین لازم به ذکر است که بررسی‌های انجام شده در استفاده از انتگرال فازی بروی هر سه روش \رفمت\ و SEP و MCE نشان می‌دهد که انتخاب حریصانه دانش عامل‌ها به عنوان دانش جمعی (انتخاب \مق{Const-One} به عنوان تابع  $g(\cdot)$) همیشه باعث حداکثر شدن سرعت و کیفیت یادگیری نمی‌شود که این نتیجه‌گیری به اهمیت استفاده از انتگرال فازی به عنوان تابع ادغام‌کننده دانش‌ عامل‌ها می‌افزاید.

همچنین در نهایت، در انتهای فصل آزمایش‌ها نشان داده شد که می‌توان معیار خبرگی ارائه شده در تعریف
\ref{experties_definition}
را به کل محیط خلاصه کرد؛ یعنی عاملی خبره‌تر است که میزان حضور آن در کل محیط کمتر باشد -- یعنی با تعداد گام کمتری به اهداف خود برسد. همین نتیجه‌گیری باعث می‌شود که آزمودن دیگر توابع برای مدل کردن $g(\cdot)$ (مثلا تابع اندازه‌گیری-$\lambda$ سوگنو) نیازی نباشد.

در این پژوهش تعادلی بین کلی و جزئی نگری به عملکرد عامل‌ها در هنگام ادغام دانش‌های آن‌ها برقرار شد. همچنین تاثیر دیگر روش‌های انتخاب عمل را در ترکیب با معیار‌های ارائه شده را مورد بررسی قرار گرفته است و به این نتیجه رسیدیم که تابع بولتزمن نتیجه‌ی با کیفیت‌تری را تولید می‌کند. همچنین دستاورد‌های این پژوهش را با در نظر گرفتن ماهیت غیرافزایشی بودن ذات مساله ارائه دادیم.

یکی از مزایای روش پیشنهادی این است که در عین کارایی و قدرت روشی ساده در مفهومی و پیاده‌سازی می‌باشد که این سادگی طبق آنچه که در آزمایش‌ها آمده است نهایتا منجر شد که روش پیشنهادی از پیچیدگی کمتری برخوردار باشد. از دیگر مزیت روش پیشنهادی کلی بودن فرضیه خبرگی‌ای که این پژوهش برمبنای آن ارائه شد، می‌باشد که می‌توان آن را به تمامی مسائل یادگیری مشارکتی به راحتی اعمال کرد.

\قسمت{راهکارهای آینده و پیشنهادها}
همانطور که آزمایش‌ها نشان دادند با توجه به معیار خبرگی ارائه شده در قسمت یادگیری مشارکتی اگر فقط دانش عامل خبره را در نظر بگیریم حداکثر نتیجه‌ی ممکن (در قالب روش پیشنهادی) را خواهیم گرفت. در طی این پژوهش دو مفهوم مهم ارائه شد: ۱. انتگرال فازی چوکت می‌تواند عملگر بسیار قوی‌ای نسبت به روش‌ها سنتی چون میانگین‌گیری وزنی باشد. ۲. فرضیه خبرگی معرفی شده بخوبی می‌تواند هر نوع معیار خبرگی را توجیه کند.

در این پژوهش سعی شده است که حداکثر نتیجه‌ی ممکن حاصل از استفاده از این دو مفهوم باهم را استخراج کنیم ولی پیشنهادها زیر می‌تواند شروع خوبی برای پژوهش‌های آینده در این زمینه باشد.

\begin{enumerate}
\فقره ارائه‌ی معیار خبرگی جدیدی مبتنی بر فرضیه خبرگی(فرضیه‌ی \ref{experties_theorem}) معرفی شده در این پژوهش که چهارچوبی کلی جهت تعریف معیارهای خبرگی را تعریف می‌کند؛ سپس آزمایش معیار خبرگی تعریف شده بجهت آزمودن فرضیه خبرگی ارائه شده.
\فقره بررسی تاثیر استفاده از انتگرال فازی چوکت در پژوهش‌های گذشته.
\فقره بررسی این موضوع که «آیا انتخاب دانش عاملی که از همه خبره‌تر است همیشه باعث حداکثر شدن کیفیت و سرعت یادگیری جمع می‌شود یا خیر؟».
\فقره ارائه‌ی روشی جهت وفقی شدن پارامترهای معرفی شده در بخش \ref{sec:sub:determine_f_g} با هدف بهبود کیفیت و سرعت یادگیری هرچه بیشتر روش پیشنهادی.
\فقره بررسی شرایط و ویژگی‌های توابع $g(\cdot)$ در حالت کلی.
\end{enumerate}