\قسمت{معیارهای ارزیابی}
جهت ارزیابی هر سیستمی به معیارهایی نیاز است، در یادگیری مشارکتی نیز دو معیار «سرعت و دقت یادگیری» که محاسبه آن‌ها بر اساس میانگین تعداد قدم‌ها در هر چرخه یادگیری محاسبه میشود کاربرد دارد\مرجع{thesis:pakizeh2013multi, mohammad2015speedup}. در صورتی که تعداد قدم‌های چرخه‌های یادگیری نگه‌داری و به صورتی نموداری که در محور عمودی میانگین تعداد قدم‌ها و در محور افقی چرخه یادگیری باشد رسم شود، می‌توان آخرین نقطه از محور را که نشانه میانگین تعداد چرخه‌ها در آخرین چرخه یادگیری است را به‌عنوان کیفیت و مساحت زیر این نمودار را به‌عنوان سرعت یادگیری در نظر گرفت. نکته مهم در اینجا این است که هرچه این مقادیر کمتر باشند روش پیشنهادی از سرعت و کیفیت بالاتری برخوردار خواهد بود. در شکل \ref{fig:learning_speed_qual} این معیارها نمایش داده‌شده است.

\fig{learning_speed_qual}{سرعت و کیفیت یادگیری از معیارهای ارزیابی‌ و مقایسه‌ی عملکرد الگوریتم‌های یادگیری تقویتی می‌باشد \مرجع{mohammad2015speedup}.}