\قسمت{اندازه‌گیری‌ و انتگرال‌ فازی}
\label{SEC:FI_PREVIEW}
برای درک روش پیشنهادی نیاز به داشتن اطلاعات پایه در مورد اندازه‌گیری‌های فازی\زیرنویس{\مق{Fuzzy measures}} و انتگرال فازی داریم که با هدف جمع‌آوری اطلاعات\زیرنویس{\مق{Aggregate Information}} ارائه شده‌اند. اندازه‌گیری‌های فازی پیش‌زمینه‌ای بر انتگرال‌های فازی هستند که قبل از آنکه آشنایی با انتگرال‌های فازی نیاز به معرفی اندازه‌گیری‌های فازی داریم. اگر فرض کنیم که تعداد منبع اطلاعاتی
$X = \{x_1, x_2, \cdots, x_n\}$
که این منابع اطلاعاتی اطلاعات دریافتی از سنسورها، پاسخ‌های داده شده به یک پرسشنامه و غیره باشند. اندازه‌گیری فازی میزان ارزش اطلاعاتی این منابع را در اختیار ما می‌گذارد. معمولا اندازه‌گیری فازی توسط تابع
$g : 2^{|X|} \rightarrow [0, 1]$
تعریف می‌شود که ورودی آن یک زیرمجموعه‌ای از منابع اطلاعاتی می‌باشد و خروجی آن یک مقدار مابین صفر و یک که میزان ارزش اطلاعاتی که آن زیرمجموعه از منابع اطلاعاتی ورودی تابع را مشخص می‌کند.\بند
این تابع باید دارای شرایط مرزی تعریف شده و یکنوختی باشد که در ادامه به معرفی شرایط می‌پردازیم\مرجع{torra2006interpretation}:\\
\textbf{۱. شرایط مرزی:}
اگر اطلاعاتی در دست نداریم ارزش صفر را دارد و کلیه اطلاعاتی حداکثر ارزش ۱ را دارد.
\begin{equation}
g(\emptyset) = 0,\hspace{10pt} g(X) = 1
\end{equation}
\textbf{۲. یکنواختی - غیر کاهشی:}
اگر اطلاعات بیشتری به دست آمد ارزش کلیه اطلاعات که شامل اطلاعات جدید می‌باشد حداقل به اندازه زمانی است که آن اطلاعات جدید بدست نیامده است.
\begin{equation}
A \subseteq B \subseteq X \Rightarrow g(A) \leq g(B) \leq 1
\end{equation}

مقادیر تابع $g$ یا توسط کارشناس ارائه می‌شود یا توسط یک تابعی مدل می‌شود، یکی از توابع معروف برای تخمین مقادیر تابع $g$ تابع اندازه‌گیری-$\lambda$ سوگنو\زیرنویس{\مق{Sugeno $\lambda$-Measure}} می‌باشد که به صورت زیر تعریف می‌شود\مرجع{leszczynski1985sugeno}.
\begin{equation}\label{eq:sugeno-lambda-measure}
g(\{x_1,\cdots,x_l\}) = {1 \over \lambda}\left[\prod_{i=1}^l(1 + \lambda g_i) - 1\right]
\end{equation}
که در معادله \ref{eq:sugeno-lambda-measure} مقدار $g_i$ها مقادیر ارزش هریک از منابع اطلاعاتی است و $\lambda$ بگونه‌ای تعیین می‌گردد که $g_\lambda(X) = 1$ شود که این مقدار برابر با جواب معادله‌ی زیر باشد.
\begin{equation}\label{eq:sugeno-lambda-measure:rooting}
\lambda + 1 = \prod_{i=1}^{n} (1 + \lambda g_i), \hspace{1em} \lambda \in (-1, \infty) 
\end{equation}

نکته‌ای که در رابطه با تابع اندازه‌گیری-$\lambda$ سوگنو باید توجه کرد این است که به ازای مقادیر $n$ مختلف باید ریشه‌یابی بروی متغییر $\lambda$ صورت گیرد؛ این ویژگی‌ باعث می‌شود که این تابع در بعضی از کاربردها کارایی نداشته باشد.

انتگرال فازی در واقع یک تعمیمی به روش میانگین وزنی\زیرنویس{\مق{Weighted Arithmetic Mean}} می‌باشد بطوری که نه تنها مشخصه‌های مهم تک تک ویژگی‌ها را در نظر می‌گیرد بلکه اطلاعات تعاملات بین ویژگی‌ها را نیز در نظر می‌گیرید\مرجع{tehrani2012preference}. از میان انتگرال‌های فازی دو انتگرال سوگنو\زیرنویس{Sugeno} و چوکت\زیرنویس{Choquet} از الگوریتم‌هایی هستند که می‌توانند بروی هر اندازه‌گیری‌ فازی مورد استفاده واقع شود\مرجع{de1992characterization}. فرض کنیم که تابعی چون
$h : X \rightarrow [0, 1]$
وجود دارد که مقادیر منابع اطلاعاتی را به بازه‌ی $[1, 0]$ نگاشت می‌کند. در واقع $h$ تابع پشتیبان\زیرنویس{Support} منابع اطلاعاتی می‌باشد. انتگرال فازی سوگنو به صورت زیر تعریف می‌شود\مرجع{grabisch1995fuzzy, de1992characterization}:
\begin{eqnarray}
\int_{s} h \circ g = \mathcal{S}_g(h) = \bigvee_{i=1}^{n} h(x_{\pi_i^s}) \wedge g(A_i^s)\label{eq:sugeno_integral}\\
h \xrightarrow{\pi^s} h(x_{\pi_1^s}) \leq h(x_{\pi_2^s}) \leq \cdots \leq h(x_{\pi_n^s})\label{eq:fi_sugeno_perm_op}\\
A_i^s = \{\pi_i^s, \pi_{i+1}^s, \cdots, \pi_n^s\}\label{eq:fi_sugeno_perm_val}
\end{eqnarray}
در انتگرال‌ سوگنو لازم است که مقادیر منابع اطلاعاتی را مرتب کنیم که $\pi^s$ عملگر جایگشت انتگرال فازی سوگنو می‌باشد. نمادهای $\vee$ و $\wedge$ به ترتیب عمگرهای $\max$ و $\min$ می‌باشد. انتگرال فازی چوکت به صورت زیر تعریف می‌شود\مرجع{murofushi1994non, de1992characterization}:
\begin{eqnarray}
\int_{c} f \circ g = \mathcal{C}_g(f) = \sum_{i = 1}^{n} \left( f(x_{\pi_{(i)}^c}) - f(x_{\pi_{(i-1)}^c}) \right) \cdot g(A_i^c)\label{eq:choquet_integral}\\
f \xrightarrow{\pi^c} f(x_{\pi_1^c}) \leq f(x_{\pi_2^c}) \leq \cdots \leq f(x_{\pi_n^c})\label{eq:fi_choquet_perm_op}\\
A_i^c = \{\pi_i^c, \pi_{i+1}^c, \cdots, \pi_n^c\}\label{eq:fi_choquet_perm_val}\\
\pi^c_0 = 0, \hspace{10pt} x_{\pi^c_0} = 0
\end{eqnarray}
در رابطه‌ی بالا
$f : X \rightarrow \mathbb{R}$
می‌باشد که از وجه تمایز انتگرال فازی چوکت با سوگنو می‌‌باشد و $\pi^c$ عملگر جایگشت انتگرال فازی چوکت می‌باشد.

انتگرال‌های فازی سوگنو و چوکت در حالت کلی دارای تفاوت‌هایی هستند که از جمله‌ی مهم‌ترین این ویژگی‌ها تفاوت تعریف توابع $h$ و $f$ در این انتگرال‌ها می‌باشد که باعث می‌شود انتگرال چوکت برای تبدیل‌های مثبت خطی\زیرنویس{\مق{Positive Linear Transformation}} مناسب باشد؛ بدین معنی که تجمیع اعداد کاردینال\زیرنویس{\مق{Cardinal Aggregation}} (که اعداد دارای مفاهیم واقعی هستند) را انتگرال چوکت بهتر مدل می‌کند در حالی انتگرال سوگنو برای اعداد ترتیبی\زیرنویس{\مق{Ordinal Numbers}} مناسب است\مرجع{grabisch1996application}. به همین علت در این پژوهش انتگرال فازی چوکت مورد استفاده قرار گرفته است زیرا که ورودی انتگرال اعداد کاملا معنی‌دار می‌باشد و اعمال تابع $h$ بروی مقادیر منابع اطلاعاتی، معانی آن‌ها را تغییر داده و اطلاعات بدرد نخوری را تولید خواهد کرد.