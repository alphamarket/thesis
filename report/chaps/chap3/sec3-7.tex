\قسمت{نتیجه‌گیری}
در این فصل مطالبی که برای درک بهتر روش پیشنهادی نیاز است آورده شده است. درروش پیشنهادی از یادگیری $Q$ با سیاست‌های انتخاب عمل بولترمن، $\varepsilon$-حریصانه، انتگرال فازی چوکت و در راستای آزمایش روش پیشنهادی از محیط‌های آزمایشی «پلکان مارپیچ» و «صید و صیاد» استفاده‌شده است. در فصل شش سعی شده با استفاده از مطالب ارائه‌شده در این فصل و روش پیشنهادی که در فصل چهارم آورده شده است و نتایجی که در فصل ۵ بدست آمده است، به جمع‌بندی مطالب پایان‌نامه و روش پیشنهادی پرداخته شود.