\قسمت{اندازه‌گیری‌ و انتگرال‌ فازی}
\label{SEC:FI_PREVIEW}
برای درک روش پیشنهادی نیاز به داشتن اطلاعات پایه در مورد اندازه‌گیری‌های فازی\زیرنویس{\مق{Fuzzy measures}} و انتگرال فازی که با هدف جمع‌آوری اطلاعات\زیرنویس{\مق{Aggregate Information}} ارائه شده‌اند، داریم. اندازه‌گیری‌های فازی پیش‌زمینه‌ای بر انتگرال‌های فازی هستند که قبل از آشنایی با انتگرال‌های فازی نیاز به معرفی اندازه‌گیری‌های فازی داریم. اگر فرض کنیم که تعدای منبع اطلاعاتی
$X = \{x_1, x_2, \cdots, x_n\}$
که این منابع اطلاعاتی اطلاعات دریافتی از حسگرها، پاسخ‌های داده شده به یک پرسشنامه و غیره وجود داشته باشند. اندازه‌گیری فازی میزان ارزش اطلاعاتی این منابع را در اختیار ما می‌گذارد. معمولا اندازه‌گیری فازی توسط تابع
$g : 2^{|X|} \rightarrow [0, 1]$
تعریف می‌شود که ورودی آن یک زیرمجموعه‌ای از منابع اطلاعاتی می‌باشد و خروجی آن یک مقدار مابین صفر و یک که میزان ارزش اطلاعاتی که آن زیرمجموعه از منابع اطلاعاتی ورودی تابع را مشخص می‌کند.
این تابع باید دارای شرایط مرزی تعریف شده و یکنواختی باشد که در ادامه به معرفی شرایط می‌پردازیم\مرجع{torra2006interpretation}:

\begin{enumerate}
\فقره \textbf{شرایط مرزی:}
اگر اطلاعاتی در دست نداریم ارزش صفر را دارد و کلیه اطلاعات ارزش ۱ را دارد.
\begin{equation}
g(\emptyset) = 0,\hspace{10pt} g(X) = 1
\end{equation}
\فقره \textbf{یکنواختی - غیر کاهشی:}
اگر اطلاعات بیشتری به دست آمد ارزش کلیه اطلاعات که شامل اطلاعات جدید می‌باشد حداقل به اندازه زمانی است که آن اطلاعات جدید بدست نیامده است.
\begin{equation}
A \subseteq B \subseteq X \Rightarrow g(A) \leq g(B) \leq 1
\end{equation}
\end{enumerate}

%\begin{definition}[اندازه‌گیری‌های غیرافزایشی]\setstretch{\thebaselinestretch}\label{non_additive_definition}
%اگر فرض کنیم $(X, A)$ فضای قابل اندازه‌گیری باشد که $X$ مجموعه‌ی مرجع\زیرنویس{\مق{Reference Set}} و $A \subseteq X$، آنگاه تابع مجموعه‌ای مانند $\mu$ که $\mu: A \rightarrow [0, 1]$ اندازه‌گیر غیرافزایشی می‌گویند هرگاه شرایط زیر را ارضا کند\مرجع{torra2014non}.
%
%\begin{latin}
%\begin{itemize}
%\item $\mu(\emptyset) = 0, \hspace{10pt} \mu(X) = 1$
%\item $A \subseteq B \Rightarrow \mu(A) \leq \mu(B)$
%\end{itemize}
%\end{latin}
%\end{definition}

مقادیر تابع $g$ یا توسط کارشناس ارائه می‌شود یا توسط یک تابعی مدل می‌شود، یکی از توابع معروف برای تخمین مقادیر تابع $g$ تابع اندازه‌گیری-$\lambda$ سوگنو\زیرنویس{\مق{Sugeno $\lambda$-Measure}} می‌باشد که به صورت زیر تعریف می‌شود\مرجع{leszczynski1985sugeno}.
\begin{equation}\label{eq:sugeno-lambda-measure}
g(\{x_1,\cdots,x_l\}) = {1 \over \lambda}\left[\prod_{i=1}^l(1 + \lambda g_i) - 1\right]
\end{equation}
که در معادله \ref{eq:sugeno-lambda-measure} مقدار $g_i$ها مقادیر ارزش هریک از منابع اطلاعاتی است و $\lambda$ بگونه‌ای تعیین می‌گردد که $g(X) = 1$ شود که این مقدار برابر با جواب معادله‌ی زیر باشد.
\begin{equation}\label{eq:sugeno-lambda-measure:rooting}
\lambda + 1 = \prod_{i=1}^{n} (1 + \lambda g_i), \hspace{1em} \lambda \in (-1, \infty) 
\end{equation}

نکته‌ای که در رابطه با تابع اندازه‌گیری-$\lambda$ سوگنو باید توجه کرد این است که به ازای مقادیر $n$ مختلف باید ریشه‌یابی بروی متغیر $\lambda$ صورت گیرد؛ این ویژگی‌ باعث می‌شود که این تابع در بعضی از کاربردها کارایی نداشته باشد. به عنوان مثال در کار این پژوهش از آنجایی که تعداد عامل‌ها متفاوت می‌باشد، در صورت استفاده از تابع اندازه‌گیری-$\lambda$ باید به ازای هر دفعه تغییر در تعداد عامل‌ها (از آنجایی که مقادیر $n$ تغییر می‌کند) یکبار روی متغیر $\lambda$ ریشه‌یابی صورت بگیرد که امکان همچین ریشه‌یابی‌‌ای بدون سربار محاسباتی سنگین میسر نیست. لذا در کاربردهایی که تعداد $n$ متغیر می‌باشد استفاده از تابع اندازه‌گیری-$\lambda$ عاقلانه‌ نیست.

انتگرال فازی در واقع یک تعمیمی به روش میانگین وزنی\زیرنویس{\مق{Weighted Arithmetic Mean}} می‌باشد بطوری که نه تنها مشخصه‌های مهم تک تک ویژگی‌ها را در نظر می‌گیرد بلکه اطلاعات تعاملات بین ویژگی‌ها را نیز در نظر می‌گیرید\مرجع{tehrani2012preference}. از میان انتگرال‌های فازی دو انتگرال سوگنو\زیرنویس{Sugeno} و چوکت\زیرنویس{Choquet} از الگوریتم‌هایی هستند که می‌توانند بر روی هر اندازه‌گیری‌ فازی مورد استفاده واقع شوند\مرجع{de1992characterization}. فرض کنیم که تابعی چون
$h : X \rightarrow [0, 1]$
وجود دارد که مقادیر منابع اطلاعاتی را به بازه‌ی $[0,1]$ نگاشت می‌کند. در واقع $h$ تابع پشتیبان\زیرنویس{Support} منابع اطلاعاتی می‌باشد. انتگرال فازی سوگنو به صورت
\ref{eq:sugeno_integral} تا \ref{eq:fi_sugeno_perm_val}
تعریف می‌شود\مرجع{grabisch1995fuzzy, de1992characterization}:

\begin{eqnarray}
\int_{s} h \circ g = \mathcal{S}_g(h) = \bigvee_{i=1}^{n} h(x_{\pi_i^s}) \wedge g(A_i^s)\label{eq:sugeno_integral}\\
h \xrightarrow{\pi^s} h(x_{\pi_1^s}) \leq h(x_{\pi_2^s}) \leq \cdots \leq h(x_{\pi_n^s})\label{eq:fi_sugeno_perm_op}\\
A_i^s = \{x_{\pi_i}^s, x_{\pi_{i+1}}^s, \cdots, x_{\pi_n}^s\}\label{eq:fi_sugeno_perm_val}
\end{eqnarray}

در روابط
\ref{eq:sugeno_integral} تا \ref{eq:fi_sugeno_perm_val}
نماد $s$ بیان‌گر «سوگنو» می‌‌باشد. در انتگرال‌ سوگنو لازم است که مقادیر منابع اطلاعاتی را مرتب کنیم که $\pi^s$ عملگر جایگشت انتگرال فازی سوگنو می‌باشد که خروجی مقادیر تابع $h$ را به ترتیب صعودی مرتب می‌کند. نمادهای $\vee$ و $\wedge$ به ترتیب عمگرهای $\max$ و $\min$ می‌باشد. در این انتگرال ابتدا مقادیر دریافتی از منابع اطلاعاتی به تابع پشتیبان $h$ ارسال می‌شود و سپس مقادیر خروجی تابع پشتیبان به ازای همه‌ی اطلاعات دریافتی را به صورت صعودی توسط عملگر جایگشت $\pi^s$ مرتب می‌شود. مجموعه‌ی $A_i^s$ اندیس عناصر مرتب شده مقادیر تابع پشتیبان از اندیس $i$ام تا اندیس $n$ می‌باشد. سپس طبق آنچه که در \ref{eq:sugeno_integral} آمده است از کوچکترین مقدار $h(x_{\pi_1^s})$ شروع می‌کنیم با $g(A_1^s)$ کمینه‌گیری می‌کنیم و سپس می‌رویم به دومین کوچک‌ترین عنصر $h(x_{\pi_2^s})$ و همین کار را تا آخرین (بزرگترین) عنصر تکرار می‌کنیم و سپس یک بیشینه‌گیری روی این مقادیر انجام می‌دهیم که خروجی انتگرال سوگنو می‌شود.

انتگرال فازی چوکت به صورت \ref{eq:choquet_integral} تعریف می‌شود\مرجع{murofushi1994non, de1992characterization}. در این رابطه
$f : X \rightarrow \mathbb{R}$
می‌باشد که از وجه تمایز انتگرال فازی چوکت با سوگنو می‌‌باشد و $\pi^c$ عملگر جایگشت انتگرال فازی چوکت می‌باشد.

\begin{eqnarray}
\int_{c} f \circ g = \mathcal{C}_g(f) = \sum_{i = 1}^{n} \left( f(x_{\pi_{(i)}^c}) - f(x_{\pi_{(i-1)}^c}) \right) \cdot g(A_i^c)\label{eq:choquet_integral}\\
f \xrightarrow{\pi^c} f(x_{\pi_1^c}) \leq f(x_{\pi_2^c}) \leq \cdots \leq f(x_{\pi_n^c})\label{eq:fi_choquet_perm_op}\\
A_i^c = \{x_{\pi_i}^c, x_{\pi_{i+1}}^c, \cdots, x_{\pi_n}^c\}\label{eq:fi_choquet_perm_val}\\
f(x_{\pi^c_0}) = 0\label{eq:fi_choquet_default_vals}
\end{eqnarray}

در روابط
\ref{eq:choquet_integral} تا \ref{eq:fi_choquet_perm_val}
نماد $c$ بیان‌گر «چوکت» می‌‌باشد. عمکرد انتگرال چوکت شباهت نزدیکی با انتگرال سوگنو دارد به این صورت که در انتگرال چوکت مقادیر دریافتی از منابع اطلاعاتی را به تابع پشتیبان $f(\cdot)$ ارسال می‌شود و خروجی این تابع را به ازای تمامی ورودی‌ها توسط عملگر جایگشت $\pi^c$ به صورت صعودی مرتب می‌کنیم. تعریف $A_i^c$ در \ref{eq:fi_choquet_perm_val} مشابه انتگرال سوگنو می‌باشد. طبق تعریف انجام شده در \ref{eq:fi_choquet_default_vals} مقدار 0 را به اول مقادیر مرتب شده $f(\cdot)$ در \ref{eq:fi_choquet_perm_op} اضافه می‌کنیم. سپس توسط رابطه‌ی \ref{eq:choquet_integral} مجموع ضرب اختلاف دو عنصر متوالی مرتب شده $f(\cdot)$ در $g(A_i^c)$ را به عنوان خروجی انتگرال چوکت حساب می‌کنیم.

انتگرال‌های فازی سوگنو و چوکت در حالت کلی دارای تفاوت‌هایی هستند که از جمله‌ی مهم‌ترین این ویژگی‌ها تفاوت تعریف توابع $h$ و $f$ در این انتگرال‌ها می‌باشد که باعث می‌شود انتگرال چوکت برای مسائلی که مقادیر اعداد حائز اهمیت است، مناسب باشد و از طرف دیگر انتگرال سوگنو زمانی مطلوب است که تنها ترتیب اعداد مد نظر باشد\مرجع{grabisch1996application}. به همین علت در این پژوهش انتگرال فازی چوکت مورد استفاده قرار گرفته است زیرا که ورودی انتگرال اعداد کاملا معنی‌دار می‌باشد و اعمال تابع $h$ بروی مقادیر منابع اطلاعاتی، معانی آن‌ها را تغییر داده و اطلاعات نامطلوبی تولید خواهد کرد.