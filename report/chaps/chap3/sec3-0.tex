\قسمت{مقدمه}
در این فصل سعی شده تا موضوعاتی که درروش پیشنهادی به‌کاررفته‌اند و به درک بهتر موضوع کمک می کنند تشریح می‌شوند در این جهت ابتدا در مورد روش های یادگیری و روش یادگیری $Q$ که معمولاً در کارهای یادگیری مشارکتی استفاده  میشود توضیح داده شده است. مطمئناً شناخت یادگیری تقویتی حتی به ‌صورت جزئی می‌تواند در درک یادگیری مشارکتی بسیار مؤثر باشد.
ازآنجایی‌که بررسی عملکرد روش پیشنهادی با روش یادگیری مشارکتی بر مبنای کوتاه‌ترین فاصله تجربه‌شده انجام می‌شود در ادامه به تشریح معیارهای SEP و شک پرداخته خواهد شد. بعد از آن محیط‌های آزمایشی و معیارهای ارزیابی استفاده شده در آزمایش‌های این پژوهش تشریح خواهد شد. همچنین از آنجایی که در این پژوهش از انتگرال‌ فازی چوکت استفاده شده است گذری خلاصه بر اندازه‌گیری‌های فازی و غیرافزایشی شده است و سپس از بین انتگرال‌های فازی دو انتگرال همه کاره سوگنو و چوکت که می‌توان به‌روی هر نوع داده‌ای اعمال کرد را معرفی کردیم و نشان داده شده است چرا در کاربرد مورد استفاده در این پژوهش فقط از انتگرال چوکت بهره برده شده است.