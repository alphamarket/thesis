\قسمت{یادگیری تقویتی}
معمولا به آن سری از روش‌های یادگیری گفته میشود که عامل به دنبال رسیدن به یادگیری از طرق ارتباط با محیط است. در این دسته از روش‌ها به عامل ارزش اعمال گفته نمی‌شود و عامل با تعاملی که با محیط دارد باید ارزش اعمال را کشف نماید.
واژه تقویتی از روش‌های آموزش حیوانات گرفته‌شده است چراکه انسان در یادگیری حیوانات راه ارتباطی مستقیمی نمی‌شناسد پی برای اینکه به عامل‌های حیوانی ارزش اعمال را نشان دهد در صورتی که حیوان عمل مناسبی انجام دهد یک خوراکی به‌عنوان پاداش و در صورت عدم انجام عمل مناسب به روشی عامل را جریمه می‌نماید. عامل حیوانی نیز از طریق پاداش و جریمه دریافتی آموزش می‌بیند.

\begin{algorithm}[t]\setstretch{1.2}
\caption{\rl{الگوریتم یادگیری $Q$}}\label{alg:q-learning}
\begin{latin}
\begin{algorithmic}[1]
\Procedure{Q-LEARNING}{}
\Ensure {Intialize the $Q$ matrix};
\While {not End Of Learning}
	\State {Visit the state $s$;}
	\State {Select an action $a$ based on an action selection policy;} %\Comment{e.g base on Boltzmann, $\epsilon$-greedy, etc.}
	\State {Carry out the $a$ and observe a reward $r$ at the new state $s'$;}
	\State {$Q[s,a] \gets Q[s,a] + \alpha (r + \lambda \max\limits_{a'}(Q[s',a']) - Q[s,a])$;}
	\State {$s \gets s'$;}
\EndWhile
\State \Return $Q$;
\EndProcedure
\end{algorithmic}
\end{latin}
\end{algorithm}

یادگیری تقویتی در سیستم‌های کامپیوتری نیز با ایده گرفتن از یادگیری حیوانات عمل می‌نماید تا عامل‌ها به یادگیری برسند. در این روش مطابق شبه کد \ref{alg:q-learning} فرایند یادگیری به بخش‌هایی با عنوان چرخه یادگیری شکسته میشود. هر چرخه یادگیری از قرار دادن عامل در یک حالت تصادفی شروع و تا رسیدن به یک حالت پایانی ادامه دارد. در طول هر چرخه تا زمان رسیدن به حالت پایانی عامل وظیفه دارد عمل را انتخاب نماید و پس از دریافت پاداش عمل انجام‌شده به بروز رسانی اطلاعات  بپردازد. که بروز رسانی داده‌ها بر اساس فرایند تصادفی مارکو و روش برنامه نوسی پویا انجام میشود. بر اساس تئوری ارائه‌شده در فرایند تصادفی مارکو باید انتخاب عمل به صورتی باشد که انجام هر عمل در هر حالت ضمانت شود. معمولا در یادگیری تقویتی این انتخاب عمل به‌وسیله روش‌هایی چون بولتزمن انجام میشود. در شکل \ref{fig:rl_schema} می‌توان فرایند یادگیری را مشاهده کرد.

\fig[1]{rl_schema}{شمایی از فرایند یادگیری تقویتی در تعامل با محیط}

