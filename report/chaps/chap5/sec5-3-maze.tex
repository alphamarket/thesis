\زیرقسمت{مقایسه‌ در محیط پلکان مارپیچ}
آزمایش‌های مربوط به این قسمت در ۴ بخش صورت گرفته است؛ ۱. مقایسه در سرعت و کیفیت یادگیری، ۲. مقایسه در پیچیدگی زمانی، ۳. مقایسه در میزان باروری، ۴. مقایسه در تاثیر تعداد عامل‌ها میزان کیفیت و سرعت یادگیری می‌باشد.

\زیرزیرقسمت{سیاسیت انتخاب عمل «بولتزمن»}
\پاراگراف{مقایسه در سرعت و کیفیت یادگیری:} نتایج حاصل از اجرای الگوریتم‌ها در محیط پلکان مارپیچ در شکل
\ref{fig:boltzmann/maze.pref.compare.png}
آمده است. در این شکل محور افقی تعداد تلاش‌های یادگیری عامل را نشان می‌دهد که در تلاش اول عامل بدون دانش اولیه شروع به تعامل با محیط می‌کند و در تلاش ۱۰۰۰ام عامل به اجرای خود پایان می‌دهد. محور عمودی نموار میانگین تجمعی تعداد قدم‌های عامل را نشان می‌دهد. اعداد کناری برچسب‌ها (گوشه بالا سمت راست) متوسط تعداد قدم‌ در آخرین تلاش عامل می‌باشد که انتظار می‌رود عامل آگاهی نسبی کاملی از محیط دارد را نشان می‌دهد که این عدد هرچقدر کمتر باشد نشان می‌دهد که عامل در طی رسیدن به هدف تعداد گام کمتری برداشته است و در نتیجه دانش و شناخت بهتری از محیط دارد.

\fig[.8]{boltzmann/maze.pref.compare.png}{مقایسه در سرعت و کیفیت یادگیری در محیط پلکان مارپیچ با تابع بولتزمن}

همانطور که مشاهده می‌شود روش SEP دارای ۲٪ بهبود نسبت به IL می‌باشد در حالی که روش پشنهادی در زمانی که از انتگرال فازی استفاده می‌کند در بدترین حالت دارای ۱۸٪ بهبود و در بهترین حالات دارای ۳۳٪ بهبود می‌باشد که نسبت به روش SEP تقریبا ۹ الی ۱۶ برابر نتیجه را بهبود داده است. در صورتی که از میانگین وزنی بجای انتگرال فازی استفاده شود نتایج با اختلاف اندکی (کمتر از ۱-٪) \زیرخط{بدتر} از یادگیری IL بوده است که نشان ‌می‌دهد که استفاده از انتگرال فازی چقدر می‌تواند نسبت به روش‌های سنتی و معمولی چون میانگین وزنی موثر واقع شود. نتایج این قسمت را می‌توان در جدول
\ref{tab:maze_pref_compare}
خلاصه کرد.

\begin{table}
\centering
\caption{مقایسه در کیفیت یادگیری در محیط پلکان مارپیچ با تابع بولتزمن}\label{tab:maze_pref_compare}
\begin{latin}
\begin{tabular}{|*8{l|}}
\hline
\multicolumn{3}{|c|}{}& \multicolumn{5}{c|}{REFMAT}
\\\hline
& IL & SEP & wsum & fci-mean & fci-max & fci-k-mean & fci-const-one  
\\\hline
IL & \%0.0 & & & & & &  
\\\hline
SEP & \%2.2 & \%0.0 & & & & &  
\\\hline
wsum & \%-0.2 & \%-2.3 & \%0.0 & & & &  
\\\hline
fci-mean & \%14.9 & \%12.5 & \%15.1 & \%0.0 & & &  
\\\hline
fci-max & \%18.0 & \%15.5 & \%18.2 & \%2.7 & \%0.0 & &  
\\\hline
fci-k-mean & \%24.0 & \%21.4 & \%24.2 & \%7.9 & \%5.1 & \%0.0 &  
\\\hline
fci-const-one & \%33.6 & \%30.7 & \%33.8 & \%16.2 & \%13.2 & \%7.7 & \%0.0  
\\\hline
\end{tabular}
\end{latin}
\end{table}

\پاراگراف{مقایسه در پیچیدگی زمانی:} در این قسمت به مقایسه‌‌ی پیچیدگی زمانی روش پیشنهادی با روش SEP مورد بررسی قرار می‌گیرد، برای محاسبه‌ی پیچیدگی زمانی به روش ریاضی کار بسیار دشوار و پرخطایی می‌باشد؛ در اینجا ما بجای محاسبه‌ی پیچیدگی زمانی ریاضی دو الگوریتم از مدت زمانی که طول می‌کشد برنامه در سیستم اجرا و خاتمه یابد استفاده می‌کنیم. در شکل
\ref{fig:boltzmann/maze.time.compare.png}
میانگین زمانی ۲۰ اجرای مستقل برحسب میلی‌ثانبه به ازای هریک از تعداد تلاش‌ها آورده شده است. همان‌طور که در این شکل مشاهده‌ی می‌شود الگوریتم IL دارای حداکثر سرعت اجرا می‌باشد زیرا که هیچ سربار محاسباتی یادگیری مشترک را ندارد؛ هدف یادگیری اشتراکی این است که می‌خواهد در ازای یک سری سربار محاسباتی کیفیت و سرعت «یادگیری» عامل‌ها را افزایش دهد. با در نظر داشتن این موضوع همانطور که قبلا دیدیم روش پیشنهادی سرعت و کیفیت یادگیری را بیشتر از روش SEP افزایش می‌دهد و در اینجا نیز می‌بینیم که دارای پیچیدگی زمانی کمتری نسبت به روش SEP می‌باشد که نشان از بهینه‌گی روش پیشنهادی نسبت به روش SEP می‌دهد.

\fig[0.6]{boltzmann/maze.time.compare.png}{مقایسه در پیچیدگی زمانی روش‌ها به ازای تعداد تلاش‌های متفاوت برحسب میلی‌ثانیه}

\پاراگراف{مقایسه در میزان باروری:}
\begin{definition}[سرعت باروری]\setstretch{\thebaselinestretch}\label{def:fertility_speed}
اگر فرض کنیم الگوریتم یادگیری تقویتی $\psi_Q(\mathcal{E})$ وجود دارد که در محیط $\mathcal{E}$ فعالیت می کند و دانش خود را در جدولی مانند $Q$ ذخیره می‌کند، سرعت باروری الگوریتم $\psi_Q(\mathcal{E})$ را سرعت همگرایی حداکثر مقدار جدول $Q$ به سمت حداکثر پاداش محیط قابل دریافت تعریف می‌کنیم.
\end{definition}
\begin{definition}[میزان باروری]\setstretch{\thebaselinestretch}\label{def:fertility_rate}
انتگرال سرعت باوری را میزان باروری الگوریتم $\psi_Q(\mathcal{E})$ که در محیط $\mathcal{E}$ فعالیت می کند و دانش خود را در جدولی مانند $Q$ ذخیره می‌کند، تعریف می‌کنیم.
\end{definition}

طبق تعاریف \ref{def:fertility_speed} و \ref{def:fertility_rate} الگوریتمی میزان باروری بیشتری دارد که سریع‌تر مقادیر جدول $Q$ خود را به سمت بیشنه مقداری که می‌توانند داشته باشد(یعنی بیشنه پاداشی که از محیط می‌توانند کسب کند) سوق دهد. معمولا این در الگوریتم‌های یادگیری تقویتی $Q$ این کار با نتظیم مقدار سرعت یادگیری $\alpha$ صورت می‌گیرد که باعث می‌شود الگوریتم‌ها با سرعت بیشتری به یادگیری نحوه‌ی تعامل با محیط بپردازند. لذا در شرایط یکسان می‌توان گفت الگوریتمی بهتر عمل می‌کند که نحوه‌ی تعامل با محیط را سریع‌تر نسبت به دیگر الگوریتم‌ها یاد می‌گیرید. بدین منظور تعریف \ref{def:fertility_rate} به نظر می‌رسد که می‌تواند معیار مناسبی برای مقایسه‌ی سرعت یادگیری الگوریتم‌ها باشد.

در شکل

آورده شده است حداکثر میزان جدول $Q$ روش‌ها در هر تلاش آورده شده است. همانطور که قبلا در تعریف محیط پلکان مارپیچ آورده شده است حداکثر مقدار پاداش این محیط مقدار ۱۰ می‌باشد لذا همان‌طور که مشاهده می‌شود الگوریتم‌ها با شیب‌های متفاتی حداکثر مقدار جداول خود را به سمت حداکثر مقدار پاداش قابل دریافت از محیط سوق می‌دهند. در این شکل سرعت باروری شیب نمودار در هر تلاش می‌باشد و میزان باروری مساحت زیر نمودار می‌باشد.

\fig[.3]{boltzmann/maze.qtable.max.compare.png}{نمودار باروری الگوریتم‌ها مختلف}