\قسمت{نتیجه‌گیری}
در فصل به بررسی نتایج آزمایش‌های گوناگونی که برای مقایسه‌ی روش پیشنهادی با روش SEP آورده شده بودند پرداختیم؛ مقایسه‌ها بر اساس ۴ معیار صورت گرفت: ۱. مقایسه در سرعت و کیفیت یادگیری، ۲. مقایسه در پیچیدگی زمانی، ۳. مقایسه در میزان باروری، ۴. مقایسه تاثیر تعداد عامل‌ها بر میزان کیفیت و سرعت یادگیری. همانطور که دیدیم این روش پیشنهادی در هر ۴ معیار نسبت به روش SEP برتری قابل توجهی داشت. همچنین به مقایسه‌ی تاثیر استفاده از سیاست‌های انتخاب عمل بولتزمن و $\varepsilon$-حریصانه پرداختیم و نشان دادیم که استفاده از بولتزمن نسبت به $\varepsilon$-حریصانه کیفیت و سرعت یادگیری بسیار بهتری بدست می‌دهد. همچنین نشان دادیم که معیار خبرگی معرفی شده در تعریف \ref{experties_definition} مستقل از تعداد و اندازه‌ی افراز‌های محیط می‌باشد، سپس با تحلیل نتایج آزمایش‌ها به مطالب این بخش خاتمه دادیم. در فصل بعدی به جمع‌بندی مطالب و دست‌آورد‌های این پژوهش می‌پردازیم.