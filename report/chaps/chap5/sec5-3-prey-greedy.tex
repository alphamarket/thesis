\زیرزیرقسمت{سیاست انتخاب عمل «$\varepsilon$-حریصانه»}
\پاراگراف{مقایسه در سرعت و کیفیت یادگیری:} نتایج حاصل از اجرای الگوریتم‌ها در محیط صید و صیاد در شکل
\ref{fig:greedy/prey.pref.compare.png}
آمده است. شرایط این آزمایش به مشابه شرایط آزمایش با تابع بولتزمن می‌باشد.

\fig[.8]{greedy/prey.pref.compare.png}{مقایسه در سرعت و کیفیت یادگیری با تابع $\varepsilon$-حریصانه در محیط صید و صیاد}

\begin{table}[h!]
\centering
\caption{مقایسه در میزان درصد بهبود کیفیت یادگیری در محیط صید و صیاد با تابع $\varepsilon$-حریصانه}\label{tab:prey_pref_compare_greedy}
\begin{latin}
\begin{tabular}{|*8{l|}}
\hline
\multicolumn{3}{|c|}{}& \multicolumn{5}{c|}{REFMAT}
\\\hline
& IL & SEP & wsum & fci-mean & fci-max & fci-k-mean & fci-const-one
\\\hline
IL &0.0 & & & & & &
\\\hline
SEP &-1.3 &0.0 & & & & &
\\\hline
wsum &16.7 &18.3 &0.0 & & & &
\\\hline
fci-mean &31.7 &33.4 &12.8 &0.0 & & &
\\\hline
fci-max &36.8 &38.6 &17.2 &3.9 &0.0 & &
\\\hline
fci-k-mean &39.6 &41.5 &19.6 &6.0 &2.1 &0.0 &
\\\hline
fci-const-one &53.5 &55.5 &31.5 &16.6 &12.2 &10.0 &0.0
\\\hline
\end{tabular}
\end{latin}
\end{table}

همانطور که خلاصه‌ی این نتایج را در جدول
\ref{tab:prey_pref_compare_greedy}
مشاهده می‌کنیم می‌بینیم که همانند نتایج بدست آمده در آزمایش‌های قبلی روش پیشنهادی با حدود ٪۵۳ بهبود نسبت به IL داشته است درحالی که روش SEP با تابع $\varepsilon$-حریصانه نه تنها نتایج بهبود داده‌ نشده است بلکه حدود ٪۱- بدتر شده است!

\پاراگراف{مقایسه در سرعت اجرا:} در شکل
\ref{fig:greedy/prey.time.compare.png}
نیز می‌بینیم که در محیط صید و صیاد نیز روش پیشنهادی دارای سرعت اجرای بیشتری نسبت به روش SEP می‌باشد که نشان از بهینه‌گی روش پیشنهادی نسبت به روش SEP می‌دهد.

\fig[0.8]{greedy/prey.time.compare.png}{مقایسه در سرعت اجرای روش‌ها به ازای تعداد تلاش‌های متفاوت برحسب میلی‌ثانیه با تابع $\varepsilon$-حریصانه در محیط صید و صیاد}

\پاراگراف{مقایسه در میزان باروری:}
در شکل
\ref{fig:greedy/prey.qtable.max.compare.png}
میزان باروری روش پیشنهادی از دیگر روش‌ها بیشتر بوده و همانند آزمایش‌های قبلی در اینجا نیز نشان داده شده است که روش SEP در زمانی که به صورت تصادفی محیط را کاوش کند کمترین باروری را دارد که مطابق دلایل ذکر شده در مقایسه‌ی میزان باروری در آزمایش‌های گذشته این مساله نشان از ضعف بزرگ روش SEP می‌دهد.

\fig[.8]{greedy/prey.qtable.max.compare.png}{نمودار باروری الگوریتم‌ها مختلف با تابع $\varepsilon$-حریصانه در محیط صید و صیاد}

\پاراگراف{مقایسه تاثیر تعداد عامل‌ها میزان کیفیت و سرعت یادگیری:} همان‌طور که در شکل
\ref{fig:greedy/prey.multi-agent.pref.compare.png}
آمده است، روش SEP در زمانی ۲۰ عامل در حال یادگیری و اشتراک گذاری دانش‌های خود هستند نسبت به زمانی که فحقط ۲ عامل در حال تعامل مشارکتی با محیط هستند 9-٪ در خروجی الگوریتم تاثیر منفی داشته است؛ بدین معنی که در زمانی که از تابع $\varepsilon$-حریصانه استفاده شود روش SEP به افزایش تعداد عامل فقط منجر به بدتر شدن عملکرد عامل‌ها در یادگیری مشارکتی می‌شود. این در حالی است که در همین شرایط میزان بهبود نتیجه‌ی روش پیشنهادی ٪۵۵ می‌باشد. نشان می‌دهد روش پیشنهادی در ازای افزایش تعداد عامل‌ها به دلیل اینکه دانش جمعی نیز افزایش می‌یابد کیفیت خروجی آن نیز بطور چشم‌گیری بهتر می‌شود. در حالی که در روش SEP اگر کار نتایج بدتر نشود بهتر نمی‌شود که از ضعف بزرگ روش SEP خبر می‌دهد.

\fig[.8]{greedy/prey.multi-agent.pref.compare.png}{مقایسه تاثیر تعداد عامل‌ها میزان کیفیت و سرعت یادگیری با تابع $\varepsilon$-حریصانه در محیط صید و صیاد}

\پاراگراف{نتیجه‌گیری:} نتیجه‌ای که از مقایسه‌ی روش پیشنهادی در هر چهار مقایسه‌ی بالا می‌توان گرفت همچون نتیجه‌ای که از نتایج تابع بولتزمن، روش پیشنهادی بهبود چشم‌گیری به روش SEP در محیط صید و صیاد و سیاست انتخاب عمل حریصانه داده است.