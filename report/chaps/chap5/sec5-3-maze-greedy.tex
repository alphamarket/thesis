\زیرزیرقسمت{سیاست انتخاب عمل «$\varepsilon$-حریصانه»}
\پاراگراف{مقایسه در سرعت و کیفیت یادگیری:} نتایج حاصل از اجرای الگوریتم‌ها در محیط پلکان مارپیچ در شکل
\ref{fig:greedy/maze.pref.compare.png}
آمده است. در این شکل محور افقی تعداد تلاش‌های یادگیری عامل را نشان می‌دهد که در تلاش اول عامل بدون دانش اولیه شروع به تعامل با محیط می‌کند و در تلاش ۱۰۰۰ام عامل به اجرای خود پایان می‌دهد. محور عمودی نموار میانگین تجمعی تعداد قدم‌های عامل را نشان می‌دهد. اعداد کناری برچسب‌ها (گوشه بالا سمت راست) متوسط تعداد قدم‌ در آخرین تلاش عامل می‌باشد که انتظار می‌رود عامل آگاهی نسبی کاملی از محیط دارد را نشان می‌دهد که این عدد هرچقدر کمتر باشد نشان می‌دهد که عامل در طی رسیدن به هدف تعداد گام کمتری برداشته است و در نتیجه دانش و شناخت بهتری از محیط دارد.

\fig[.8]{greedy/maze.pref.compare.png}{مقایسه در سرعت و کیفیت یادگیری در محیط پلکان مارپیچ با تابع حریصانه}

همانطور که مشاهده می‌شود روش SEP دارای ۲٪ بهبود نسبت به IL می‌باشد در حالی که روش پشنهادی در زمانی که از انتگرال فازی استفاده می‌کند در بدترین حالت دارای ۱۸٪ بهبود و در بهترین حالات دارای ۳۳٪ بهبود می‌باشد که نسبت به روش SEP تقریبا ۹ الی ۱۶ برابر نتیجه را بهبود داده است. در صورتی که از میانگین وزنی بجای انتگرال فازی استفاده شود نتایج با اختلاف اندکی (کمتر از ۱-٪) \زیرخط{بدتر} از یادگیری IL بوده است که نشان ‌می‌دهد که استفاده از انتگرال فازی چقدر می‌تواند نسبت به روش‌های سنتی و معمولی چون میانگین وزنی موثر واقع شود. نتایج این قسمت را می‌توان در جدول
\ref{tab:maze_pref_compare_greedy}
خلاصه کرد.

\begin{table}
\centering
\caption{مقایسه در میزان بهبود کیفیت یادگیری در محیط پلکان مارپیچ با تابع حریصانه}\label{tab:maze_pref_compare_greedy}
\begin{latin}
\begin{tabular}{|*8{l|}}
\hline
\multicolumn{3}{|c|}{}& \multicolumn{5}{c|}{REFMAT}
\\\hline
& IL & SEP & wsum & fci-mean & fci-max & fci-k-mean & fci-const-one  
\\\hline
IL & \%0.0 & & & & & &  
\\\hline
SEP & \%2.2 & \%0.0 & & & & &  
\\\hline
wsum & \%-0.2 & \%-2.3 & \%0.0 & & & &  
\\\hline
fci-mean & \%14.9 & \%12.5 & \%15.1 & \%0.0 & & &  
\\\hline
fci-max & \%18.0 & \%15.5 & \%18.2 & \%2.7 & \%0.0 & &  
\\\hline
fci-k-mean & \%24.0 & \%21.4 & \%24.2 & \%7.9 & \%5.1 & \%0.0 &  
\\\hline
fci-const-one & \%33.6 & \%30.7 & \%33.8 & \%16.2 & \%13.2 & \%7.7 & \%0.0  
\\\hline
\end{tabular}
\end{latin}
\end{table}

\پاراگراف{نتیجه‌گیری:} نتیجه‌ای که از مقایسه‌ی روش پیشنهادی در هر چهار مقایسه‌ی بالا می‌توان گرفت این است که روش پیشنهادی بهبود چشم‌گیری به روش SEP در محیط پلکان مارپیچ و سیاست انتخاب عمل بولتزمن داده است.