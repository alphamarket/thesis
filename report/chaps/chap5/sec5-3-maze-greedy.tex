\زیرزیرقسمت{سیاست انتخاب عمل «$\varepsilon$-حریصانه»}
\پاراگراف{مقایسه در سرعت و کیفیت یادگیری:} نتایج حاصل از اجرای الگوریتم‌ها در محیط پلکان مارپیچ در شکل
\ref{fig:greedy/maze.pref.compare.png}
آمده است. شرایط این آزمایش به مشابه شرایط آزمایش با تابع بولتزمن می‌باشد.

\fig[.8]{greedy/maze.pref.compare.png}{مقایسه در سرعت و کیفیت یادگیری در محیط پلکان مارپیچ با تابع حریصانه}

همانطور که مشاهده می‌شود روش SEP دارای ۳-٪ بهبود نسبت به IL می‌باشد در حالی که روش پشنهادی در زمانی که از انتگرال فازی استفاده می‌کند در بدترین حالت دارای ۰/۶٪ بهبود و در بهترین حالات دارای ۳۴٪ بهبود می‌باشد که نسبت به روش SEP تقریبا ۴ الی ۳۸ برابر نتیجه را بهبود داده است. در صورتی که از میانگین وزنی بجای انتگرال فازی استفاده شود نتایج با اختلافی حدود ۷-٪ \زیرخط{بدتر} از یادگیری IL بوده است که نشان ‌می‌دهد که استفاده از انتگرال فازی چقدر می‌تواند نسبت به روش‌های سنتی و معمولی چون میانگین وزنی موثر واقع شود. البته در شکل \ref{fig:greedy/maze.pref.compare.png} باید توجه کرد که روش SEP در همان ابتدای کار خود به شدت میانگین حرکت عامل‌ها را کاهش داده ولی به دلیل ماهیت الگوریتم SEP اشباع جداول الگوریتم توانایی ادامه‌ی سرشکن کردن بیشتر میانگین حرکت عامل‌ها را ندارد. میانگین نتایج این قسمت را می‌توان در جدول
\ref{tab:maze_pref_compare_greedy}
خلاصه کرد.

\begin{table}
\centering
\caption{مقایسه در میزان بهبود کیفیت یادگیری در محیط پلکان مارپیچ با تابع حریصانه}\label{tab:maze_pref_compare_greedy}
\begin{latin}
\begin{tabular}{|*8{l|}}
\hline
\multicolumn{3}{|c|}{}& \multicolumn{5}{c|}{REFMAT}
\\\hline
& IL & SEP & wsum & fci-mean & fci-max & fci-k-mean & fci-const-one  
\\\hline
IL & \%0.0 & & & & & &  
\\\hline
SEP & \%-3.0 & \%0.0 & & & & &  
\\\hline
wsum & \%-7.3 & \%-4.4 & \%0.0 & & & &  
\\\hline
fci-mean & \%0.6 & \%3.7 & \%8.5 & \%0.0 & & &  
\\\hline
fci-max & \%12.2 & \%15.6 & \%20.9 & \%11.5 & \%0.0 & &  
\\\hline
fci-k-mean & \%24.0 & \%27.8 & \%33.7 & \%23.2 & \%10.5 & \%0.0 &  
\\\hline
fci-const-one & \%34.5 & \%38.7 & \%45.1 & \%33.7 & \%20.0 & \%8.5 & \%0.0  
\\\hline
\end{tabular}
\end{latin}
\end{table}


\پاراگراف{نتیجه‌گیری:} نتیجه‌ای که از مقایسه‌ی روش پیشنهادی در هر چهار مقایسه‌ی بالا می‌توان گرفت این است که روش پیشنهادی بهبود چشم‌گیری به روش SEP در محیط پلکان مارپیچ و سیاست انتخاب عمل بولتزمن داده است.