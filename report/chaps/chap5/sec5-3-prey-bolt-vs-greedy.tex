\زیرزیرقسمت{مقایسه‌ی بین نتایج حاصل از سیاست انتخاب عمل بولتزمن و $\varepsilon$-حریصانه}
در حالت کلی همانند محیط پلکان مارپیچ در محیط صید و صیاد تابع بولتزمن نتایج یکنواتر و پایدارتری نسبت به تابع $\varepsilon$-حریصانه از خود نشان داد و در هر دوی این توابع روش پیشنهادی نتیجه‌ی بهتری نسبت به روش SEP ارائه داد. در این قسمت به مقایسه‌ی نتایج بدست آمده توسط هر دو روش در هر دو سیاست انتخاب عمل می‌پردازیم.

\پاراگراف{مقایسه در سرعت و کیفیت یادگیری:} مقایسه‌ی این قسمت را بطور خلاصه می‌توان در جدول
\ref{tab:prey_pref_greedy_bolt_comp}
دید. که نسبت کیفیت نتیجه‌ی حاصل از تابع $\varepsilon$-حریصانه نسبت به تابع بولتزمن همگی بزرگتر از ۱ می‌باشد، که نشان می‌دهد که استفاده از تابع $\varepsilon$-حریصانه در کیفیت خروجی تاثیری منفی دارد.

\begin{table}
\centering
\caption{مقایسه در سرعت و کیفیت یادگیری نسبت کیفیت نتیجه‌ی حاصل از تابع $\varepsilon$-حریصانه نسبت به تابع بولتزمن}\label{tab:prey_pref_greedy_bolt_comp}
\begin{latin}
\begin{tabular}{*4{c|}}
\multicolumn{2}{c}{} &\multicolumn{2}{c}{Boltzmann}
\\
\multicolumn{2}{c|}{}& SEP & REFMAT
\\\hline
\parbox[t]{2mm}{\multirow{2}{*}{\rotatebox[origin=c]{90}{$\varepsilon$-greedy}}} & SEP & 8.07 & 9.75
\\
& REFMAT & 5.19 & 6.27
\\\hline
\end{tabular}
\end{latin}
\end{table}

\پاراگراف{مقایسه در سرعت اجرا:}
در جدول
\ref{tab:prey_time_greedy_bolt_comp}
نسبت میانگین سرعت اجرای روش‌ها آمده است، که نشان می‌دهد هر روش در زمانی که از تابع $\varepsilon$-حریصانه استفاده می‌کند زمان بیشتری را تلف می‌کند نسبت به زمانی که از تابع بولتزمن استفاده می‌کند. همانند آنچه که در محیط پلکان مارپیچ مشاهده کردیم در اینجا نیز تابع بولزمن سریع‌تر عامل را به سمت اهداف هدایت می‌کند.

\begin{table}
\centering
\caption{مقایسه در نسبت میانگین سرعت اجرا حاصل از استفاده تابع $\varepsilon$-حریصانه نسبت به تابع بولتزمن}\label{tab:prey_time_greedy_bolt_comp}
\begin{latin}
\begin{tabular}{*5{c|}}
\multicolumn{2}{c}{} &\multicolumn{3}{c}{Boltzmann}
\\
\multicolumn{2}{c|}{}& SEP & REFMAT & IL
\\\hline
\parbox[t]{2mm}{\multirow{3}{*}{\rotatebox[origin=c]{90}{$\varepsilon$-greedy}}} & SEP & 3.27 & 4.10 & 6.95
\\
& REFMAT & 2.74 & 3.44 & 5.83
\\
& IL & 1.31 & 1.65 & 2.79
\\\hline
\end{tabular}
\end{latin}
\end{table}

\پاراگراف{مقایسه در میزان باروری:} همانطور که در جدول
\ref{tab:prey_qmax_greedy_bolt_comp}
آمده است استفاده از تابع $\varepsilon$-حریصانه به نسبت تابع بولتزمن باعث باروری بیشتر جدول $Q$ می‌شود.

\begin{table}
\centering
\caption{مقایسه در نسبت میزان باروری حاصل از استفاده تابع $\varepsilon$-حریصانه نسبت به تابع بولتزمن}\label{tab:prey_qmax_greedy_bolt_comp}
\begin{latin}
\begin{tabular}{*5{c|}}
\multicolumn{2}{c}{} &\multicolumn{3}{c}{Boltzmann}
\\
\multicolumn{2}{c|}{}& SEP & REFMAT & IL
\\\hline
\parbox[t]{2mm}{\multirow{3}{*}{\rotatebox[origin=c]{90}{$\varepsilon$-greedy}}} & SEP & 1.19 & 0.82 & 1.07
\\
& REFMAT & 1.49 & 1.03 & 1.35
\\
& IL & 1.17 & 0.80 & 1.05
\\\hline
\end{tabular}
\end{latin}
\end{table}

\پاراگراف{مقایسه تاثیر تعداد عامل‌ها بر میزان کیفیت و سرعت یادگیری:} همانطور که در جدول
\ref{tab:prey_agents_greedy_bolt_comp}
مشاهده می‌شود در زمانی که از تابع $\varepsilon$-حریصانه استفاده می‌شود در روش پیشنهادی تاثیر تعداد عامل‌ها به مراتب بیشتر از زمانی است که از تابع بولتزمن استفاده می‌کنیم. این در حالی می‌باشد که در روش SEP اضافه کردن عامل‌ها به محیط نه تنها به بهبود دانش خروجی الگوریتم کمکی نمی‌کند بلکه نتایج را بدتر نیز می‌کند!

\begin{table}[t]
\centering
\caption{مقایسه در نسبت شیب تاثیر تعداد عامل‌ها میزان کیفیت نتیجه‌ی حاصل از تابع $\varepsilon$-حریصانه نسبت به تابع بولتزمن}\label{tab:prey_agents_greedy_bolt_comp}
\begin{latin}
\begin{tabular}{*4{c|}}
\multicolumn{2}{c}{} &\multicolumn{2}{c}{Boltzmann}
\\
\multicolumn{2}{c|}{}& SEP & REFMAT
\\\hline
\parbox[t]{2mm}{\multirow{2}{*}{\rotatebox[origin=c]{90}{$\varepsilon$-greedy}}} & SEP & -2.52 & -0.07
\\
& REFMAT & 379.32 & 10.65
\\\hline
\end{tabular}
\end{latin}
\end{table}