\قسمت{مقایسه‌ی روش پیشنهادی با روش کوتاه‌ترین مسیر تجربه شده}
در این قسمت به مقایسه‌ی روش پیشنهادی با روش «کوتاه‌ترین مسیر تجربه شده» که از بروزترین تکنیک ارائه شده در این شاخه از یادگیری مشارکتی می‌باشد می‌پردازیم\مرجع{mohammad2015speedup}. کلیه‌ی این آزمایش‌ها در دو محیط «پلکان مارپیچ» و «صید و صیاد» صورت گرفته است. نتیجه‌ی هر آزمایش حاصل میانگین ۲۰ اجرای مستقل تمامی الگوریتم‌ها می‌باشد. همچنین به غیر از مواردی که صراحتا قید شده است تعداد عامل‌ها ۳ عدد می‌باشد -- البته بدیهی است که یادگیری مستقل تک عامله(یا به اختصار IL\زیرنویس{\مق{Individual Learning}}) شامل این قاعده نمی‌باشد. همچنین در کلیه‌ی آزمایش‌ها عامل‌ها از ۲۰۰ چرخه یادگیری بهره می‌برند و در هر چرخه عامل ۵ بار تلاش می‌کند که در مجموع ۱۰۰۰ تلاش صورت می‌گیرد. کلیه‌ی پارامتر‌های مربوط قسمت یادگیری مستقل الگوریتم \ref{alg:proposed} اعمال شده در آزمایشات این فصل منطبق بر پارامترهای تعریف شده در\مرجع{mohammad2015speedup} می‌باشد که نتایج قایل قیاس باشند. در ضمن در این فصل اختصارهای جدول \ref{tab:abbreviation} را نیز داریم.

\begin{table}[t]
\centering
\caption{لیست اختصارهای استفاده شده در این فصل}\label{tab:abbreviation}
\begin{tabular}{r|r}
اختصار & معنی
\\\midrule
\lr{REFMAT} & روش پیشنهادی\\
\lr{IL} & یادگیری مستقل تک عامله\\
\lr{SEP} & روش کوتاه‌ترین مسیر تجربه شده\\
\midrule
\lr{wsum} & میانگین وزنی\\
\lr{fci-max} & الگوریتم \مق{Max} به عنوان مدل کننده‌ی تابع $g(\cdot)$\\
\lr{fci-mean} & الگوریتم \مق{Mean} به عنوان مدل کننده‌ی تابع $g(\cdot)$\\
\lr{fci-k-mean} & الگوریتم \مق{K-Mean} به عنوان مدل کننده‌ی تابع $g(\cdot)$\\
\lr{fci-const-one} & الگوریتم \مق{Const-One} به عنوان مدل کننده‌ی تابع $g(\cdot)$\\
\midrule
\lr{Rand-Walk} & جستجوی کاملا مکاشفانه محیط\\
\bottomrule
\end{tabular}
\end{table}

در این فصل در حالت کلی ما در دو بخش سیاست انتخاب عمل «بولتزمن» و «$\varepsilon-$حریصانه» (که از این به بعد به اختصار «تابع بولتزمن» و «تابع حریصانه» خطاب خواهیم کرد.) به مقایسه‌ی نتایج می‌پردازیم. طبق آنچه که در ادامه مشاهده خواهیم کرد چه در صورت استفاده از تابع بولتزمن و چه تایع حریصانه روش پیشنهادی چه در سرعت یادگیری و چه در کیفیت یادگیری بهتر از روش SEP می‌باشد.

برای اینکه نشان دهیم که استفاده از انتگرال فازی در بهبود نتیجه تاثیر بسزایی دارد از تابع میانگین وزنی (یا به اختصار \مق{wsum}\زیرنویس{\مق{Weighted Sum}}) نیز استفاده کرده‌ایم. بدین صورت که بجای اینکه بعد از استخراج میزان خبرگی هر عامل جداول $Q$ آن‌ها را به نسبت خبرگی‌ای که دارند باهم جمع می‌کنیم تا جدول $Q$ مشارکتی تولید شود. تابع میانگین وزنی روشی است که در پژوهش‌های اخیر به کررات از آن استفاده کرده‌اند\مرجع{mohammad2015speedup, pakizeh2013multi, ahmadabadi2000expertness}. که یکی از اهداف ما در این پژوهش نمایش قدرت انتگرال‌های فازی در کاربرد‌های مختلف می‌باشد به‌طوری که اگر در پژوهش‌های قبلی به درستی از انتگرال فازی بهره برده می‌شد می‌توان به قطع گفت که می‌توانستند نتایج بهتری را بدست بیاورند.

\زیرقسمت{مقایسه‌ در محیط پلکان مارپیچ}
آزمایش‌های مربوط به این قسمت در ۴ بخش صورت گرفته است؛ ۱. مقایسه در سرعت و کیفیت یادگیری، ۲. مقایسه در پیچیدگی زمانی، ۳. مقایسه در میزان باروری، ۴. مقایسه تاثیر تعداد عامل‌ها میزان کیفیت و سرعت یادگیری می‌باشد.

\subimport{\curdir}{sec5-3-maze-boltzmann}
\subimport{\curdir}{sec5-3-maze-greedy}
\subimport{\curdir}{sec5-3-maze-bolt-vs-greedy}
\زیرقسمت{مقایسه‌ در محیط صید و صیاد}
آزمایش‌های مربوط به این قسمت در ۴ بخش صورت گرفته است؛ ۱. مقایسه در سرعت و کیفیت یادگیری، ۲. مقایسه در پیچیدگی زمانی، ۳. مقایسه در میزان باروری، ۴. مقایسه تاثیر تعداد عامل‌ها میزان کیفیت و سرعت یادگیری می‌باشد.

\subimport{\curdir}{sec5-3-prey-boltzmann}
\subimport{\curdir}{sec5-3-prey-greedy}
\subimport{\curdir}{sec5-3-prey-bolt-vs-greedy}