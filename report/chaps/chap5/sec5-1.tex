\قسمت{مقدمه}
در این فصل به ارائه‌ی آزمایش‌های صورت گرفته بروی روش پیشنهادی می‌پردازیم و در طی این آزمایش‌ها روش‌ پیشنهادی را با روش کوتاه‌ترین مسیر تجربه شده(یا به اختصار \مق{SEP}) مقایسه می‌کنیم که آخرین و مدرن‌ترین روش‌ ارائه شده در جهت بهبود یادگیری مشارکتی می‌باشد\مرجع{mohammad2015speedup}. آزما‌یش‌ها بروی دو محیط «پلکان مارپیچ» و «صید و صیاد» صورت گرفته است. آزمایش‌ها به دو دسته تقسیم بندی شده است؛ دسته اول آزمایش‌هایی که روش پیشنهادی را در مقابل روش \مق{SEP} قرار می‌دهد و عملکرد روش پیشنهادی را مورد سنجش قرار می‌دهد. دسته دوم آزمایش‌ها مربوط به آزمون رفتار روش پیشنهادی در صورت تغییر در پارامتر‌های متخلف آن می‌باشد. همچنین اثر استفاده از سیاست‌های انتخاب عمل مختلف در الگوریتم \ref{alg:proposed} نیز بررسی شده است. در روش‌های مرتبط مدرن قبلی \مرجع{mohammad2015speedup, pakizeh2013multi} که این پژوهش ادامه‌ی کار آن‌ها می‌باشد فقط از سیاست انتخاب عمل \مق{Boltzmann} استفاده کرده‌اند؛ در این پژوهش علاوه بر \مق{Boltzmann} تاثیر استفاده از روش $\varepsilon-\text{greedy}$ بروی هردو روش پیشنهادی و \مق{SEP} نیز مورد بررسی واقع گردیده است.