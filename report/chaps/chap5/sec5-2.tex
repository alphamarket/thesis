\قسمت{رفتار الگوریتم‌های معرفی شده برای $g(\cdot)$}
در این قسمت به بررسی رفتار الگوریتم‌های \ref{alg:g:const-one} تا \ref{alg:g:k-mean} معرفی شده برای $g(\cdot)$ بروی دو توزیع فرضی خواهیم پرداخت، زیرا که در طی اجرای آزمایش‌های مختلف نتایج تاثیر این توابع بر اجرای الگوریتم پیشنهادی \ref{alg:proposed} آورده شده است، لذا بجهت درکت علت تاثیرات مختلف هرکدام ازین توابع بروی نتیجه‌ی الگوریتم پیشنهادی در آزمایش‌ها، درک نحوه‌ی رفتار الگوریتم‌های \ref{alg:g:const-one} تا \ref{alg:g:k-mean} ضروری است.

برای نمایش نحوه‌ی رفتار هرکدام از الگوریتم‌ها دو توزیع فرضی شکل \ref{fig:fci-g-dists} فرض شده است. در صورت اعمال الگوریتم‌های \ref{alg:g:const-one} تا \ref{alg:g:k-mean} بروی دو توزیع آورده شده در شکل \ref{fig:fci-g-dists} توزیع‌های جدیدی بصورت آنچه که در شکل \ref{fig:fci-g-functions} آمده است بدست می‌آیند. همانطور که در شکل \ref{fig:fci-g-functions} می‌بینیم اعمال الگوریتم \مق{Const-One} بروی دو توزیع مقدار ثابت ۱ را برمی‌گرداند. اعمال الگوریتم \مق{Max} در هر نقطه حداکثر مقدار هر دو توزیع را برمی‌گرداند. الگوریتم \مق{Mean} میانگین دو توزیع را در هر نقطه حساب می‌کند و در نهایت الگوریتم \مق{K-Mean} میانگین $k$ام هردو توزیع را محاسبه میکند که همانطور که می‌بینیم میانگین $k$ام به سبب ماهیت الگوریتم به سمت بیشترین مقدار پیش‌قدر\زیرنویس{Bias} می‌باشد.

\fig{fci-g-dists}{دو توزیع فرضی بجهت نمایش نحوه‌ی رفتار الگوریتم‌های \ref{alg:g:const-one} تا \ref{alg:g:k-mean} بروی آن‌ها.}
\fig{fci-g-functions}{نمایش توزیع‌های جدید بدست آمده بعد از اعمال الگوریتم‌های \ref{alg:g:const-one} تا \ref{alg:g:k-mean} بروی دو توزیع فرضی شکل  \ref{fig:fci-g-dists}}

\زیرقسمت{تعابیر مختلف انتگرال فازی چوکت از داده‌ها برمبنای $g(\cdot)$}
الگوریتم‌های \ref{alg:g:const-one} تا \ref{alg:g:k-mean} به تنهایی فقط در نقش یک عملگر بازی می‌کند ولی در هنگام ترکیب دانش با انتگرال فازی چوکت به دانش خروجی الگوریتم از دیدگاه‌های متفاوتی نگاه می‌کنند. از آنجایی که در فصل‌های قبلی نیز آورده شد انتگرال فازی در واقع یک تعمیم الگوریتم دهنده‌ی میانگین وزنی می‌باشد که علاوه بر ویژگی‌هایی که روش میانگین وزنی ارائه می‌دهد می‌تواند اندازه‌گیری‌های غیرافزایشی را نیز مدل کند. لذا با تغییر تابع $g(\cdot)$ می‌توان باعث شد که انتگرال فازی چوکت تعابیر مختلفی از داده‌های ورودی خود ارائه دهد. از بین الگوریتم‌ها فقط الگوریتم \مق{Const-One} دارای تعبیر صریح ریاضی می‌باشد که در
\ref{eq:choquet-using-const-one}
آمده است، بقیه‌ی الگوریتم‌ها دارای تعابیر صریح نیستند و فقط می‌توانیم بر اساسی نمایشی که در شکل \ref{fig:fci-g-functions} آمده است شهودی از نحوه‌ی تغییر رفتار انتگرال فازی به ازای هریک از الگوریتم‌ها ارائه داد.
\begin{equation}
g = \text{Const-One}(\cdot) \equiv \begin{cases}
g(X) &= 1\\
g(\emptyset) &= 0\\
g_{A \subseteq X}(A) &= 1
\end{cases} \Rightarrow \mathcal{C}_g(f) \equiv \max\{f(x_{\pi^c_{(1)}}), \cdots, f(x_{\pi^c_{(n)}})\}\label{eq:choquet-using-const-one}
\end{equation}

 برای نمایش شهودی نحوه‌ی تغییر رفتار انتگرال فازی چوکت در شکل
\ref{fig:fci}
سه منبع اطلاعاتی با مقادیر $y = 1$ و $y = 2$ و $y = 3$ در نظر گرفته شده است و مقدار ارزش هرکدام از این‌ها به ترتیب $g = \begin{bmatrix}0.1 & 0.4& 0.3\end{bmatrix}^T$ در نظر گرفته شده است. سپس انتگرال فازی چوکت را با در نظر گرفتن تابع همانی به عنوان تابع $f(\cdot)$ بروی این ۳ منبع اطلاعاتی اعمال کردیم و همانطور که می‌بینیم مقداری که انتگرال فازی چوکت به ازای $g = \text{Const-One}(\cdot)$ تولید می‌کند برابر با حداکثر مقدار منابع اطلاعاتی دریافتی می‌باشد. در حالت کلی هرچقدر میانگین تابع $g_{A \subseteq X}(A)$ به سمت مقدار ۱ متمایل باشد خروجی انتگرال فازی چوکت به سمت بیشینه مقدار منابع اطلاعاتی پیش‌قدر می‌شود و در صورتی که این میانگین به سمت صفر متمایل باشد خروجی به کمینه مقدار پیش‌قدر می‌شود.

\fig{fci}{نمایش رفتار انتگرال فازی بروی منابع اطلاعاتی $y = 1$ و $y = 2$ و $y = 3$ به ازای توابع $g(\cdot)$های مختلف.}