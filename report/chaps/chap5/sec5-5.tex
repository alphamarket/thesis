\قسمت{بررسی تاثیر استفاده از انتگرال فازی در بهبود دانش جمعی}
همانطور که در بخش‌های قبلی دیدیم، استفاده از توابع $g(\cdot)$ مختلف در انتگرال فازی نتایج روش پیشنهادی موثر واقع شده‌اند و در نهایت با استفاده از تابع $\text{Const-One}$ که باعث می‌شود انتگرال فازی به یک عملگر بیشینه‌گیری تبدیل شود و بطور حریصانه در هر مرحله‌ی به اشتراک‌گذاری دانش، فقط دانش عاملی را به عنوان دانش جمعی خروجی الگوریتم در نظر بگیرد؛ در این بخش می‌خواهیم این موضوع را بررسی کنیم که تاثیر انتگرال فازی در بهبود دانش جمعی چگونه می‌باشد؟ به عنوان مثال آیا همیشه استفاده از انتگرال فازی در بهبود دانش جمعی موثر واقع است؟

در پاسخ به این سوالات، ما روش SEP را با این‌بار با استفاده از انتگرال فازی مورد آزمون قرار دادیم که همانطور که در شکل‌ \ref{fig:sep_maze_fci} مشاهده می‌کنیم وقتی در روش SEP بجای میانگین وزنی (روش پیشنهادی SEP) از انتگرال فازی با توابع معرفی شده در الگوریتم‌های \ref{alg:g:max} تا \ref{alg:g:k-mean} استفاده کنیم برعکس نتایج بدست آمده خروجی انتگرال فازی کیفیت و سرعت یادگیری را بهبود نمی‌بخشد ولی در صورتی که از الگوریتم \ref{alg:g:const-one} استفاده کنیم بهترین نتیجه‌ی ممکن را می‌گیریم. ولی با این حال باز بهترین نتیجه‌ی روش SEP در زمانی که بطور حریصانه دانش جمعی را بدست بیاوریم از بهترین نتیجه‌ی روش پیشنهادی کمتر است که نشان می‌دهد که معیار ارائه شده در روش پیشنهادی بهتر از معیار پیشنهادی روش SEP است.

\begin{figure}
\centering
\caption{تاثیر استفاده از انتگرال فازی در روش SEP بروی کیفیت و سرعت یادگیری در محیط پلکان مارپیچ}\label{fig:sep_maze_fci}
\begin{tabular}{*1c}
\subf{\includegraphics[width=.8\textwidth]{boltzmann/pref/sep/env/maze/fci-check.png}}
     {الف - استفاده از انتگرال فازی در روش SEP در محیط پلکان مارپیچ}
\\
\subf{\includegraphics[width=.8\textwidth]{boltzmann/pref/sep/env/prey/fci-check.png}}
{ب - استفاده از انتگرال فازی در روش SEP در محیط صید و صیاد}
\end{tabular}
\end{figure}