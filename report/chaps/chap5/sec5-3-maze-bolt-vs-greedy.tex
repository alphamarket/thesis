\زیرزیرقسمت{مقایسه‌ی بین نتایج حاصل از سیاست انتخاب عمل بولتزمن و $\varepsilon$-حریصانه}
در حالت کلی در محیط پلکان مارپیچ تابع بولتزمن نتایج یکنواتر و پایدارتری\زیرنویس{Stable} نسبت به تابع $\varepsilon$-حریصانه از خود نشان داد و در هر دوی این توابع روش پیشنهادی نتیجه‌ی بهتری نسبت به روش SEP ارائه داد. در این قسمت به مقایسه‌ی نتایج بدست آمده توسط هر دو روش در هر دو سیاست انتخاب عمل می‌پردازیم.

\پاراگراف{مقایسه در سرعت و کیفیت یادگیری:} مقایسه‌ی این قسمت را بطور خلاصه می‌توان در جدول
\ref{tab:pref_greedy_bolt_comp}
مشاهده کرد. که نسبت کیفیت نتیجه‌ی حاصل از تابع $\varepsilon$-حریصانه نسبت به تابع بولتزمن همگی بزرگتر از ۱ می‌باشد، که نشان می‌دهد که استفاده از تابع $\varepsilon$-حریصانه در کیفیت خروجی تاثیری منفی دارد.

\begin{table}
\centering
\caption{مقایسه در سرعت و کیفیت یادگیری نسبت کیفیت نتیجه‌ی حاصل از تابع $\varepsilon$-حریصانه نسبت به تابع بولتزمن}\label{tab:pref_greedy_bolt_comp}
\begin{latin}
\begin{tabular}{*4{c|}}
\multicolumn{2}{c}{} &\multicolumn{2}{c}{Boltzmann}
\\
\multicolumn{2}{c|}{}& SEP & REFMAT
\\\hline
\parbox[t]{2mm}{\multirow{2}{*}{\rotatebox[origin=c]{90}{$\varepsilon$-greedy}}} & SEP & 7.27 & 9.42
\\
& REFMAT & 5.20 & 6.79
\\\hline
\end{tabular}
\end{latin}
\end{table}

\پاراگراف{مقایسه در سرعت اجرا:}
در جدول
\ref{tab:time_greedy_bolt_comp}
نسبت میانگین سرعت اجرای روش‌ها آمده است، قطر اصلی این جدول همگی مقادیر بزرگتر از ۱ دارد که نشان می‌دهد هر روش در زمانی که از تابع $\varepsilon$-حریصانه استفاده می‌کند زمان بیشتری را تلف می‌کند(صرف جستجوی بی‌مورد محیط می‌کند) نسبت به زمانی که از تابع بولتزمن استفاده می‌کند. این مساله نشان می‌دهد که تابع بولتزمن سریع‌تر عامل را به سمت اهداف هدایت می‌کند -- که این نکته در قسمت «مقایسه‌ی سرعت و کیفیت یادگیری» نیز قابل استنتاج است.

\begin{table}
\centering
\caption{مقایسه در نسبت میانگین سرعت اجرای حاصل از استفاده تابع $\varepsilon$-حریصانه نسبت به تابع بولتزمن}\label{tab:time_greedy_bolt_comp}
\begin{latin}
\begin{tabular}{*5{c|}}
\multicolumn{2}{c}{} &\multicolumn{3}{c}{Boltzmann}
\\
\multicolumn{2}{c|}{}& SEP & REFMAT & IL
\\\hline
\parbox[t]{2mm}{\multirow{3}{*}{\rotatebox[origin=c]{90}{$\varepsilon$-greedy}}} & SEP & 1.64 & 2.05 & 10.23
\\
& REFMAT & 1.72 & 2.15 & 10.73
\\
& IL & 0.56 & 0.70 & 3.49
\\\hline
\end{tabular}
\end{latin}
\end{table}

\پاراگراف{مقایسه در میزان باروری:} همانطور که در جدول
\ref{tab:qmax_greedy_bolt_comp}
آمده است همه‌ی مقادیر نسبت‌ها بیشتر از ۱ می‌باشد که بدین معنی است که استفاده از تابع $\varepsilon$-حریصانه با این حال که کیفیت و سرعت یادگیری کمتری نسبت به تابع بولتزمن دارد و عامل‌ها در حالت کلی زمان زیادی صرف گشت و گذار در محیط می‌کند به نسبت باعث باروری بیشتر جدول $Q$ می‌شود.

\begin{table}
\centering
\caption{مقایسه در نسبت میزان باروری حاصل از استفاده تابع $\varepsilon$-حریصانه نسبت به تابع بولتزمن}\label{tab:qmax_greedy_bolt_comp}
\begin{latin}
\begin{tabular}{*5{c|}}
\multicolumn{2}{c}{} &\multicolumn{3}{c}{Boltzmann}
\\
\multicolumn{2}{c|}{}& SEP & REFMAT & IL
\\\hline
\parbox[t]{2mm}{\multirow{3}{*}{\rotatebox[origin=c]{90}{$\varepsilon$-greedy}}} & SEP & 1.08 & 1.25 & 1.23
\\
& REFMAT & 1.03 & 1.20 & 1.18
\\
& IL & 1.09 & 1.27 & 1.25
\\\hline
\end{tabular}
\end{latin}
\end{table}

\پاراگراف{مقایسه تاثیر تعداد عامل‌ها بر میزان کیفیت و سرعت یادگیری:} در جدول
\ref{tab:agents_greedy_bolt_comp}
نسبت شیب تاثیر تعداد عامل‌ها میزان کیفیت نتیجه‌ی حاصل از تابع $\varepsilon$-حریصانه نسبت به تابع بولتزمن آمده است؛ همانطور که مشاهده می‌شود در زمانی که از تابع $\varepsilon$-حریصانه استفاده می‌شود در روش پیشنهادی تاثیر تعداد عامل‌ها به مراتب بیشتر از زمانی است که از تابع بولتزمن استفاده می‌کنیم. این در حالی می‌باشد که در روش SEP اضافه کردن عامل‌ها به محیط تفاوت زیادی در دانش خروجی الگوریتم در هر دو تابع ایجاد نمی‌کند.

\begin{table}
\centering
\caption{مقایسه نسبت شیب تاثیر تعداد عامل‌ها میزان کیفیت نتیجه‌ی حاصل از تابع $\varepsilon$-حریصانه نسبت به تابع بولتزمن}\label{tab:agents_greedy_bolt_comp}
\begin{latin}
\begin{tabular}{*4{c|}}
\multicolumn{2}{c}{} &\multicolumn{2}{c}{Boltzmann}
\\
\multicolumn{2}{c|}{}& SEP & REFMAT
\\\hline
\parbox[t]{2mm}{\multirow{2}{*}{\rotatebox[origin=c]{90}{$\varepsilon$-greedy}}} & SEP & 0.59 & 0.09
\\
& REFMAT & 73.02 & 10.67
\\\hline
\end{tabular}
\end{latin}
\end{table}