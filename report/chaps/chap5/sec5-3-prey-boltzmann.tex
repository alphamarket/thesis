\زیرزیرقسمت{سیاست انتخاب عمل «بولتزمن»}

\fig[.8]{boltzmann/prey.pref.compare.png}{مقایسه در سرعت و کیفیت یادگیری در محیط صید و صیاد با تابع بولتزمن با تابع بولتزمن در محیط صید و صیاد}

\begin{table}
\centering
\caption{مقایسه در میزان درصد بهبود کیفیت یادگیری در محیط صید و صیاد با تابع بولتزمن}\label{tab:prey_pref_compare}
\begin{latin}
\begin{tabular}{|*8{l|}}
\hline
\multicolumn{3}{|c|}{}& \multicolumn{5}{c|}{REFMAT}
\\\hline
& IL & SEP & wsum & fci-mean & fci-max & fci-k-mean & fci-const-one
\\\hline
IL &0.0 & & & & & &
\\\hline
SEP &3.3 &0.0 & & & & &
\\\hline
wsum &3.1 &-0.2 &0.0 & & & &
\\\hline
fci-mean &16.7 &13.0 &13.2 &0.0 & & &
\\\hline
fci-max &17.3 &13.5 &13.7 &0.5 &0.0 & &
\\\hline
fci-k-mean &19.0 &15.2 &15.4 &2.0 &1.4 &0.0 &
\\\hline
fci-const-one &24.9 &20.9 &21.1 &7.0 &6.4 &4.9 &0.0
\\\hline
\end{tabular}
\end{latin}
\end{table}

\پاراگراف{مقایسه در سرعت و کیفیت یادگیری:} نتایج حاصل از اجرای الگوریتم‌ها در محیط صید و صیاد در شکل
\ref{fig:boltzmann/prey.pref.compare.png}
آمده است. در این شکل محور افقی تعداد تلاش‌های یادگیری عامل را نشان می‌دهد که در تلاش اول عامل بدون دانش اولیه شروع به تعامل با محیط می‌کند و در تلاش ۱۰۰۰ام عامل به اجرای خود پایان می‌دهد. محور عمودی نموار میانگین تجمعی تعداد قدم‌های عامل را نشان می‌دهد. اعداد کناری برچسب‌ها (گوشه بالا سمت راست) متوسط تعداد قدم‌ در آخرین تلاش عامل می‌باشد که انتظار می‌رود عامل آگاهی نسبی کاملی از محیط دارد را نشان می‌دهد که این عدد هرچقدر کمتر باشد نشان می‌دهد که عامل در طی رسیدن به هدف تعداد گام کمتری برداشته است و در نتیجه دانش و شناخت بهتری از محیط دارد.

همانطور که مشاهده می‌شود روش SEP دارای ۳٪ بهبود نسبت به IL می‌باشد در حالی که روش پشنهادی در زمانی که از انتگرال فازی استفاده می‌کند در بدترین حالت دارای ۱۷٪ بهبود و در بهترین حالات دارای ۲۵٪ بهبود می‌باشد که نسبت به روش SEP تقریبا ۹ الی ۱۶ برابر نتیجه را بهبود داده است. در صورتی که از میانگین وزنی بجای انتگرال فازی استفاده شود حدود ۳٪ بهبود نسبت به یادگیری IL مشاهده می‌شود (همانند SEP) که نشان ‌می‌دهد که استفاده از انتگرال فازی چقدر می‌تواند نسبت به روش‌های سنتی و معمولی چون میانگین وزنی موثر واقع شود. نتایج این قسمت را می‌توان در جدول
\ref{tab:prey_pref_compare}
خلاصه کرد.

\پاراگراف{مقایسه در پیچیدگی زمانی:} در این قسمت به مقایسه‌‌ی پیچیدگی زمانی روش پیشنهادی با روش SEP مورد بررسی قرار می‌گیرد، برای محاسبه‌ی پیچیدگی زمانی به روش ریاضی کار بسیار دشوار و پرخطایی می‌باشد؛ در اینجا ما بجای محاسبه‌ی پیچیدگی زمانی ریاضی دو الگوریتم از مدت زمانی که طول می‌کشد برنامه در سیستم اجرا و خاتمه یابد استفاده می‌کنیم. در شکل
\ref{fig:boltzmann/prey.time.compare.png}
میانگین زمانی ۲۰ اجرای مستقل برحسب میلی‌ثانبه به ازای هریک از تعداد تلاش‌ها آورده شده است. همان‌طور که در این شکل مشاهده‌ی می‌شود الگوریتم IL دارای حداکثر سرعت اجرا می‌باشد زیرا که هیچ سربار محاسباتی یادگیری مشترک را ندارد؛ هدف یادگیری اشتراکی این است که می‌خواهد در ازای یک سری سربار محاسباتی کیفیت و سرعت «یادگیری» عامل‌ها را افزایش دهد. با در نظر داشتن این موضوع همانطور که قبلا دیدیم روش پیشنهادی سرعت و کیفیت یادگیری را بیشتر از روش SEP افزایش می‌دهد و در اینجا نیز می‌بینیم که دارای پیچیدگی زمانی کمتری نسبت به روش SEP می‌باشد که نشان از بهینه‌گی روش پیشنهادی نسبت به روش SEP می‌دهد.

\fig[0.8]{boltzmann/prey.time.compare.png}{مقایسه در پیچیدگی زمانی روش‌ها به ازای تعداد تلاش‌های متفاوت برحسب میلی‌ثانیه با تابع بولتزمن در محیط صید و صیاد}

\پاراگراف{مقایسه در میزان باروری:} همانطور که در شکل
\ref{fig:boltzmann/prey.qtable.max.compare.png}
مشاهده می‌کنیم روش معرفی شده در زمانی که به صورت تصادفی اقدام به انتخاب عمل می‌کند بیشتر از زمانی که IL و SEP با بصورت تصادفی اقدام به انتخاب عمل می‌کند جدول $Q$ را بارور می‌کند که از قدرت روش ارائه شده خبر می‌دهد. همچنین در مورد روش SEP می‌بینیم که در زمانی که بصورت تصادفی اقدام به عمل می‌کند باروری کمتری نسبت به روش پیشنهادی و IL دارد؛ یعنی میزان باروری روش SEP وابستگی زیادی به سیاست انتخاب عمل دارد و در صورت نداشتن سیاست انتخاب عمل خاصی بشدت عملکردش کاسته می‌شود ولی در روش پیشنهادی میزان این وابستگی از شدت کمتری برخوردار است که از دیگر امتیازات مثبت روش پیشنهادی می‌باشد -- همانند نتایج حاصله در محیط پلکان مارپیچ.

\fig[.8]{boltzmann/prey.qtable.max.compare.png}{نمودار باروری الگوریتم‌ها مختلف با تابع بولتزمن در محیط صید و صیاد}

\پاراگراف{مقایسه تاثیر تعداد عامل‌ها میزان کیفیت و سرعت یادگیری:}

همان‌طور که در شکل
\ref{fig:boltzmann/prey.multi-agent.pref.compare.png}
آمده است، روش SEP در زمانی ۲۰ عامل در حال یادگیری و اشتراک گذاری دانش‌های خود هستند نسبت به زمانی که فقط ۲ عامل در حال تعامل مشارکتی با محیط هستند فقط ۲٪ در خروجی الگوریتم تاثیر مثبت داشته است. این در حالی است که در همین شرایط میزان بهبود نتیجه‌ی روش پیشنهادی ۳۸٪ می‌باشد. که نشان می‌دهد روش SEP نسبت به افزایش تعداد عامل‌ها رفتاری تقریبا خنثی از خود نشان می‌دهد درحالی که روش پیشنهادی در ازای افزایش تعداد عامل‌ها به دلیل اینکه دانش جمعی نیز افزایش می‌یابد کیفیت خروجی آن نیز بهتر می‌شود.

\fig[.8]{boltzmann/prey.multi-agent.pref.compare.png}{مقایسه تاثیر تعداد عامل‌ها میزان کیفیت و سرعت یادگیری با تابع بولتزمن در محیط صید و صیاد}

\پاراگراف{نتیجه‌گیری:} نتیجه‌ای که از مقایسه‌ی روش پیشنهادی در هر چهار مقایسه‌ی بالا می‌توان گرفت این است که روش پیشنهادی بهبود چشم‌گیری به روش SEP در محیط صید و صیاد و سیاست انتخاب عمل بولتزمن داده است.