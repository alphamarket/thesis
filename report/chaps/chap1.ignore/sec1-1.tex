\قسمت{پیش‌گفتار}
راهبری\زیرنویس{Navigation} و \جام در دنیای رباتیک مدرن امروز به دلیل ایفای نقش بسیار مهم در موفقیت ربات‌های خودمختار، به عنوان یکی از چالش‌های مهم و هیجان انگیز در جوامع دانشگاهی و صنعتی شناخته شده است. امروزه ربات‌های چندپره\زیرنویس{Multirotor
}\مرجع{wiki:moltirotor} کاربردهای فراوانی در صنعت، تفریحات عموم و قدرت نظامی ایفا می‌کند، که تنوعی به اندازه‌ی پهپادهایی به قطر کمتر از ۳ سانتی‌متر\مرجع{site:smallest_quad} تا غول‌هایی که می‌توانند انسان را از زمین به پرواز درآورند\مرجع{site:human_quad}، دارند. علت رشد نسبتا سریع این نوع از پهپادها نسبت به پهپادهای هم‌تراز خود مانورپذیری ساده‌تر و نسبتا کم‌هزینه بودن ساخت این خانوده از پهپاد‌ها می‌باشد. امکان برخواست\زیرنویس{Land} و فرود‌آمدن درجا و همچنین قابلیت معلق ماندن در هوا باعث شده این خانواده از پهپادها به ابزاری مناسب برای عملیات‌های نظارتی\زیرنویس{Surveillance}، جستجو و نجات باشند.\بند
همانقدر که طرفداران این خانواده‌ بیشتر می‌شود انتظارات بیشتری نیز از آن‌ها می‌رود، امکانات و انتظاراتی که شاید از کمتر رباتی می‌رود، این پهپادها باید در کنار دارا بودن مانورپذیری و حرکات نمایشی‌ای که بتوانند کاربران عادی خود را سرگرم نگه دارند، باید دارای سیستم‌های تعبیه‌شده برای حفظ امنیت ربات و اطرافیان آن باشد؛ لذا خطر سقود در هر وسیله‌ی هوایی یک خطر جدی می‌باشد که علاوه‌بر اعمال خسارت به خود ربات خطر جانی برای افراد حاظر در محیط پیرامون وسیله که تحت تاثیر سقوط آن قرار می‌گیرند دارد. یکی از دلایل سقوط وسایل هوایی(و بخصوص در پهپاد‌ها) بخورد با موانع احتمالی موجود در مسیر است؛ به همین سبب مساله‌ی \جام به یکی از چالش برانگیزترین و هیجان انگیزترین مساله‌ی دنیای امروز رباتیک بدل شده است، بطوری که فقط در همین ۱۰ سال اخیر صدها مقاله‌ی پژوهشی در این راستا به چاپ رسیده است. که این پژوهش گامی کوچک در راستای ارائه و بهبود روشی برای تامین امنیت پهپادهای خانواده چندپره‌ می‌باشد.