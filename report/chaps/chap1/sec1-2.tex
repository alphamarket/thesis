\قسمت{تعریف مساله}
در این پژوهش هدف، طراحی و توسعه یک سیستم تشخیص و \جام برای ممانعت برخورد با موانع موجود در مسیر حرکت پهپادها چندپره(به صورت خاص) می‌باشد. محیطی که ربات در آن به پرواز درمی‌آید، برای ربات ناشناخته بوده و ربات هیچ دانشی قبلی نسبت به محیط و موانع موجود در آن ندارد. سیستم تشخیص مانع\زیرنویس{Obstacle Detection System} باید به گونه‌ای طراحی شود که موانع موجود در مسیر را از فاصله‌ی معقولی تشخیص کند و موقعیت نسبی آن با پهپاد را در واحد-فاکتورهای\زیرنویس{Factor-units} مشخص کند، سپس با دادن اطلاعات بدست آمده از موانع موجود در محیط به سیستم \جام\زیرنویس{Obstacle Avoidance System} یا طرح‌ریز ماموریت\زیرنویس{Mission Planner} تحویل داده و این سیستم وظیفه‌‌ی تصمیم‌گیری مسیر و جهت‌دهی پهپاد را بر اساس اطلاعات محیطی دریافتی به عهده خواهد داشت.\بند
از آنجا که راهبری در حالت کلی دربرگیرنده‌ی مباحث و شاخه‌های گسترده‌ای می‌باشد لذا به‌جهت قابل انجام بودن این پژوهش در مهلت مقرر از پیاده‌سازی انواع روش‌های مکان‌یابی و راهبری‌های مبتنی بر آن خودداری کرده و صرفا بروی مساله‌ی \جام که خود مساله‌ای وسیع و چالش برانگیز می‌باشد تمرکز می‌کنیم.