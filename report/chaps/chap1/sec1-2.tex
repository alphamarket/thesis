\قسمت{اهداف و نوآوری‌های پایان‌نامه}
بطور خلاصه هدف این پژوهش ارائه‌ی معیاری نرم (ساده در عین موثر بودن) بجهت محاسبه‌ی خبرگی عامل‌ها با در نظر گرفتن خاصبت غیرافزایشی \مرجع{torra2014non} خبرگی و دانش عامل‌ها می‌باشد. در طی دست‌یابی به هدف تعیین شده در این پژوهش ابتدا چهارچوبی به نام «فرضیه‌ی خبرگی» معرفی شد که توانایی استخراج هر معیاری برای محاسبه‌ی خبرگی از طریق این فرضیه مسیر باشد؛ سپس با استفاده از فرضیه‌ی خبرگی معرفی شده و با در نظر داشتن هدف تعیین شده برای این پژوهش در مورد ارائه‌ی معیار نرم، معیار خبرگی جدیدی به نام «میزان ارجاع» تعریف شد.

در طی این پژوهش الگوریتمی برای ترکیب دانش‌ عامل‌ها با در نظر داشتن میزان خبرگی معرفی شده هر عامل ارائه شد. در این الگوریتم از انتگرال فازی به عنوان عملگر ترکیب کننده دانش عامل‌ها استفاده کردیم و طبق آزمایشات نشان دادیم که انتگرال فازی چوکت می‌تواند نتایج بسیار بهتری نسبت به روش‌های سنتی چون میانگیری وزن‌دار می‌تواند تولید کند زیرا انتگرال فازی چوکت می‌تواند خاصیت غیرافزایشی مساله را برخلاف میانگین وزنی مدل کند. در نهایت علاوه بر اینکه نشان دادیم انتگرال فازی می‌تواند برای ترکیب دانش عامل‌ها موثر واقع شود؛ بدون در نظر گرفتن تاثیر انتگرال فازی، معیار معرفی شده که بر اساس فرضیه‌ی خبرگی تعریف شده است می‌تواند خبرگی عامل‌ها را بدرستی تعیین کنند و در صورتی که در هنگام ترکیب دانش عامل‌ها فقط دانش عاملی را در نظر بگیریم که با توجه به معیار خبرگی معرفی شده حداکثر خبرگی را دارد، در چهاچوب روش ارائه‌ی شده و در مقایسه با مدرن‌ترین روش ارائه شده به نام «کوتاه‌ترین مسیر تجربه شده»\مرجع{mohammad2015speedup} حداکثر بهبود را خواهد داد.