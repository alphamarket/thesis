\قسمت{اهداف و نوآوری‌های پایان‌نامه}
بطور خلاصه هدف این پژوهش ارائه‌ی معیاری نرم (ساده در عین موثر بودن) بجهت محاسبه‌ی خبرگی عامل‌ها با در نظر گرفتن خاصبت غیرافزایشی \مرجع{torra2014non} خبرگی و دانش عامل‌ها می‌باشد -- خاصیت غیرافزایشی دانش (خبرگی) عامل‌ها می‌گوید که ارزش دانش (خبرگی) چند عامل باهم لزوما برابر با مجموع ارزش دانش (خبرگی) تک‌تک آن‌ها نمی‌باشد. در طی دست‌یابی به هدف تعیین شده در این پژوهش ابتدا چهارچوبی به نام «فرضیه‌ی خبرگی» معرفی شد که توانایی استخراج معیارهای زیادی برای محاسبه‌ی خبرگی از طریق این فرضیه میسر باشد؛ سپس با استفاده از فرضیه‌ی خبرگی معرفی شده و با در نظر داشتن هدف تعیین شده برای این پژوهش در مورد ارائه‌ی معیار نرم، معیار خبرگی جدیدی به نام «میزان ارجاع» تعریف شد.

در طی این پژوهش الگوریتمی برای ترکیب دانش‌ عامل‌ها با در نظر داشتن میزان خبرگی معرفی شده هر عامل ارائه شد. در این الگوریتم از انتگرال فازی به عنوان عملگر ترکیب کننده دانش عامل‌ها استفاده کردیم و طبق آزمایش‌ها نشان دادیم که انتگرال فازی چوکت می‌تواند نتایج بهتری نسبت به روش‌های سنتی چون میانگیری وزن‌دار تولید کند زیرا انتگرال فازی چوکت می‌تواند خاصیت غیرافزایشی مساله را برخلاف میانگین وزنی مدل کند. دستآورد‌های این پژوهش به صورت خلاصه  به شرح زیر می‌باشد:
\begin{itemize}\setlength\itemsep{-0.5em}
\فقره معرفی چهارچوبی به‌نام «فرضیه‌ی خبرگی» برای تعریف معیارهای خبرگی جدید.
\فقره تعریف معیار خبرگی جدید به‌نام «میزان ارجاع» در چهارچوب معرفی شده توسط «فرضیه‌ی خبرگی».
\فقره استفاده از «انتگرال فازی چوکت» در ترکیب دانش‌های عامل‌ها با توجه به میزان خبرگی عامل‌ها.
\فقره تعریف معیاری جدید به‌نام «میزان باروری» به جهت سنجش سرعت یادگیری الگوریتم‌ها.
\فقره بررسی تاثیر سیاست‌های انتخاب عمل $\varepsilon$-حریصانه\زیرنویس{\مق{$\varepsilon$-greedy}} در یادگیری مشارکتی -- پژوهش‌های قبلی این موضوع را مورد بررسی قرار نداده‌اند.
\فقره اثبات صحت فرضیه و معیار خبرگی معرفی شده در این پژوهش با توجه نتایج آزمایش‌ها.
\end{itemize}