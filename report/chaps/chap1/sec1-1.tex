\قسمت{یادگیری مشارکتی در سیستم‌های چند عامله}
انسان در طول حیات خود یادگیری زیادی انجام می‌دهد اما اگر قرار بود هر انسان از  صفر شروع به جمع‌آوری اطلاعات کند و از عامل‌های دیگر یادگیری نداشته باشد بدون شک هنوز انسان‌ها همانند انسان‌های اولیه زندگی می‌کردند. این رشدی که امروزه در زندگی انسانی دیده می‌شود مدیون انتقال اطلاعات و دانش بین عامل‌های انسانی است. بر همین اساس در\مرجع{panait2005cooperative} یادگیری مشارکتی را سیستمی می‌داند که عامل‌ها در ان با همکاری یکدیگر به یادگیری یک وظیفه مشترک می‌پردازند. می‌توان آثار مثبت یادگیری مشارکتی در سیستم‌های چند عامله را چنین برشمرد.

\begin{itemize}%\setlength\itemsep{-.5em}
\فقره افزایش سرعت و دقت یادگیری.
\فقره آزادسازی عامل از بهینگی محلی.
\فقره کمک به تنظیم پارامترهای محلی عامل‌ها.
\end{itemize}

نکته مهمی که باید در نظر داشت عامل‌های موجود در یادگیری مشارکتی است. در بعضی از روش‌ها عامل‌های یک سیستم از توانایی‌های متفاوتی برخوردار هستند. به مجموعه عامل‌های موجود در این محیط‌ها تیم گفته می‌شود\مرجع{mohammad2015speedup, thesis:pakizeh2013multi} و در سیستم‌های دیگر مشابه پژوهش پیش رو عامل‌ها با توانایی‌های یکسان در نظر گرفته می‌شوند. گذشته از تفاوت بین عامل‌ها، یادگیری مشارکتی نیز همانند دیگر شاخه‌های هوش مصنوعی از چالش‌های فراوانی برخوردار است.فارغ از چالش‌های مشترکی که بیت یادگیری مشارکتی و روش یادگیری استفاده در ان وجود دارد چالش‌های جدیدی در یادگیری مشارکتی وجود دارد که می‌توان با سؤالاتی تعدادی از این چالش‌ها را نشان داد.

\begin{itemize}%\setlength\itemsep{-.5em}
\فقره چه اطلاعاتی باید بین عامل‌ها ردوبدل شود؟
\فقره چه زمان باید اطلاعات منتقل شود؟
\فقره ترکیب داده‌های دریافتی باید به چه صورت باشد؟
\فقره بر اساس چه معیاری می‌توان عامل‌ها را مقایسه کرد؟
\end{itemize}

در رابطه با هریک از این چالش‌ها کارهای فراوانی چون پنددهی، تقلید و خبرگی انجام شدم است که تعدادی از آن‌ها در فصل دوم آورده شده است. تمام این روش‌‌های تشریح شده در فصل دوم از یادگیری تقویتی به عنوان الگوریتم اصلی عامل برای یادگیری نحوه‌ی تعامل با محیط بهره برده‌اند و سعی در ارائه‌ی روش جهت ترکیب مناسب داده‌ها نموده‌اند. در ترکیب داده‌ها همیشه نیازی به معیاری جهت سنجش میزان درستی داده‌ها (دانش هر عامل) وجود دارد اما از آنجایی که عامل‌ها محیط را نمی‌شناسند در نتیجه دستیابی به معیاری صحیح برای این منظور کار دشواری می‌باشد.

در روش‌هایی چون خبرگی سعی شده تا معیارهایی جهت سنجش داده‌ها ارائه شود اما در کار پنددهی عامل‌ها زمانی که از داده‌ی خود مطمئن باشند به عامل دیگر بازخورد می‌دهند و این بازخورد در درک عامل از محیط موثر خواهد بود. معمولا معیارهای معرفی شده در پژوهش‌های صورت گرفته بروی خبرگی یا با دیدگاه خیلی جزئی به بررسی خبرگی عامل‌ها می‌پردازند یا بصورت خیلی کلی؛ در حالت خلاصه کار اصلی که در این پژوهش انجام دادیم ارائه‌ی  چهارچوب کلی برای تولید انواع معیارها و سپس ارائه‌ی معیاری که خیلی کلی یا جزئی نباشد و در عین حال بتواند عملکرد بهتری نسبت به روش‌های قبلی ارائه دهد.