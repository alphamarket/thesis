\قسمت{یادگیری مشارکتی در سیستم‌های چند عامله}
انسان در طول حیات خود یادگیری زیادی انجام می‌دهد اما اگر قرار بود هر انسان از  صفر شروع به جمع‌آوری اطلاعات کند و از عامل‌های دیگر یادگیری نداشته باشد بدون شک هنوز انسان‌ها همانند انسان‌های اولیه زندگی می‌کردند. این رشدی که امروزه در زندگی انسانی دیده می‌شود مدیون انتقال اطلاعات و دانش بین عامل‌های انسانی است. بر همین اساس در فلان یادگیری مشارکتی را سیستمی می‌داند که عامل‌ها در ان با همکاری یکدیگر به یادگیری یک وظیفه مشترک می‌پردازند. می‌توان آثار مثبت یادگیری مشارکتی در سیستم‌های چند عامله را چنین برشمرد.

\begin{itemize}\setlength\itemsep{-.5em}
\فقره افزایش سرعت و دقت یادگیری.
\فقره آزادسازی عامل از بهینگی محلی.
\فقره کمک به تنظیم پارامترهای محلی عامل‌ها.
\end{itemize}

نکته مهمی که باید در نظر داشت عامل‌های موجود در یادگیری مشارکتی است. در بعضی از روش‌ها عامل‌های یک سیستم از توانایی‌های متفاوتی برخوردار هستند. به عامل‌های موجود در این محیط‌ها تیم گفته می‌شود و در سیستم‌های دیگر مشابه پژوهش پیش رو عامل‌ها با توانایی‌های یکسان در نظر گرفته می‌شوند. گذشته از تفاوت بین عامل‌ها، یادگیری مشارکتی نیز همانند دیگر شاخه‌های هوش مصنوعی از چالش‌های فراوانی برخوردار است.فارغ از چالش‌های مشترکی که بیت یادگیری مشارکتی و روش یادگیری استفاده در ان وجود دارد چالش‌های جدیدی در یادگیری مشارکتی وجود دارد که می‌توان با سؤالاتی تعدادی از این چالش‌ها را نشان داد.

\begin{itemize}\setlength\itemsep{-.5em}
\فقره چه اطلاعاتی باید بین عامل‌ها ردوبدل شود.
\فقره چه زمان باید اطلاعات منتقل شود.
\فقره ترکیب داده‌های دریافتی باید به چه صورت باشد.
\فقره بر اساس چه معیاری می‌توان عامل‌ها را مقایسه کرد.
\end{itemize}

کارهای فراوانی در مورد هر یک از این چالش‌ها انجام‌شده است که تعدادی از آن‌ها در فصل بعد خواهد آمد. در این کارها سعی در پژوهش انجام‌شده و در کارهایی که در راستای شناخت این زمینه در ادامه خواهد آمد از یادگیری تقویتی استفاده‌شده است.یادگیری تقویتی نیز دارای چالش‌های فراوانی است ک در فصل آینده موردبررسی قرار می‌گیرد.