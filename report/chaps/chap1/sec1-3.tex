\قسمت{چالش‌های موجودر در مساله}
برای دست‌یابی اهداف نوشته شده در این پژوهش چالش‌های زیادی وجود دارد، اولین و بزرگترین چالش ساخت پهپاد می‌باشد. ساخت پهپادی با امکانی که بتوان توسط رایانه اطلاعات آن را که شامل اطلاعات حسگرهای سونار، دوربین‌های استریو و اطلاعات سیستم موقعیت‌یاب جهانی(\مق{GPS}\زیرنویس{Global Positioning System}) را دریافت و سپس بعد از پردازش داده‌های دریافتی دستورهای کنترلی را به پهپاد ارسال کرد. با توجه به محدودهای برخی یارانه‌ها باس‌داده\زیرنویس{Databus} در حجم و تعداد دریافت هم‌زمان دو تصویر استریو به همراه دیگر اطلاعات چالشی بزرگ در راه‌اندازی پهپاد مورد استفاده در این پژوهش است.\بند
 دومین چالشی که در طی پیاده‌سازی این پژوهش باید بر آن غالب می‌شدم، چالش تشخیص مانع با استفاده از تصاویر استریو و حسگرهای سونار بود، باید عمق‌سنجی از تصاویر استریو و ترکیب این اطلاعات با اطلاعات بدست آمده توسط حسگرها به تشخیص موانع و موقعیت نسبی آن‌ها پرداخته و سپس با ارائه‌ی اطلاعات حس شده از محیط به سیستم \جام تحویل داده می‌شد و این سیستم طبق وظیفه‌ی تعریف شده برای آن، باید حرکت بعدی متناسب با موقیت موانع موجود در مسیر را معین کند؛ که همه‌ی این موارد باید به صورت بلادرنگ\زیرنویس{Real-time} صورت میگرفت بنابراین بهینه‌گی روش و کدهای نوشته از درجه اهمیت بسزایی برخوردار هستند.\بند