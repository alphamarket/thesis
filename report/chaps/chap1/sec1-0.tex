رایانه‌ها از قدرت بالایی در محاسبات برخوردار هستند تا جایی که محاسباتی که عامل انسانی ممکن است در چند سال انجام دهد را می‌توانند در کسری از ثانیه انجام دهند. در سال‌هایی که از عمر کامپیوتر گذشته هم پیشرفت‌های فراوانی در این زمینه صورت گرفته است که بر سرعت اجرای محاسبات کامپیوتر افزوده است. اما باوجود تمام این پیشرفت‌ها هنوز هم مسائل زیادی هستند که کامپیوتر نمی‌تواند در زمان قابل‌قبول آن‌ها را حل کند. یک روش که در راستای افزایش سرعت حل مسائل و با بهره‌گیری از زندگی جمعی عامل‌های انسانی پیشنهاد شد تقسیم مسئله به چندین زیر مسئله و حل هر یک توسط یک رایانه بوده است این موضوع شاخه‌ای تحت عنوان سیستم‌های توزیع‌شده را به وجود آورده است.

از طرف دیگر یکی از مسائل مهم در دنیای رایانه‌ها که ازنظر محاسباتی به زمان زیاد نیاز دارد موضوعات هوش مصنوعی هستند که در ترکیب با سیستم‌های توزیع‌شده به هوش مصنوعی توزیع‌شده بدل شده‌اند. هوش مصنوعی توزیع‌شده نیز در قالب حل مسائل هوش مصنوعی و یادگیری فعالیت می‌کند. در حل مسائل توزیع‌شده یک مسئله که قابلیت اجرای موازی داشته باشد را به بخش‌هایی تقسیم کرده و هر بخش در رایانه‌ای حل‌شده و نهایتاً نتایج ترکیب می‌شوند این ترکیب می‌تواند یک یا چند مرتبه انجام شود.

دسته دیگر از مسائل هوش مصنوعی توزیع مسائل یادگیری هستند که با عنوان سیستم‌های چندعاملی شناخته می‌شوند این سیستم‌های در راستای یادگیری از چند عامل بهره می‌برند. بر اساس روابطی که می‌توان بین این عامل‌ها تعریف کرد سیستم‌های چندعاملی می‌توانند رقابتی و یا مشارکتی باشند در سیستم‌های چند عامله رقابتی عامل‌ها به دنبال افزایش سود شخصی خود هستند که در زمان‌های زیادی به قیمت کاهش سود دیگر عامل‌ها خواهد بود؛ در سیستم‌های چندعاملی مشارکتی که موضوع پژوهش پیش رو نیز هست عامل‌ها به دنبال افزایش سود گروهی هستند. در شکل
\ref{fig:research_place}
جایگاه یادگیری مشارکتی در سیستم‌های چند عامله آورده شده است.

\begin{figure}
\centering
\begin{tikzpicture}
\begin{scope}[node distance=2cm, on grid]
       %\draw[help lines] (-6,-9) grid (6,1);
\node (ai) [block] {هوش‌مصنوعی};
\node (dist_sys) [block, right of=ai, node distance=4cm] {\rl{سیستم‌های توزیع شده}};
\node (dist_ai) [block, below of=ai] at ($(ai)!0.5!(dist_sys)$) {\rl{هوش‌مصنوعی توزیع شده}};
\node (multi_agent_sys) [block, below left of=dist_ai, node distance=2.8cm] {\rl{سیستم‌های چندعامله}};
\node (dist_prob_solving) [block, right of=multi_agent_sys, node distance=4cm] {\rl{حل مساله‌ی توزیع شده}};
\node (multi_agent_sys_compet) [block, below left of=multi_agent_sys, node distance=2.9cm, xshift=-.4cm] {\rl{سیستم‌های چندعامله رقابتی}};
\node (multi_agent_sys_coop) [block, right of=multi_agent_sys_compet, node distance=5cm] {\rl{سیستم‌های چندعامله مشارکتی}};
\node (team_learning) [block, below left of=multi_agent_sys_coop, node distance=2.9cm, xshift=-.5cm] {\rl{یادگیری تیمی}};
\node (co_learning) [block, right of=team_learning, node distance=5cm] {\rl{یادگیری همزمان}};

\path[every edge/.append style={|-|}, -] 
	(dist_ai) 
		edge (ai)
		edge (dist_sys);
\path[every edge/.append style={|-|}, -] 
	(dist_ai) 
		edge (multi_agent_sys)
		edge (dist_prob_solving);
\path[every edge/.append style={|-|}, -] 
	(multi_agent_sys) 
		edge (multi_agent_sys_compet)
		edge (multi_agent_sys_coop);
\path[every edge/.append style={|-|}, -] 
	(multi_agent_sys_coop) 
		edge (team_learning)
		edge (co_learning);
\end{scope}
\end{tikzpicture}
\caption{جایگاه پژوهش انجام شده\مرجع{thesis:pakizeh2013multi, mohammad2015speedup}}\label{fig:research_place}
\end{figure}

سیستم‌های چند عامله مشارکتی در دو دیدگاه موردبررسی قرار می‌گیرند در دیدگاه اول عامل‌ها در یک محیط قرارگرفته و سعی در یادگیری محیط و مهم‌تر از آن سعی دررسیدن به یک هماهنگی دارند و این هماهنگی در این راستا است که عامل‌ها بتوانند در محیط با همکاری هم به اهداف مشخص‌شده برسند. در دیدگاه دوم عامل‌ها قرار نیست با همکاری کار کنند و فقط سعی دارند با همکاری یکدیگر به یادگیری برسند این عامل‌ها در محیط‌های جداگانه و مشابهی قرار می‌گیرند و در طول فرایند یادگیری با هم ارتباط دارند. در این ارتباط داده‌های به دست آمده را به یکدیگر منتقل می‌نمایند تا زمانی که عامل‌ها تمام محیط را به‌خوبی شناسایی قرار دهند.