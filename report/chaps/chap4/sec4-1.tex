\قسمت{مقدمه}
در این فصل جزییات روش پیشنهادی به طور مفصل معرفی خواهد شد، روش ارائه شده در حالت کلی از دو قسمت تشکیل شده است؛ اولین و مهم‌ترین قسمت ارائه یک معیار خبرگی جدید به نام معیار خبرگی «ارجاع» که برای هر عامل در هر چرخه یادگیری محاسبه می‌شود و در یک «ماتریس ارجاع» نگه‌داری می‌شود. دومین قسمت مربوط به ترکیب دانش‌های عامل‌ها هستند که با استفاده از معیار «ارجاع» با استفاده از یک مدل انتگرال فازی، صورت می‌گیرد. همانطور که در فصل بعدی نیز نشان داده خواهده شد استفاده از مدل انتگرال فازی به دلیل خواصی مهمی که این مدل دارد باعث می‌شود سرعت و کیفیت یادگیری به طرز چشم‌گیری افزایش یابد. در این فصل ابتدا به معرفی معیار «ارجاع» و دلیل استفاده از این معیار می‌پردازیم سپس یادگیری مشارکتی چندعامله با استفاده از ماتریس ارجاع و انتگرال فازی معرفی خواهد شد و در نهایت نشان داده خواهد شد که چرا استفاده از انتگرال فازی نتایج بهتری را نسبت به مدل های سنتی چون مدل مجموع وزنی\زیرنویس{\مق{Weighted Sum}} را ارائه می‌دهد.