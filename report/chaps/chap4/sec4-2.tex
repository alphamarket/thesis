\قسمت{معیار خبرگی - ماتریس ارجاع و خاطره}
در دنیای واقعی «خبرگی» تعاریف متعددی به خود گرفته است، در روانشناسی خبرگی به معنی عملکرد برتر عامل تلقی می‌شود. در جامعه شناسی خبره به فردی گفتی برچسب خبرگی توسط یک گروهی به فرد زده شده است و آن گروه به توانایی که آن فرد در اختیار دارد علاقه‌مند\زیرنویس{Interested} است. در فلسفه خبره به فردی گفته می‌شود که دانشی که فرد تازه‌کار در اختیار ندارد را دارا می‌باشد\مرجع{ExpertWi84:online}. اگر تعاریف مختلف «خبرگی» را بررسی کنیم می‌بینیم که همه‌ی تعاریف در واقع تعبیری از میزان کیفیت عمکرد عامل نسبت به دیگر عامل‌ها می‌باشد. این تعبیر کلی از «خبرگی» انگیزه‌ای شد که درصدد معرفی معیاری برآیم که در حالت کلی بتوان به کلیه‌ی تعاریف «خبرگی» قابل تعمیم باشد.

\begin{hypo}[خبرگی]\setstretch{\thebaselinestretch}\label{experties_theorem}
فرض می‌کنیم عامل $\mathcal{A}$ در محیط $\mathcal{E}$ در پی رسیدن به یک مجموعه اهداف
$\mathcal{G} \subseteq \{g_1, g_2, \cdots, g_n\}$
می‌باشد. میزان خبرگی عامل رابطه‌ی معکوسی با میزان تلاش عامل برای رسیدن به اهداف تعریف شده خود دارد.
\end{hypo}

طبق آنچه که در فرضیه بالا آورده شده است از بین چند عاملی که در یک محیط و یک مجموعه از اهداف فعالیت می‌کنند، عاملی خبره‌تر است که تلاش کمتری برای رسیدن به آن مجموعه اهداف می‌کند. شاید این مساله در نگاه اول نامتعارف به ذهن برسد ولی در فعالیت‌های روزمره ما انسان‌ها نیز به کررات شاهد این امر می‌باشیم. به عنوان مثال رانندگی دو فرد مبتدی و حرفه‌ای را در نظر بگیریم؛ فرد مبتدی هنگام رانندگی تمام حواس خود را معطوف به رانندگی می‌کند تلاش بسیار زیادی برای کنترل نسبت میزان کلاچ و گاز می‌کند و هنگام رانندگی به طور طبیعی رانندگی نمی‌کند و ... ولی فرد خبره کلیه موارد ذکر شده را بطور خودکار و طبیعی انجام می‌دهد بطوری که انگار رانندگی مانند دیگر رفتارهای طبیعی وی چون نفس کشیدن می‌باشد، که بصورت خودکار صورت می‌پذیرد. از این گونه مثال‌ها از کاربرد فرضیه
\ref{experties_theorem}
در زندگی روزمره ما زیاد می‌توان یافت.\بند
توجه شود که در فرضیه
\ref{experties_theorem}
عبارت «میزان تلاش» عامل می‌تواند در کاربردهای مختلف تعابیر مختلفی به خود بگیرد، مثلا در مثال راننده‌ی مبتدی و خبره میزان نسبت مسافت طی شده بر زمان رانندگی را می‌توان به عنوان «میزان تلاش» عامل در نظر گرفت که در شرایط یکسان راننده‌ی خبره‌تر به طور نسبی در زمان کوتاه‌تری یک مسافت مشخصی را طی خواهد کرد (در رد کردن پیچ و خم‌های ترافیک و مدت زمان ترمز و ... زمان کمتری را تلف می‌کند). یا به عنوان مثال دیگر، دانشجوی قوی و دانشجوی ضعیف را مورد بررسی قرار دهیم، دانشجویی خبره هست که زمان کمتری را صرف حل صحیح یک مساله خاص کند (با فرض اینکه دانشجوها حتما باید مساله را حل کنند). همانطور که دیدیم کمیت «میزان تلاش» عامل برای مسائل مختلف معیار متفاوتی را دربر می‌گیرد ولی همگی از همان اصل معرفی شده در فرضیه
\ref{experties_theorem}
تبعیت می‌کنند.\بند

در یادگیری مشارکتی با استفاده از فرضیه
\ref{experties_theorem}
می‌توان با تعریف
\ref{experties_definition}
یک معیار خبرگی جدید را معرفی کرد که مبنی و پایه‌ی دستآوردهای این پژوهش می‌باشد.

\begin{definition}[معیار خبرگی «میزان ارجاع»]\setstretch{\thebaselinestretch}\label{experties_definition}
فرض می‌کنیم مجموعه‌ای از عامل‌ها
$\mathbb{A} = \{\mathcal{A}_1, \mathcal{A}_2, \cdots, \mathcal{A}_m\}$
در محیط $\mathcal{E}$ در پی رسیدن به یک مجموعه اهداف
$\mathcal{G} \subseteq \{g_1, g_2, \cdots, g_n\}$
می‌باشند. اگر ما به طور مجازی و دلخواه محیط $\mathcal{E}$ را به $k$ ناحیه‌ مانند $e_i$ افراز کنیم بطوری که
$\mathcal{E} = \{\cup_{i=1}^{k} e_i\hspace{5pt}|\hspace{5pt}\forall i,j\in\{1,2,\cdots,k\} \land i \neq j : e_i \cap e_j = \varnothing\}$
میزان ارجاع هر عامل در هر ناحیه را میزان حضور آن عامل را در آن ناحیه تعریف می‌کنیم.
\end{definition}

در تشریح آنچه که در تعریف
\ref{experties_definition}
آمده است می‌توان گفت که در سیستم‌های چندعاملی که همگی عوامل در یک محیط به صورت مستقل در حال فعالیت هستند؛ محیط را به چند ناحیه دلخواه افراز می‌کنیم که اجتماع نواحی باهم کل محیط $\mathcal{E}$ را تشکیل دهند و هیچ دو ناحیه‌ای اشتراکی باهم نداشته باشند\مرجع{schechter1996handbook}. در این چنین افرازی از محیط، در هرناحیه عاملی که نسبت به بقیه خبره‌تر است، نسبت به بقیه عوامل در همان ناحیه میزان تمایل حضور کمتری را از خود نشان می‌دهند. به عبارت دیگر عاملی که خبره‌تر است تمایل دارد کوتاه‌ترین مسیر رسیدن به اهداف خود را طی کند که نهایتا منجر خواهد شد که میزان حضور عامل در هریک از نواحی محیط کمینه شود.\بند
آنچه که در فرضیه
\ref{experties_theorem}
در مورد «میزان تلاش» عامل آمده است در تعریف
\ref{experties_definition}
در به صورت «میزان حضور عامل در هر ناحیه» تعریف شده است. بطوری که طبق فرضیه مطرح شده میزان خبرگی عامل در هر ناحیه رابطه‌ی معکوسی با میزان حضور عامل در همان ناحیه را دارد. زیرا اگر عامل نسبت به محیط خود شناخت کامل‌تری داشته در هنگام تلاش برای رسیدن به اهداف خود به علت شناخت خوبی که از محیط دارد کمتر در محیط پرسه می‌زند (کمتر تلاش می‌کند) و با تعداد گام کمتری به سمت اهداف خود حرکت می‌کند -- در واقع مسیر بهتری/کوتاه‌تری برای رسیدن به هدف را می‌شناسد. این موضوع در نهایت منجر می‌شود که عاملی که در هر ناحیه خبره‌تر است در همان ناحیه میزان پرسه زدن (حضور/تلاش) کمتری نسبت به دیگر عامل‌ها که از خبرگی نسبی کمتری برخوردار است را داشته باشد.

معیار تعریف شده در تعریف
\ref{experties_definition}
قبلا به صورت جزیی توسط احمدآبادی و همکاران\مرجع{ahmadabadi2000expertness} ارائه شده است ولی معیار تعریف شده در این پژوهش تفاوت‌هایی با معیار احمدآبادی و همکاران دارد که به شرح زیر است:
\begin{enumerate}
\فقره \textbf{چهارچوب:} تعریف خبرگی ارائه شده در این پژوهش (تعریف
\ref{experties_definition})
براساس چهارچوبی است که در فرضیه‌ی
\ref{experties_theorem}
آورده شده است، ولی معیار خبرگی احمدآبادی و همکاران براساس هیچ چهارچوبی تعریف شده است.
\فقره \textbf{میانگین تعداد قدم‌ها:} احمدآبادی و همکاران میانگین تعداد قدم‌های رسیدن به هدف(یا طبق تعریف
\ref{experties_definition}
میانگین میزان ارجاع عامل در کل محیط -- در زمانی که کل محیط را یک ناحیه در نظر بگیریم) را به عنوان معیار خبرگی در نظر گرفته‌اند در حالی که در تعریف
\ref{experties_definition}
حرفی از میانگین آورده نشده است.  ایرادی که معیار احمدآبادی و همکاران دارد این است که هنگامی که می‌خواهیم خبرگی عامل‌ها را بسنجیم صحیح نیست میانگین تعداد گام‌ها در نظر بگیرم زیرا ممکن است عامل در ابتدا بسیار نادان بوده ولی بعد از طی مدتی به وسیله‌ی تجاربی خاص به عاملی بسیار دانا تبدیل شود و اگر میانگین‌گیری صورت گیرد آنگاه نادانی گذشته به میزان خبرگی کنونی تاثیر گذاشته و خبرگی عامل کمتر از میزان واقعی تخمین زده شود. در تعریف
\ref{experties_definition}
خبرگی کنونی عامل مورد نظر است و کاری با مسیری که عامل برای کسب خبرگی کنونی‌اش طی کرده است نداریم.
\فقره \textbf{انعطاف:} معیار احمدآبادی و همکاران از انعطاف برخوردار نیست و در خبرگی عامل‌ها را بصورت میانگین خبرگی در کل محیط محاسبه می‌کند در حالی که طبق تعریف
\ref{experties_definition}
خبرگی عامل در نواحی مختلف از محیط قابل محاسبه است و همانطور که بعدها خواهیم دید خبرگی عامل‌ها در هر ناحیه به عنوان معیاری برای ترکیب دانش عامل‌ها نسبت به آن ناحیه مورد استفاده واقع خواهد؛ زیرا که عاملی ممکن است در حالت کلی محیط را آنچنان نشناخته باشد ولی در یک یا چند ناحیه بخصوص این عامل شناخت کامل‌تری از آن نواحی داشته باشد که معیار احمدآبادی و همکاران نمی‌تواند این مساله را در نظر بگیرد.
\end{enumerate}



تا به اینجا گفته شد که عاملی که از خبرگی بیشتری برخوردار است لزوما کمتر در محیط پرسه می‌زند و با طی کردن مسیر کوتاه‌تر به سمت اهداف خود، تلاش کمتری می‌کند ولی چند سوال در اینجا مطرح می‌شود که برای حل مساله نیازمند پاسخ به آن‌ها هستیم.
\begin{enumerate}
\فقره میزان حضور عامل را در نواحی مختلف، که محیط از $d$-بعد تشکیل شده است چگونه مدل شود؟
\فقره اگر عاملی که در هر چرخه یادگیری به یکی از نواحی کلا وارد نشد و میزان پرسه زدن عامل در آن ناحیه صفر شود؛ آیا این مقدار کمینه پرسه زدن، نشان دهنده‌ی خبرگی عامل در آن ناحیه است؟
\فقره چگونه در معیار خبرگی ارائه شده باید مساله عدم حضور عامل در یکی از نواحی را مدل کرد، بگونه‌ای که اثر سوئی بر تجربه‌ی دیگر عامل‌ها در آن نواحی، در هنگام ترکیب دانش عامل‌ها نداشته باشد؟
\end{enumerate}
پاسخ به این سوالات برای حل مساله با استفاده از معیار خبرگی پیشنهادی (تعریف \ref{experties_definition}) ضروری است. در پاسخ به سوال اول، ما به ازای کلیه‌ی نواحی یک ماتریسی به نام «ماتریس ارجاع» (یا به اختصار \رفمت\زیرنویس{\مق{Reference Matrix}}) در نظر میگیرم که در ابتدا صفر مقداردهی شده‌اند و هر دفعه که عامل از موقعیتی‌ به موقعیت دیگر می‌رود مقدار آن ناحیه‌ای که موقعیت جدید در آن واقع است را یک واحد افزایش می‌دهیم بدین وسیله میزان حضور عامل در نواحی مختلف را می‌شماریم. همانطور که در قسمت آزمایشات این پایان‌نامه نشان داده شده است میزان ریز یا درشت بودن این نواحی در کیفیت نتیجه تاثیرگذار نیست! یعنی عملا چه ما در حالت کلی، کل محیط را به عنوان یک ناحیه در نظر بگیریم و میزان حضور عامل در این ناحیه را بشماریم (که معادل می‌شود با تعداد گام‌های عامل در طی رسیدن به هدف) یا در حالت جزئی به ازای هر موقعیت موجود را یک ناحیه در نظر بگیرم (که معادل می‌شود با تعداد ملاقات هر یکی از موقعیت‌ها توسط عامل) به یک نتیجه می‌رسیم.\بند
به همین دلیل در پاسخ به سوال دوم، اگر تعداد نواحی زیاد باشد (مثلا هر موقعیت یک ناحیه باشد -- حداکثر تعداد نواحی) ممکن است عامل در طی رسیدن به هدف برخی از نواحی را کلا ملاقات نکند و مقدار ارجاع به آن نواحی صفر شود و از طرفی طبق تعریف \ref{experties_definition} عاملی که تعداد حضور کمتری در نواحی مختلف داشته باشد از خبرگی بیشتری در آن نواحی برخوردار است و در این شرایط که مقدار ارجاع عامل به ناحیه‌ای صفر باشد را نمی‌توان به خبرگی عامل در آن ناحیه نسبت داد زیرا که آن عامل در کل، آن ناحیه را ملاقات نکرده است که بخواهد تجربه‌ای را در تعامل با آن ناحیه کسب کند تا بتواند خبرگی خود را در آن ناحیه افزایش دهد. برای حل این مشکل و پاسخ به سوال سوم، ماتریسی جدیدی به نام ماتریس خاطره (یا به اختصار \رسمت\زیرنویس{Recall Matrix}) را معرفی می‌کنیم. این ماتریس وظیفه‌ی نگه‌داری آخرین ارجاعات غیر صفر عامل را به هرکدام از نواحی تعریف شده را دارد و در زمان‌هایی که مقدار یک ناحیه در ماتریس \رفمت\ صفر باشد مقدار آن ناحیه از ماتریس \رسمت\ بروز رسانی می‌شود که میزان پرسه زدن عامل در آن ناحیه در آخرین باری عامل آن ناحیه را ملاقات کرده است را می‌دهد؛ در صورتی که مقدار پرسه زدن یک ناحیه در ماتریس \رفمت\ مقداری غیر صفر باشد مقدار ماتریس \رسمت\ با مقدار کنونی \رفمت\ آن ناحیه بروز رسانی می‌شود.\بند
دلیل استفاده از ماتریس \رسمت\ این است که در یادگیری تقویتی عامل زمانی می‌توان دانش (سیاست/خبرگی) خود را نسبت به نحوه‌ی عمل در یک موقعیت بهبود ببخشد که آن موقعیت را ملاقات کند. حال اگر عامل موقعیتی را ملاقات نکند دانش وی در آن موقعیت ثابت خواهد ماند به همین دلیل اگر عامل ناحیه‌ای را ملاقات نکند و مقدار \رفمت\ آن ناحیه صفر باشد می‌دانیم که دانش (خبرگی) عامل در آن ناحیه در این چرخه‌ی یادگیری ثابت مانده است و در صورتی که دوباره در آن ناحیه قرار می‌گرفت، \زیرخط{حدودا} به همان میزان آخرین ملاقات در آن محیط پرسه خواهد زد. به عبارت دیگر در یک چرخه یادگیری اگر هر ناحیه ملاقات نشده، مورد ملاقات واقع می‌شد، تقریبا به میزان آخرین تعداد ارجاع شده برای آن نواحی، مورد ارجاع واقع می‌شد.