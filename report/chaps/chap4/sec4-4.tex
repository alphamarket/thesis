\قسمت{علت کارکرد انتگرال فازی چوکت در انتقال دانش}
\label{sec:why_fci_works}
در این قسمت به بررسی شهودی اینکه چرا انتگرال فازی چوکت برای انتقال(ترکیب) دانش‌های عامل‌ها می‌تواند موثر واقع باشد می‌پردازیم. این شهود بعدها در آزمایش‌ها نشان داده خواهد شد که صحت دارد. انتگرال فازی چوکت یک سری ویژگی‌ها دارد که برای مدل کردن انتقال دانش آن را کاندیدای مناسبی می‌کند. از مهم‌ترین ویژگی‌ها می‌توان به موارد زیر اشاره کرد \مرجع{grabisch1995fuzzy}.
\begin{enumerate}
\فقره \textbf{محدود است:} اگر شرایط مرزی تابع $g(\cdot)$ برقرار باشد انتگرال‌ فازی هیچ‌گاه بیشتر از حداکثر مقدار $f(x_{\pi_i})$ها و کمتر از حداقل مقدار آن‌ها خروجی نمی‌دهد\مرجع{murofushi1994non}. یعنی دانش تولیدی خارج از محدوده‌ی دانش فعلی عامل‌ها نمی‌باشد فقط ترکیب مناسبی از این دانش‌ها به عنوان خروجی برگشت داده می‌شود که این در کاربرد یادگیری تقویتی به این معنی است که هیچ‌گاه مقادیر جدول $Q$ بیشتر یا کمتر از آنچه که تجربه شده نمی‌شود؛ در نتیجه در صورت کران‌دار بودن پاداش‌های دریافتی از محیط جدول Q خروجی انتگرال فازی نیز کران‌دار است که نتیجه می‌دهد الگوریتم پیشنهادی حتما همگرا خواهد \کادربی{شد\مرجع{thesis:pakizeh2013multi}}.
\فقره \textbf{می‌تواند اندازه‌گیری‌های غیرافزایشی را مدل کند:} معمولا روش‌هایی که تا ‌کنون در این زمینه ارائه شده است از میانگین وزنی خبرگی عامل‌ها برای بدست آوردن جدول $Q$ مشترک استفاده کرده‌اند\مرجع{pakizeh2013multi, mohammad2015speedup, ahmadabadi2000expertness}. این در حالی هست که میانگین وزن‌دار قسمتی از مدل اندازه‌گیری‌ غیرافزایشی می‌باشد. بنابراین با درنظر گرفتن مدل‌های غیرافزایشی که در ماهیت مساله هست قدرت و انعطاف بیشتری نسبت به روش‌هایی که فقط از میانگین وزنی استفاده کرده‌اند، در اختیار داریم.
\end{enumerate}

\زیرقسمت{اثبات همگرایی روش پیشنهادی}
در مورد همگرایی روش پیشنهادی می‌توان گفت که از آنجایی که در خطوط
\ref{alg:proposed:q_start} تا \ref{alg:proposed:q_end}
الگوریتم پیشنهادی
\ref{alg:proposed}
که یادگیری تقویتی Q بدون دخل و تصرفی آورده شده است، اثبات همگرایی یادگیری تقویتی عامل‌ها به قوت خود باقی است\مرجع{watkins1992q}. حال باید نشان دهیم که ترکیب یادگیری تقویتی با انتگرال فازی همگرایی یادگیری تقویتی را برهم نمی‌زند.

\begin{proof}\large
فرض می‌کنیم که به تعداد $l$ عدد عامل وجود دارد که در هر مرحله‌ی یادگیری مشارکتی در زمان $t$، مقادیر $Q(s, a)$ عامل‌ها به ترتیب میزان بهینگی مقادیر آن‌ها به صورت زیر می‌باشند.
\begin{equation}\label{eq:proof:q_1_l}
Q_{\pi_1}^t(s, a) \preceq Q_{\pi_2}^t(s, a) \preceq \cdots \preceq Q_{\pi_l}^t(s, a) \preceq Q^*(s, a),\hspace{.25cm} \forall s, a, t
\end{equation}

که $\pi_1$ اندیس عاملی است که نسبت به دیگر عامل‌ها دارای مقدار $Q(s, a)$ با کمترین بهینگی می‌باشد و $\pi_l$ اندیس بهینه‌ترین مقدار $Q(s, a)$ است و عملگر $\preceq$ به معنی «کمتر یا مساوی بودن از دیدگاه بهینگی» می‌باشد و $Q^*(s, a)$ مقدار سیاست بهینه‌ در $(s, a)$ می‌باشد. طبق اثبات همگرایی یادگیری تقویتی (بدون در نظر گرفتن انتگرال فازی) زمانی که پاداش‌های دریافتی محیطی محدود و نرخ یادگیری محدود به بازه‌ی $[0, 1)$ باشد آنگاه داریم\مرجع{watkins1992q}:
\begin{equation}\label{eq:proof:q_t}
Q_{\pi_i}^t(s, a) \rightarrow Q^*(s, a)\hspace{.5cm}\text{as}\hspace{.25cm}t \rightarrow \infty,\hspace{.5cm}\forall s, a, i \in [1 \cdots l]
\end{equation}

با توجه به اینکه که انتگرال فازی به ازای هر $(s, a)$ مقداری مابین حداکثر و حداقل مقادیر جداول Q عامل‌ها در آن $(s, a)$ را تولید می‌کند، طبق خاصیت محدود بودن انتگرال فازی چوکت می‌توانیم بگوییم که دانش تولیدی حاصل از ترکیب جداول $Q_i, \forall i \in [1 \cdots l]$ عامل‌ها در زمان $t$ به جدول $Q_\text{REFMAT}^t$ می‌رسیم که دارای خاصیت زیر است.
\begin{equation}\label{eq:proof:q_refmat}
Q_{\pi_1}^t(s, a) \preceq Q_\text{REFMAT}^t(s, a) \preceq Q_{\pi_l}^t(s, a),\hspace{.25cm} \forall s, a, t  
\end{equation}

از
\ref{eq:proof:q_1_l} تا \ref{eq:proof:q_refmat} 
می‌توان به این نتیجه رسید از آنجایی که
$Q_{\pi_1}^t(s, a)$
به ازای هر $t$ دلخواه و با توجه به شرایط ذکر شده در اثبات همگرایی\مرجع{watkins1992q} نهایتا همگرا می‌شود و همچنین
$Q_{\pi_l}^t(s, a)$
نیز در $t \rightarrow \infty$ همگرا می‌شود و کلیه‌ی جداول مابین این دو یعنی
$Q_{\pi_i}^t(s, a), \forall s, a, t, i \in [1 \cdots l]$
نیز همگرا می‌شوند
(\ref{eq:proof:q_1_l} و \ref{eq:proof:q_t})\
؛ در نتیجه جدول تولیدی توسط انتگرال فازی چوکت که خاصیت \ref{eq:proof:q_refmat} در آن برقرار می‌باشد نیز در $t \rightarrow \infty$ همگرا خواهد شد.
\end{proof}

