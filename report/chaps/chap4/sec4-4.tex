\قسمت{علت کارکرد انتگرال فازی چوکت در انتقال دانش}
در این قسمت به بررسی شهودی اینکه چرا انتگرال فازی چوکت برای انتقال(ترکیب) دانش‌های عامل‌ها می‌تواند موثر واقع باشد می‌پردازیم. این شهود بعدها در آزمایش‌ها نشان داده خواهد شد که صحت دارد. انتگرال فازی چوکت یک سری ویژگی‌ها دارد که برای انتقال دانش مدل می‌کند. از مهم‌ترین ویژگی‌ها را می‌توان به موارد زیر اشاره کرد \مرجع{grabisch1995fuzzy}.
\begin{enumerate}
\فقره \textbf{محدود است:} اگر شرایط مرزی و یکنوایی تابع $g(\cdot)$ برقرار باشد انتگرال‌ فازی هیچ‌گاه بیشتر از حداکثر مقدار $f(x_{\pi_i})$ها و کمتر از حداقل مقدار آن‌ها خروجی نمی‌دهد\مرجع{murofushi1994non}. یعنی دانش تولیدی خارج از محدوده‌ی دانش فعلی عامل‌ها نمی‌باشد فقط ترکیب بهینه‌ای از این دانش‌ها به عنوان خروجی برگشتت داده می‌شود که این در کاربرد یادگیری تقویتی به این معنی است که هیچ‌گاه مقادیر جدول $Q$ بیشتر یا کمتر از آنچه که تجربه شده نمی‌شود و این باعث می‌شود که ضمانت همگرایی یادگیری تقویتی $Q$ با اعمال انتگرال فازی چوکت نقض نشود و الگوریتم حتما همگرا شود؛ ولی در صورتی که روشی خارج از دانش کنونی عامل‌ها خروجی دهد ضمانتی برای همگرایی عامل‌ها وجود نخواهد داشت.
\فقره \textbf{می‌تواند اندازه‌گیری‌های غیرافزایشی مدل کند:} معمولا روش‌هایی که تا به‌کنون در این زمینه ارائه شده است از میانگین وزنی خبرگی عامل‌ها برای بدست آوردن جدول $Q$ مشترک استفاده کرده‌اند\مرجع{pakizeh2013multi, mohammad2015speedup, ahmadabadi2000expertness}. این درحال هست که میانگین وزن‌دار قسمتی از مدل اندازه‌گیری‌های غیرافزایشی می‌باشد. بنابرین با درنظر گرفتن مدل‌های غیرافزایشی که در ماهیت مساله هست قدرت و انعطاف بیشتری در اختیار داریم نسبت به روش‌هایی که فقط از میانگین وزنی استفاده کرده‌اند.
\end{enumerate}

\begin{definition}[اندازه‌گیری‌های غیرافزایشی]\setstretch{\thebaselinestretch}\label{non_additive_definition}
اگر فرض کنیم $(X, A)$ فضای قابل اندازه‌گیری باشد که $X$ مجموعه‌ی مرجع\زیرنویس{\مق{Reference Set}} و $A \subseteq X$، آنگاه تابع مجموعه‌ای مانند $\mu$ که $\mu: A \rightarrow [0, 1]$ اندازه‌گیر غیرافزایشی می‌گویند هرگاه شرایط زیر را ارضا کند\مرجع{torra2014non}.

\begin{latin}
\begin{itemize}
\item $\mu(\emptyset) = 0, \hspace{10pt} \mu(X) = 1$
\item $A \subseteq B \Rightarrow \mu(A) \leq \mu(B)$
\end{itemize}
\end{latin}
\end{definition}

تورا و همکاران\مرجع{torra2014non} یک مجموعه جامعی در مورد اندازه‌گیری‌های غیرافزایشی ایجاد کرده‌اند که جزئیات این مطلب خارج از حوصله‌ی این نوشتار است و در صورت تمایل به کسب اطلاعات بیشتر در مورد اندازه‌گیری‌های غیرافزایشی و انتگرال‌های فازی می‌توانید به آن مراجعه نمایید.

