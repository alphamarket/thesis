\قسمت{علت کارکرد انتگرال فازی چوکت در انتقال دانش}
\label{sec:why_fci_works}
در این قسمت به بررسی شهودی اینکه چرا انتگرال فازی چوکت برای انتقال(ترکیب) دانش‌های عامل‌ها می‌تواند موثر واقع باشد می‌پردازیم. این شهود بعدها در آزمایش‌ها نشان داده خواهد شد که صحت دارد. انتگرال فازی چوکت یک سری ویژگی‌ها دارد که برای مدل کردن انتقال دانش آن را کاندیدای مناسبی می‌کند. از مهم‌ترین ویژگی‌ها می‌توان به موارد زیر اشاره کرد \مرجع{grabisch1995fuzzy}.
\begin{enumerate}
\فقره \textbf{محدود است:} اگر شرایط مرزی تابع $g(\cdot)$ برقرار باشد انتگرال‌ فازی هیچ‌گاه بیشتر از حداکثر مقدار $f(x_{\pi_i})$ها و کمتر از حداقل مقدار آن‌ها خروجی نمی‌دهد\مرجع{murofushi1994non}. یعنی دانش تولیدی خارج از محدوده‌ی دانش فعلی عامل‌ها نمی‌باشد فقط ترکیب مناسبی از این دانش‌ها به عنوان خروجی برگشت داده می‌شود که این در کاربرد یادگیری تقویتی به این معنی است که هیچ‌گاه مقادیر جدول $Q$ بیشتر یا کمتر از آنچه که تجربه شده نمی‌شود؛ در نتیجه در صورت کران‌دار بودن پاداش‌های دریافتی از محیط جدول Q خروجی انتگرال فازی نیز کران‌دار است که نتیجه می‌دهد الگوریتم پیشنهادی حتما همگرا خواهد \کادربی{شد\مرجع{thesis:pakizeh2013multi}}.
\فقره \textbf{می‌تواند اندازه‌گیری‌های غیرافزایشی را مدل کند:} معمولا روش‌هایی که تا ‌کنون در این زمینه ارائه شده است از میانگین وزنی خبرگی عامل‌ها برای بدست آوردن جدول $Q$ مشترک استفاده کرده‌اند\مرجع{pakizeh2013multi, mohammad2015speedup, ahmadabadi2000expertness}. این در حالی هست که میانگین وزن‌دار قسمتی از مدل اندازه‌گیری‌ غیرافزایشی می‌باشد. بنابراین با درنظر گرفتن مدل‌های غیرافزایشی که در ماهیت مساله هست قدرت و انعطاف بیشتری نسبت به روش‌هایی که فقط از میانگین وزنی استفاده کرده‌اند، در اختیار داریم.
\end{enumerate}

تورا و همکاران\مرجع{torra2014non} یک مجموعه جامعی در مورد اندازه‌گیری‌های غیرافزایشی ایجاد کرده‌اند که جزئیات این مطلب خارج از حوصله‌ی این نوشتار است و در صورت تمایل به کسب اطلاعات بیشتر در مورد اندازه‌گیری‌های غیرافزایشی و انتگرال‌های فازی می‌توانید به آن مراجعه نمایید.

