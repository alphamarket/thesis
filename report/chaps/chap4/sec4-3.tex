\قسمت{یادگیری مشارکتی $Q$ با استفاده از ماتریس ارجاع و انتگرال فازی}
آنچه که تا به اکنون در مورد روش پیشنهادی این پژوهش آورده شده، معرفی یک معیار خبرگی که در برعکس بسیاری از معیارهای خبرگی که تا به کنون معرفی شده است\مرجع{ahmadabadi2000expertness, pakizeh2013multi, mohammad2015speedup} در تمامی موقعیت‌های دنیای واقعی به وفور مشاهده می‌شود و آن ارائه این تئوری است عامل خبره‌تر برای رسیدن به یک مجموعه از اهداف تلاش نسبی کمتری نسبت به دیگر عامل‌ها با خبرگی کمتر در شرایط یکسان می‌کند. حال که معیاری برای میزان خبرگی عامل‌ها در اختیار داریم چالش بعدی برای بهبود کیفیت و سرعت یادگیری مشارکتی ارائه‌ی روشی برای ترکیب دانش‌های عامل‌ها از محیط (جداول $Q$ آن‌ها) با استفاده از معیار ارائه شده می‌باشد. روش ترکیب باید بگونه‌ای باشد که کیفیت و سرعت یادگیری مشارکتی عامل‌ها را در طی زمان نسبت زمانی که عامل‌ها بدون مشارکت یاد می‌گیرند بهتر کند. همچنین کیفیت و سرعت یادگیری همبستگی مستقیمی داشته باشند با تعداد عامل‌هایی که درحال اشتراک گذاری هستند؛ به عبارت دیگر در صورت افزایش تعداد عامل‌هایی که دانش‌های خود را به اشتراک می‌گذارند مدل ترکیب کننده‌ی دانش‌های آن‌ عامل‌ها باید بتواند دانش‌ بهتری تولید کند که نهایتا منجر به بهتر شدن کیفیت و سرعت کلی یادگیری عامل‌ها شود.\بند
در این پژوهش ما انتگرال فازی را به عنوان مدل ترکیب کننده‌ی دانش‌های عامل‌ها پیشنهاد می‌دهیم. دلیل انتخاب این مدل ویژگی‌های منحصر به فردی است که این مدل کننده در اختیار دارد که مدل را کاملا مناسب برای ترکیب دانش‌ عامل‌ها می‌کند؛ که در بخش‌های آتی این فصل بطور مفصل توضیح می‌دهیم که این ویژگی‌ها چه هستند و چرا این ویژگی‌ها برای ترکیب دانش عامل‌ها مناسب هستند.

\زیرقسمت{الگوریتم پیشنهادی}
نپن
\زیرقسمت{چرخه یادگیری مستقل}

\زیرقسمت{چرخه همکاری}
