\addcontentsline{toc}{chapter}{چکیده انگلیسی}
\thispagestyle{empty}

\begin{latin}
\begin{center}

{\Huge Improvments in speed and quality of learning in multi-agent systems using a novel expertness measurement and fuzzy integral}

\vspace{1cm}

{\LARGE{Dariush Hasanpour Adeh}}

\vspace{0.2cm}

{\small d.hasanpoor@ec.iut.ac.ir}

\vspace{0.5cm}

Fall 2016

\vspace{0.5cm}

Department of Electrical and Computer Engineering

\vspace{0.2cm}

Isfahan University of Technology, Isfahan 84156-83111, Iran

\vspace{0.2cm}

Degree: M.Sc. \hspace*{3cm} Language: Farsi

\vspace{1cm}

{\small\textbf{Supervisor: Assoc. Prof. Maziar Palhang (palhang@cc.iut.ac.ir)}}
\end{center}
~\vfill



\noindent\textbf{Abstract}

\begin{small}
\bgroup
\baselineskip=0.6cm
In the real world, usually, peoples are coming together for sharing their knowledge and talking from their good and bad experiences and more or less everybody has something to say. Although we cannot ignore anybody's knowledge but it's common sense to assign more weight on the most experienced person's knowledge when we are going to decide what we need to do based on consultation from people. The achievements of this research have the same philosophy, that everybody needs to be heard, which is freedom of expression! Fuzzy integrals are one of the most powerful and flexible methods to demonstrate the freedom of expression. So we have used the fuzzy integrals for hearing everybody's knowledge and extract a knowledge which is useful for everybody.\\
\indent One of the challenges is that how to fairly answer the "what is the agents' expertise and how to determine the most and least expert agent?" question. To answer this question, in this thesis, we have proposed «the theory of expertness» which defines a framework for "expertness criteria" definitions, and based on this framework we have introduced a new expertness criteria and showed that the defined framework and criteria are much more efficient that the state of the art criteria "Shortest Experienced Path". Also, the power of using fuzzy integrals for intelligence aggregation is demonstrated.
\egroup
\end{small}

\vspace{0.5 cm}

\noindent \textbf{Key Words:}\\ Multi-agent Systems, Cooperative Learning, Reinforcement Learning, Non-additive Knowledges, Fuzzy Integral 

\end{latin}