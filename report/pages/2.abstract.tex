%*************************************************
% In this file the abstract is typeset.
% Make changes accordingly.
%*************************************************

\addcontentsline{toc}{section}{چکیده}
\newgeometry{left=2.5cm,right=3cm,top=3cm,bottom=2.5cm,includehead=false,headsep=1cm,footnotesep=.5cm}
\setcounter{page}{1}
\pagenumbering{arabic}						% شماره صفحات با عدد
\thispagestyle{empty}

~\vfill

\subsection*{چکیده}
\begin{small}
\baselineskip=0.7cm
\vspace*{0.5 cm}

معمولا در دنیایی واقعی هنگامی که افراد برای انتقال دانش گرد هم می‌آیند و از تجربیات خوب و بد گذشته خود سخن می‌گویند هرکسی متناسب با جایگاهی که دارد دارای دانشی می‌باشد و در این انتقال دانش‌ها تجربیات هیچ کسی را نمی‌توان نادیده گرفت ولی گاها پیش می‌آید که تجربیات و دانش فردی دارای بار محتویاتی بیشتری نسبت به اطرافیان خود می‌باشد، مردم معمولا از دانش فرد خبره‌تر بیشتر بهره می‌برند تا افراد دیگر. دستاورد‌های این پژوهش بر مبنای همین فلسفه بنا شده است که سخن و دانش هرکسی باید شنیده شود. انتگرال فازی یکی از قوی‌ترین و منعطف‌ترین ابزارهای ریاضی برای ترکیب اطلاعات می‌باشد، لذا در این پژوهش از انتگرال فازی برای شنیدن بازتاب ندای دانش هر عامل در دانش جمعی استفاده شده است. ولی در این راه مشکلاتی نیز وجود داشت و آن این بود که چگونه منصفانه بفهمیم که کدام عامل خبره‌تر از دیگری می‌باشد؟ در گذشته روش‌های متنوعی برای تخمین این معیار ارائه شد است که از شمارش میزان پاداش‌های مثبت و منفی عامل‌ها گرفته تا محاسبات پیچیده‌ای چون معیار‌های شوک و کوتاه‌ترین مسیر تجربه شده. در طی پژوهش که منجر به نگارش این پایان‌نامه گردید احساس شد که تمامی روش‌های قبلی در یک چیز مشترکند: بسیار پیچیده و غیر منعطف!

وجود این فصل مشترک ناکارا انگیزه‌ای شد که در صدد ارائه‌ای معیاری برآیم که نه تنها ساده باشد بلکه در زندگی روزمره ما انسان‌ها هم تجلی داشته باشد. در پی این هدف ما به ارائه‌ی تئوری جامعی برای خبرگی پرداختیم که می‌تواند منشع بسیاری از تعاریف خبرگی، در آینده گردد؛ نهایتا با استفاده از تئوری خبرگی معرفی شده تعریفی برای یک معیار خبرگی جدید ارائه دادیم و نشان دادیم که تئوری و تعریف خبرگی جدید نسبت به تعاریف قبلی بسیار کارآمد بوده است.

\noindent\textbf{واژه‌های کلیدی:}
۱- سیستم‌های چندعامله، ۲- یادگیری مشارکتی، ۳- یادگیری تقویتی، ۴- دانش غیرافزایشی، ۵- انتگرال فازی.
\end{small}