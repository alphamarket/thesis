%*************************************************
% In this file the abstract is typeset.
% Make changes accordingly.
%*************************************************

\addcontentsline{toc}{section}{چکیده}
\newgeometry{left=2.5cm,right=3cm,top=3cm,bottom=2.5cm,includehead=false,headsep=1cm,footnotesep=.5cm}
\setcounter{page}{1}
\pagenumbering{arabic}						% شماره صفحات با عدد
\thispagestyle{empty}

~\vfill

\subsection*{چکیده}
\begin{small}
\baselineskip=0.7cm
معمولا محیط‌هایی که ربات‌ها و بخصوص پهپادها در آن فعالیت دارند برای ربات‌ها و گاها برای انسان‌ها محیط‌ ناشناخته‌ای می‌باشد، دنیای مدرن که به سمت طراحی و توسعه ربات‌های خودمختار حرکت می‌کند، راهبری و \جام به دلیل ایفای نقش بسیار مهم در موفقیت ربات‌های خودمختار، به عنوان یکی از چالش‌های مهم و هیجان انگیز در جوامع دانشگاهی و صنعتی شناخته شده است. برای اینکه ربات بتواند از موقعیت اولیه به موقعیت نهایی بدون برخورد با موانع موجود در محیط اطراف خود حرکت کند، اهمیت طرح‌ریزی حرکت بیش از پیش به چشم می‌آید؛ زیرا که برای طی مسیری بدون برخورد با موانع موجود در آن، ربات باید علاوه بر دارا بودن سیستمی بجهت طرح‌ریزی مناسب مسیر، به سیستم شناسایی و \جام مجهز باشد. در این میان پهپادها که به صورت معمول در مسائل مهم، از قبیل نظامی، امداد و نجات، شناسایی و نظارت و غیره مورد استفاده واقع می‌شوند و از طرفی دیگر ساخت آن‌ها هزینه‌بر می‌باشد، بنابرین نیاز به داشتن سیستمی برای تشخیص و \جام بیش از دیگر ربات‌ها احساس می‌شود.\بند

این پژوهش با تمرکز به ربات‌های خانواده چندپره‌ها به ارائه‌ی روشی نوین برای \جام برمبنای ترکیب اطلاعات عمقی از تصاویر استریو و حسگرهای فراصوتی پرداخته است و نهایتا پیاده‌سازی‌های انجام شده نشان می‌دهد که روش پیشنهادی می‌تواند به عنوان سیستمی برای تشخیص و اجتناب از مانع برخط برای پهپادهایی مجهز به حداقل سخت‌افزار مورد نیاز مورد استفاده واقع گردد.
\vspace*{0.5 cm}

\noindent\textbf{واژه‌های کلیدی:}
۱- پهپاد، ۲- امنیت پرواز، ۳-اجتناب از موانع.
\end{small}